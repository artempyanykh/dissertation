\documentclass[12pt]{beamer}
\usepackage[T2A]{fontenc}
\usepackage[utf8]{inputenc}
\usepackage[english,russian]{babel}
\usepackage{amssymb,amsfonts,amsmath,mathtext,amsthm}
\usepackage{cite,enumerate,float,indentfirst}
\usepackage{booktabs}
\usepackage{tikz}
\usetikzlibrary{positioning}
\usetikzlibrary{arrows}
\usetikzlibrary{automata}

\graphicspath{{../images/}{images/}} 

\usetheme[secheader]{Boadilla}
\usecolortheme{seahorse}

% \usetheme{Pittsburgh}
% \usecolortheme{whale}

\beamertemplatenavigationsymbolsempty

\newcommand{\todo}{\alert}
%%% �������� �������� %%%
\newcommand{\thesisAuthor}             % �����������, ��� ������
{%
    \texorpdfstring{% \texorpdfstring takes two arguments and uses the first for (La)TeX and the second for pdf
        ������ ����� ��������% ��� ����� ������������ �� ��������� ����� ��� � ������, ��� ����� �������������� ����������
    }{%
        ������ ����� ��������% ��� ������ ��� ������� pdf-�����. � ����� ����, ���� pdf ����� ��������� ����������� ��� ����� ����������������� ��������, ����� ��������� ������������ �������.
    }%
}
\newcommand{\thesisAuthorShort}             % �����������, ��� ������ ����������
{�.�.~������}

\newcommand{\thesisUdk}                % �����������, ���
{519.83}
\newcommand{\thesisTitle}              % �����������, ��������
{\texorpdfstring{\MakeUppercase{� ������� ������������� ��� � �������� �����������, ������������ �������� �����}}{� ������� ������������� ��� � �������� �����������, ������������ �������� �����}}
\newcommand{\thesisSpecialtyNumber}    % �����������, �������������, �����
{\texorpdfstring{01.01.09}{01.01.09}}
\newcommand{\thesisSpecialtyTitle}     % �����������, �������������, ��������
{\texorpdfstring{���������� ���������� � �������������� �����������}{���������� ���������� � �������������� �����������}}
\newcommand{\thesisDegree}             % �����������, ������� �������
{��������� ������-�������������� ����}
\newcommand{\thesisCity}               % �����������, ����� ������
{������}
\newcommand{\thesisYear}               % �����������, ��� ������
{2016}
\newcommand{\thesisOrganization}       % �����������, �����������
{���������� ��������������� ����������� �����~�.�.����������}
\newcommand{\thesisOrganizationShort}  % �����������, ������� �������� ����������� ��� �������
{\todo{��� ��. �.�.����������}}

\newcommand{\thesisInOrganization}       % �����������, ����������� � ���������� ������: ������ ��������� � ...
{\todo{���������� ��������������� ������������ ����� �.�.����������}}

\newcommand{\supervisorFio}            % ������� ������������, ���
{������� �������� ����������}
\newcommand{\supervisorRegalia}        % ������� ������������, �������
{�������� ������-�������������� ����, ������}
\newcommand{\supervisorFioShort}            % ������� ������������, ���
{�.�.~�������}
\newcommand{\supervisorRegaliaShort}        % ������� ������������, �������
{�.�.-�.�.,~���.}


\newcommand{\opponentOneFio}           % �������� 1, ���
{\todo{������� ��� ��������}}
\newcommand{\opponentOneRegalia}       % �������� 1, �������
{\todo{������ ������-�������������� ����, ���������}}
\newcommand{\opponentOneJobPlace}      % �������� 1, ����� ������
{\todo{�� ����� ������� �������� ��� ����� ������}}
\newcommand{\opponentOneJobPost}       % �������� 1, ���������
{\todo{������� ������� ���������}}

\newcommand{\opponentTwoFio}           % �������� 2, ���
{\todo{������� ��� ��������}}
\newcommand{\opponentTwoRegalia}       % �������� 2, �������
{\todo{�������� ������-�������������� ����}}
\newcommand{\opponentTwoJobPlace}      % �������� 2, ����� ������
{\todo{�������� ����� ������ c ������� ������� ������� ������� ���������}}
\newcommand{\opponentTwoJobPost}       % �������� 2, ���������
{\todo{������� ������� ���������}}

\newcommand{\leadingOrganizationTitle} % ������� �����������, �������������� ������
{\todo{����������� ��������������� ��������� ��������������� ���������� ������� ����������������� ����������� �~������� ������� ������� ������� ���������}}

\newcommand{\defenseDate}              % ������, ����
{\todo{DD mmmmmmmm YYYY~�.~�~XX �����}}
\newcommand{\defenseCouncilNumber}     % ������, ����� ���������������� ������
{\todo{NN}}
\newcommand{\defenseCouncilTitle}      % ������, ���������� ���������������� ������
{\todo{�������� ����������}}
\newcommand{\defenseCouncilAddress}    % ������, ����� ���������� ���������������� ������
{\todo{�����}}

\newcommand{\defenseSecretaryFio}      % ��������� ���������������� ������, ���
{\todo{������� ��� ��������}}
\newcommand{\defenseSecretaryRegalia}  % ��������� ���������������� ������, �������
{\todo{�-�~���.-���. ����}}            % ��� ���������� ���� �����, ��������: ���� � 7.0.12-2011 + http://base.garant.ru/179724/#block_30000

\newcommand{\synopsisLibrary}          % �����������, �������� ����������
{\todo{�������� ����������}}
\newcommand{\synopsisDate}             % �����������, ���� ��������
{\todo{DD mmmmmmmm YYYY ����}}

% To avoid conflict with beamer class use \providecommand
\providecommand{\keywords}%                 % �������� ����� ��� ���������� PDF ����������� � ������������
{������������ ����, ������������� ����, �������� ����������, ������������� ����������, ������������ ��������}

%%% Local variables:
%%% coding: cp1251
%%% End:      % Основные сведения

\newcommand{\Co}{\beta}
\newcommand{\DCo}{\overline{\beta}}
\newcommand{\E}{\ensuremath{\mathbb{E}}}

\newtheorem{thm}{Теорема}[section]
\newtheorem{prop}{Утверждение}[section]

\setbeamercolor{footline}{fg=blue}
\setbeamertemplate{footline}{
  \leavevmode%
  \hbox{%
  \begin{beamercolorbox}[wd=.363333\paperwidth,ht=2.25ex,dp=1ex,center]{}%
    % И. О. Фамилия, Организация кратко
    \thesisAuthorShort, \thesisOrganizationShort
  \end{beamercolorbox}%
  \begin{beamercolorbox}[wd=.333333\paperwidth,ht=2.25ex,dp=1ex,center]{}%
    % Город, 20XX
    \thesisCity, \thesisYear
  \end{beamercolorbox}%
  \begin{beamercolorbox}[wd=.303333\paperwidth,ht=2.25ex,dp=1ex,right]{}%
  Стр. \insertframenumber{} из \inserttotalframenumber \hspace*{2ex}
  \end{beamercolorbox}}%
  \vskip0pt%
}

\newcommand{\itemi}{\item[\checkmark]}

%\title{\small{Название презентации}}
\title{\small{\thesisTitle}}
\author{\small{%
\emph{Выступающий:}~\thesisAuthorShort\\%
\emph{Руководитель:}~\supervisorRegaliaShort~\supervisorFioShort}\\%
\vspace{30pt}%
\thesisOrganization%
\vspace{20pt}%
}
\date{\small{\thesisCity, \thesisYear}}

\begin{document}

\maketitle

\section{Введение}

\begin{frame}
  \frametitle{Краткое описание модели}
  \begin{itemize}
  \item
    Между двумя игроками в течение $n \leqslant \infty$ шагов происходят
    торги за однотипные акции.
  \item
    Цена акции $s$ определяется ходом случая в соответствии с распределением $\mu$.
  \item
    Первый игрок (инсайдер) знает цену $s$, второй "--- знает только вероятностное распределение $\mu$.
  \item
    На каждом шаге торгов игроки делают ставки $i,\ j$.
    Игрок, предложивший большую ставку, покупает у другого акцию по цене $T(i, j)$.
    При равных ставках сделка не состоится.
  \end{itemize}
\end{frame}

\begin{frame}
  \frametitle{Непрерывная и дискретная постановки}
  
  \begin{tabular}{c|c|c}
    & Непрерывная модель & Дискретная модель \\
    \midrule
    Ставки & вещественные из $[0, 1]$ & вида $i/m,\ i = \overline{0, m}$ \\
    \hline \\
    Цена акции& \multicolumn{2}{c}{либо $0$, либо $1$; $P(s = 1) = p$} \\
    \hline \\
    Цена сделки& \multicolumn{2}{c}{$T(i, j) = \max(i, j)$} \\
    \hline \\
    $\{V_n(p)\}$ & растет как $\sqrt{n}$ & ограничена \\
    \bottomrule \\
  \end{tabular}

  \textbf{Цель:} показать возможность стратегического происхождения броуновского движения (случайного блуждания) в динамике цен активов.
\end{frame}

\begin{frame}
  \frametitle{Мотивация}
  
  Рассмотрение более общего симметричного механизма торгов $T(i, j)$ обозначено в работе [De Meyer,~Saley,~2002] как одно из актуальных направлений дальнейших исследований.
  
  В качестве такого механизма был выбран
  \[
    T(i, j) = \beta \max(i, j) + (1-\beta) \min(i, j),\ \beta \in [0, 1],
  \]
  предложенный в [Chatterjee,~Samuelson,~1983].
  
  Параметр $\beta$ можно интерпретировать как переговорную силу продавца.
\end{frame}

\section{Глава 1. Дискретная модель}

\begin{frame}
  \frametitle{Определение модели}
  
  \begin{itemize}
  \item 
  Множество состояний $S = \{L, H\}$.
  \item
  В состоянии $L$ цена $s=0$, в состоянии $H$ цена $s=m$.
  \item
  Игроки могут делать ставки из множества $\{0, 1, \ldots, m\}$. Обозначим множества действий первого и второго игроков через $I$ и $J$ соответственно.
  \end{itemize}
\end{frame}

\begin{frame}
  Модель отвечает повторяющейся игре $G^{m, \Co}_n(p)$ с платежными матрицами
  \begin{equation*}\footnotesize
    A^{L,\Co}(i, j) = \begin{cases}
      \DCo i + \Co j, &\, i < j, \\
      0, &\, i = j, \\
      -\Co i - \DCo j, &\, i > j,
    \end{cases}
    \qquad
    A^{H,\Co}(i, j) = \begin{cases}
      \DCo i + \Co j - m, &\, i < j, \\
      0, &\, i = j, \\
      m - \Co i - \DCo j, &\, i > j.
    \end{cases}
  \end{equation*}
  Стратегией первого игрока является последовательность ходов 
  \[
  \sigma = (\sigma_1, \sigma_2, \ldots, \sigma_t, \ldots),\ \sigma_t: S \times I^{t-1} \rightarrow \Delta(I).
  \]

  Стратегией второго игрока --- последовательность ходов 
  \[
  \tau = (\tau_1, \tau_2, \ldots, \tau_t, \ldots),\ \tau_t: I^{t-1} \rightarrow \Delta(J).
  \]

  Выигрыш задается как
  \begin{equation*}
    K^{m,\Co}_n(p, \sigma, \tau) = \E_{\Pi[p,\sigma,\tau]} \sum_{t=1}^n
    \left(
      pA^{H,\Co}(i_t^H, j_t) + (1 - p)A^{L,\Co}(i_t^L, j_t)
    \right).
  \end{equation*}
\end{frame}

\begin{frame}
  \frametitle{Оптимальная стратегия второго игрока}
  
  Следующая чистая стратегия $\tau^k$ предложена В.~К.~Доманским в [Domansky,~2007]:
  \[
    \tau^k_1 = k, \quad \tau^k_t(i_{t-1}, j_{t-1}) = \begin{cases}
      j_{t-1} - 1, & \, i_{t-1} < j_{t-1}, \\
      j_{t-1},     & \, i_{t-1} = j_{t-1}, \\
      j_{t-1} + 1, & \, i_{t-1} > j_{t-1}.
    \end{cases}
  \]
  
  Показано, что при $p \in (k - 1 + \Co, k + \Co),\ k = \overline{0, m}$ использование $\tau^k$ оптимально для второго игрока в игре $G^{m, \Co}_\infty(p)$.
\end{frame}

\begin{frame}
  \frametitle{Рекурсивная структура игры $G^{m, \Co}_n(p)$}
  
  Представим $\sigma$ как $(\sigma_1, \sigma(i),\ i \in I)$, а $\tau$ как $(\tau_1, \tau(i),\ i \in I)$.

  Параметризуем ход $\sigma_1$ инсайдера с помощью 
  \begin{itemize}
  \item
    полной вероятности $q(i)$ сделать ставку $i$
  \item
    апостериорной вероятности $p(H|i)$ состояния $H$ при использовании ставки $i$.
  \end{itemize}
  По формуле Байеса вероятность сделать ставку $i \in I$ в состоянии $s \in S$ выражается как
  \[
    \sigma^s_{1,i} = \frac{p(s|i)q_i}{p(s)}.
  \]

  Формулу выигрыша можно переписать в виде
  \begin{equation*}
    K^m_n(p, \sigma_1, \tau_1) = 
    K^m_1(p, \sigma, \tau) + 
    \sum_{i \in I} q(i) K^m_{n-1}(p(H|i), \sigma(i), \tau(i)).
  \end{equation*}
\end{frame}

\begin{frame}
  \frametitle{Оптимальная стратегия первого игрока}
  
  При $p = k/m,\ k = \overline{1, m-1}$ первый игрок рандомизирует выбор ставок $k, \, k+1$ с параметрами
  \begin{gather*}
    q_k = \Co, \quad p(H|k) = (k-1+\Co)/m,\\
    q_{k+1} = \DCo, \quad p(H|k + 1) = (k + \Co)/m.
  \end{gather*}
  При $p = (k+\Co)/m,\ k = \overline{0, m-1}$ --- с параметрами
  \begin{gather*}
    q_k = \DCo, \quad p(H|k) = k/m,\\
    q_{k+1} = \Co, \quad p(H|k + 1) = (k+1)/m.
  \end{gather*}
  
  Для остальных значений $p$ используется стратегия, дающая линейную комбинацию выигрышей в точках решетки.
\end{frame}

\begin{frame}
  Графически оптимальную стратегию первого игрока можно представить следующим образом:
  \begin{figure}[tb]
    \centering
    \begin{tikzpicture}
      [
      auto,node distance=1.5cm,
      trans/.style={->,shorten >=1pt,>=stealth',semithick},
      state/.style={shape=circle,draw,minimum size=1mm},
      ]
      \node[state,label={$0$}] (p0) {};
      \node[state,right=of p0,label={$\frac{\Co}{m}$}] (p1) {}; 
      \node[state,right=of p1,label={$\frac{1}{m}$}] (p2) {};
      \node[right=of p2] (others) {$\ldots$};
      \node[state,right=of others,label={$\frac{m-\DCo}{m}$}] (p2mm1) {};
      \node[state,right=of p2mm1,label={$1$}] (p2m) {};
      
      \path [trans]
      (p0) edge [loop left,min distance=10mm,out=225,in=135] node {$1$} (p0)
      (p1) edge[bend right] node[above] {$\DCo$} (p0)
      (p1) edge[bend left] node[above] {$\Co$} (p2)
      (p2) edge[bend left] node[below] {$\Co$} (p1)
      (p2) edge[bend right] node[below] {$\DCo$} (others)
      (p2mm1) edge[bend left] node[below] {$\DCo$} (others)
      (p2mm1) edge[bend right] node[below] {$\Co$} (p2m)
      (p2m) edge[loop right,min distance=10mm,out=45,in=-45] node {$1$} (p2m)
      ;
    \end{tikzpicture}
  \end{figure}

  Для сравнения, ниже представлена стратегия из базовой работы [Domansky,~2007], которая отвечает значению $\Co = 1$
  \begin{figure}[tb]
    \centering
    \begin{tikzpicture}
      [
      auto,node distance=1.5cm,
      trans/.style={->,shorten >=1pt,>=stealth',semithick},
      state/.style={shape=circle,draw,minimum size=1mm},
      ]
      \node[state,label={$0$}] (p0) {};
      \node[state,right=of p0,label={$\frac{1}{m}$}] (p1) {}; 
      \node[state,right=of p1,label={$\frac{2}{m}$}] (p2) {};
      \node[right=of p2] (others) {$\ldots$};
      \node[state,right=of others,label={$\frac{m-1}{m}$}] (p2mm1) {};
      \node[state,right=of p2mm1,label={$1$}] (p2m) {};
      
      \path [trans]
      (p0) edge [loop left,min distance=10mm,out=225,in=135] node {$1$} (p0)
      (p1) edge[bend right] node[above] {$1/2$} (p0)
      (p1) edge[bend left] node[above] {$1/2$} (p2)
      (p2) edge[bend left] node[below] {$1/2$} (p1)
      (p2) edge[bend right] node[below] {$1/2$} (others)
      (p2mm1) edge[bend left] node[below] {$1/2$} (others)
      (p2mm1) edge[bend right] node[below] {$1/2$} (p2m)
      (p2m) edge[loop right,min distance=10mm,out=45,in=-45] node {$1$} (p2m)
      ;
    \end{tikzpicture}
  \end{figure}
\end{frame}

\begin{frame}
  \frametitle{Значение игры}

  \begin{thm}
    Игра $G^m_\infty(p)$ имеет значение $V^m_\infty{p}$. Данная функция является кусочно-линейной, состоит из $m~+~1$ линейных сегментов, и полностью определяется своими значениями в следующих точках:
    \begin{gather*}
      V^m_\infty{(k+\Co)/m} = \frac{1}{2} \left( (m - (k + \Co))(k + \Co) + \DCo\Co
      \right),\enskip
      k = \overline{0, m - 1},\\
      V^m_\infty{0} = V^m_\infty{1} = 0.
    \end{gather*}
  \end{thm}
\end{frame}

\begin{frame}
  \begin{figure}[thb]
    \centering
    \begin{tikzpicture}[yscale=1.3,xscale=8]
      \draw[thick,->,>=stealth'] (-0.1,0) -- (1.1,0) node[right] {$p$};
      \draw[thick] (0,-0.1) -- (0,0.1);
      \node[anchor=north east] at (0,0) {$0$};
      \draw[thick] (1,-0.1) -- (1,0.1);
      \node[anchor=north west] at (1,0) {$1$};

      \draw[thick] plot file {../plots/ch1-v0.75.dat};
      
      \draw[thick,dashed] (0.15, -0.1) -- (0.15, 3.2);
      \node[anchor=north] at (0.15, -0.1) {$\Co/m$};

      \draw[thick,dashed] (0.75, -0.1) -- (0.75, 3.2);
      \node[anchor=north] at (0.75, -0.1) {$(k+\Co)/m$};
      
      \node at (0.55, 2.6) {$V^\Co_ m(p)$};
    \end{tikzpicture}
  \end{figure}

  \begin{center}
    График функции $V^{m,\Co}_\infty(p)$
  \end{center}
\end{frame}

\begin{frame}
  \begin{prop}
    При любом значении $p \in [0,1]$, $\Co \in (0,1)$ и $m \geq 3$ справедливо неравенство
    \begin{equation*}
      V^{m, \Co}_\infty(p) \geq V^{m,1}_\infty(p) = V^{m,0}_\infty(p),
    \end{equation*}
    причем равенство достигается только при $p = k/m, k = \overline{0,m}$.
  \end{prop}

  Таким образом, из всех рассматриваемых механизмов торгов, те механизмы, которые предписывают продавать акцию по наибольшей или наименьшей предложенной цене, гарантируют инсайдеру наименьший возможный выигрыш.
\end{frame}

\begin{frame}
  \begin{figure}[b]
    \centering
    \begin{tikzpicture}[yscale=1.5,xscale=8]
      \draw[thick,->,>=stealth'] (-0.1,0) -- (1.1,0) node[right] {$p$};
      \draw[thick] (0,-0.1) -- (0,0.1);
      \node[anchor=north east] at (0,0) {$0$};
      \draw[thick] (1,-0.1) -- (1,0.1);
      \node[anchor=north west] at (1,0) {$1$};

      \draw[thick] plot file {../plots/ch1-v0.5.dat};
      \draw[very thick,dashed] plot file {../plots/ch1-v1.dat};
      
      \node at (0.86, 2.6) {$V^{m,1/2}_\infty(p)$};
      \node at (0.55, 2.6) {$V^{m,1}_\infty(p)$};
    \end{tikzpicture}
    \label{ch1:fig:value-comparison}
  \end{figure}

  \begin{center}
    Графики функции $V^{m,\Co}_\infty(p)$ при значениях $\Co = 1/2$ и $\Co = 1$
  \end{center}
\end{frame}

\begin{frame}
  \frametitle{Продолжительность игры}

  \begin{prop}
    Игра $G^{m, \Co}_\infty(p)$ в среднем заканчивается за конечное количество шагов.
    При $p \in \{k/m, (k+\Co)/m\}$ ее ожидаемая продолжительность выражается формулами
    \begin{equation*} 
      \tau((k+\Co)/m) = \frac{(m-k-\Co)(k+\Co)}{\Co \DCo},\quad
      \tau(k/m) = \frac{k(m-k)}{\Co \DCo}.
    \end{equation*}
  \end{prop}

  \begin{figure}[tb]
    \centering
    \begin{tikzpicture}
      [
      auto,node distance=1.5cm,
      trans/.style={->,shorten >=1pt,>=stealth',semithick},
      state/.style={shape=circle,draw,minimum size=1mm},
      ]
      \node[state,label={$0$}] (p0) {};
      \node[state,right=of p0,label={$\frac{\Co}{m}$}] (p1) {}; 
      \node[state,right=of p1,label={$\frac{1}{m}$}] (p2) {};
      \node[right=of p2] (others) {$\ldots$};
      \node[state,right=of others,label={$\frac{m-\DCo}{m}$}] (p2mm1) {};
      \node[state,right=of p2mm1,label={$1$}] (p2m) {};
      
      \path [trans]
      (p0) edge [loop left,min distance=10mm,out=225,in=135] node {$1$} (p0)
      (p1) edge[bend right] node[above] {$\DCo$} (p0)
      (p1) edge[bend left] node[above] {$\Co$} (p2)
      (p2) edge[bend left] node[below] {$\Co$} (p1)
      (p2) edge[bend right] node[below] {$\DCo$} (others)
      (p2mm1) edge[bend left] node[below] {$\DCo$} (others)
      (p2mm1) edge[bend right] node[below] {$\Co$} (p2m)
      (p2m) edge[loop right,min distance=10mm,out=45,in=-45] node {$1$} (p2m)
      ;
    \end{tikzpicture}
  \end{figure}
\end{frame}

% \begin{frame}
% \frametitle{Цели и задачи}
% \begin{itemize}
%   \item \textbf{Предмет исследования:} 
%   \item \textbf{Исследуемые характеристики:} 
%   \item \textbf{Цель исследования:} 
%   \item \textbf{Актуальность:} 
% \end{itemize}
% \end{frame}

% \begin{frame}
% \frametitle{Проблемы}
% \begin{itemize}
%   \item Проблема 1
%   \item Проблема 2
%   \item Проблема 3    
% \end{itemize}
% \end{frame}

% \begin{frame}
% \frametitle{План работ}
% \begin{enumerate}
%   \item \textbf{Задача 1}
%   \begin{itemize}
%     \item Подзадача 1-1
%     \item Подзадача 1-2
%   \end{itemize}
%   \item \textbf{Задача 2}
%   \begin{itemize}
%     \item Подзадача 2-1
%     \item Подзадача 2-2
%     \item Подзадача 2-3
%   \end{itemize}
%   \item \textbf{Задача 3}
%   \begin{itemize}
%     \item Подзадача 3-1
%     \item Подзадача 3-2
%     \item Подзадача 3-3
%   \end{itemize}
% \end{enumerate}
% \end{frame}

% \begin{frame}
% \frametitle{Список обыкновенный}
% \begin{itemize}
%   \item Пункт 1
%   \item Пункт 2
%   \item Пункт 3
% \end{itemize}
% \end{frame}

% \begin{frame}
% \frametitle{Одиночное изображение}
% \begin{figure}[H]
%   \center
%   \includegraphics[width=0.8\linewidth]{latex}
% \end{figure}
% \end{frame}

% \begin{frame}
% \frametitle{Формулы}
% $$
% \left\{
%   \begin{array}{rl}
%     \dot x = & \sigma (y-x) \\
%     \dot y = & x (r - z) - y \\
%     \dot z = & xy - bz
%   \end{array}
% \right.
% $$
% \end{frame}

% \begin{frame}
% \frametitle{Составное изображение}
% \begin{figure}[h]
%   \begin{minipage}[h]{0.49\linewidth}
%     \textbf{Составная \\ подпись 1}
%     \center{\includegraphics[width=1\linewidth]{knuth1}}
%   \end{minipage}
%   \hfill
%   \begin{minipage}[h]{0.49\linewidth}
%     \textbf{Составная \\ подпись 2}
%     \center{\includegraphics[width=1\linewidth]{knuth2}}
%   \end{minipage}
% \end{figure}
% \end{frame}

% \begin{frame}
% \frametitle{Таблица}
% \begin{tabular}{|l|l|}
% \hline
% \textbf{Заголовок 1} & \textbf{Заголовок 2} \\
% \hline
% Сумма & $b+a$ \\
% \hline
% Разность & $a-b$ \\
% \hline
% Произведение & $a*b$ \\
% \hline
% \end{tabular}
% \end{frame}

% \begin{frame}
% \frametitle{Большой многоуровневый список}
% \begin{itemize}
%   \item \textbf{Пункт 1}
%     \begin{itemize}
%       \itemi Подпункт 1-1
%       \itemi Подпункт 1-2
%     \end{itemize}
%   \item \textbf{Пункт 2}
%     \begin{itemize}
%       \itemi Подпункт 2-1
%     \end{itemize}
%   \item \textbf{Пункт 3}
%     \begin{itemize}
%       \itemi Подпункт 3-1
%       \itemi Подпункт 3-2
%     \end{itemize}
%   \item \textbf{Пункт 4}
%     \begin{itemize}
%       \itemi Подпункт 4-1
%     \end{itemize}
%   \item \textbf{Пункт 5}
%     \begin{itemize}
%       \itemi Подпункт 5-1
%       \itemi Подпункт 5-2
%       \itemi Подпункт 5-3
%     \end{itemize}
% \end{itemize}
% \end{frame}

% \begin{frame}
% \frametitle{Четыре изображения}
% \begin{figure}[H]
%   \center
%     \includegraphics[width=0.4\linewidth]{latex}
%     \includegraphics[width=0.4\linewidth]{latex}\\
%     \includegraphics[width=0.4\linewidth]{latex}
%     \includegraphics[width=0.4\linewidth]{latex}
% \end{figure}
% \end{frame}

% %%%%%%%%%%%%%%%%%%%%%%%%%%%%%%
% \begin{frame}
% \frametitle{Перспективы развития проекта}
% \begin{itemize}
%   \item Перспектива 1
%   \item Перспектива 2
%   \item Перспектива 3
%   \item Перспектива 4
%   \item Перспектива 5
% \end{itemize}
% \end{frame}

% \begin{frame}
% \frametitle{Результаты работы}
% \begin{itemize}
%   \item Результат 1
%   \item Результат 2
%   \item Результат 3
%   \item Результат 4
% \end{itemize}
% \end{frame}

\begin{frame}
\begin{center}
Спасибо за внимание!
\end{center}
\end{frame}

\begin{thebibliography}{99}
\bibitem{demeyer02}
  De Meyer B., Saley H. \emph{On the strategic origin of Brownian motion in
    finance} // Int J Game Theory. 2002.
 
\bibitem{demeyer02arb}%
  De Meyer B., Saley H. \emph{A model of game with a continuum of states of
    nature} // Pr\'{e}publication de l’Institut Elie Cartan, Nancy. 2002.
  
\bibitem{demeyer10}%
  De Meyer B. \emph{Price dynamics on a stock market with asymmetric
    information} // Games and Economic Behavior. 2010.
  
\bibitem{domansky07}%
  Domansky V. \emph{Repeated games with asymmetric information and random price
  fluctuation at finance markets} // Int J Game Theory. 2007.

\bibitem{domansky11}%
  Доманский В.К, Крепс В.Л. \emph{Теоретико игровая модель биржевых торгов:
    стратегические аспекты формирования цен на фондовых рынках} // Журнал Новой
  экономический ассоциации. 2011

\bibitem{pyanykh16}%
  Пьяных А.И. \emph{Многошаговая модель биржевых торгов с асимметричной
    информацией и элементами переговоров} // Вестн. Моск. ун-та. Сер. 15. 2016.
  
\bibitem{chatterjee83}%
  Chatterjee K., Samuelson W. \emph{Bargaining under Incomplete Information} //
  Operations Research, 1983.
  
\bibitem{myerson83}%
  Myerson R., Satterthwaite M. \emph{Efficient Mechanisms for Bilateral Trading}
  // Journal of Economic Theory, 1983.

\end{thebibliography}

\end{document} 
