\documentclass[12pt]{beamer}
\usepackage[T2A]{fontenc}
\usepackage[utf8]{inputenc}
\usepackage[english,russian]{babel}
\usepackage{amssymb,amsfonts,amsmath,mathtext,amsthm}
\usepackage{cite,enumerate,float,indentfirst}
\usepackage{booktabs}
\usepackage{mathrsfs}
\usepackage{bbm}
\usepackage{tikz}
\usepackage{graphicx}
\usetikzlibrary{positioning}
\usetikzlibrary{arrows}
\usetikzlibrary{automata}

% \graphicspath{{../images/}{images/}} 

\usetheme[secheader]{Boadilla}
\usecolortheme{seahorse}

% \usetheme{Pittsburgh}
% \usecolortheme{whale}

\beamertemplatenavigationsymbolsempty

\newcommand{\todo}{\alert}
%%% �������� �������� %%%
\newcommand{\thesisAuthor}             % �����������, ��� ������
{%
    \texorpdfstring{% \texorpdfstring takes two arguments and uses the first for (La)TeX and the second for pdf
        ������ ����� ��������% ��� ����� ������������ �� ��������� ����� ��� � ������, ��� ����� �������������� ����������
    }{%
        ������ ����� ��������% ��� ������ ��� ������� pdf-�����. � ����� ����, ���� pdf ����� ��������� ����������� ��� ����� ����������������� ��������, ����� ��������� ������������ �������.
    }%
}
\newcommand{\thesisAuthorShort}             % �����������, ��� ������ ����������
{�.�.~������}

\newcommand{\thesisUdk}                % �����������, ���
{519.83}
\newcommand{\thesisTitle}              % �����������, ��������
{\texorpdfstring{\MakeUppercase{� ������� ������������� ��� � �������� �����������, ������������ �������� �����}}{� ������� ������������� ��� � �������� �����������, ������������ �������� �����}}
\newcommand{\thesisSpecialtyNumber}    % �����������, �������������, �����
{\texorpdfstring{01.01.09}{01.01.09}}
\newcommand{\thesisSpecialtyTitle}     % �����������, �������������, ��������
{\texorpdfstring{���������� ���������� � �������������� �����������}{���������� ���������� � �������������� �����������}}
\newcommand{\thesisDegree}             % �����������, ������� �������
{��������� ������-�������������� ����}
\newcommand{\thesisCity}               % �����������, ����� ������
{������}
\newcommand{\thesisYear}               % �����������, ��� ������
{2016}
\newcommand{\thesisOrganization}       % �����������, �����������
{���������� ��������������� ����������� �����~�.�.����������}
\newcommand{\thesisOrganizationShort}  % �����������, ������� �������� ����������� ��� �������
{\todo{��� ��. �.�.����������}}

\newcommand{\thesisInOrganization}       % �����������, ����������� � ���������� ������: ������ ��������� � ...
{\todo{���������� ��������������� ������������ ����� �.�.����������}}

\newcommand{\supervisorFio}            % ������� ������������, ���
{������� �������� ����������}
\newcommand{\supervisorRegalia}        % ������� ������������, �������
{�������� ������-�������������� ����, ������}
\newcommand{\supervisorFioShort}            % ������� ������������, ���
{�.�.~�������}
\newcommand{\supervisorRegaliaShort}        % ������� ������������, �������
{�.�.-�.�.,~���.}


\newcommand{\opponentOneFio}           % �������� 1, ���
{\todo{������� ��� ��������}}
\newcommand{\opponentOneRegalia}       % �������� 1, �������
{\todo{������ ������-�������������� ����, ���������}}
\newcommand{\opponentOneJobPlace}      % �������� 1, ����� ������
{\todo{�� ����� ������� �������� ��� ����� ������}}
\newcommand{\opponentOneJobPost}       % �������� 1, ���������
{\todo{������� ������� ���������}}

\newcommand{\opponentTwoFio}           % �������� 2, ���
{\todo{������� ��� ��������}}
\newcommand{\opponentTwoRegalia}       % �������� 2, �������
{\todo{�������� ������-�������������� ����}}
\newcommand{\opponentTwoJobPlace}      % �������� 2, ����� ������
{\todo{�������� ����� ������ c ������� ������� ������� ������� ���������}}
\newcommand{\opponentTwoJobPost}       % �������� 2, ���������
{\todo{������� ������� ���������}}

\newcommand{\leadingOrganizationTitle} % ������� �����������, �������������� ������
{\todo{����������� ��������������� ��������� ��������������� ���������� ������� ����������������� ����������� �~������� ������� ������� ������� ���������}}

\newcommand{\defenseDate}              % ������, ����
{\todo{DD mmmmmmmm YYYY~�.~�~XX �����}}
\newcommand{\defenseCouncilNumber}     % ������, ����� ���������������� ������
{\todo{NN}}
\newcommand{\defenseCouncilTitle}      % ������, ���������� ���������������� ������
{\todo{�������� ����������}}
\newcommand{\defenseCouncilAddress}    % ������, ����� ���������� ���������������� ������
{\todo{�����}}

\newcommand{\defenseSecretaryFio}      % ��������� ���������������� ������, ���
{\todo{������� ��� ��������}}
\newcommand{\defenseSecretaryRegalia}  % ��������� ���������������� ������, �������
{\todo{�-�~���.-���. ����}}            % ��� ���������� ���� �����, ��������: ���� � 7.0.12-2011 + http://base.garant.ru/179724/#block_30000

\newcommand{\synopsisLibrary}          % �����������, �������� ����������
{\todo{�������� ����������}}
\newcommand{\synopsisDate}             % �����������, ���� ��������
{\todo{DD mmmmmmmm YYYY ����}}

% To avoid conflict with beamer class use \providecommand
\providecommand{\keywords}%                 % �������� ����� ��� ���������� PDF ����������� � ������������
{������������ ����, ������������� ����, �������� ����������, ������������� ����������, ������������ ��������}

%%% Local variables:
%%% coding: cp1251
%%% End:      % Основные сведения

\newcommand{\overbar}[1]{\mkern 1.5mu\overline{\mkern-1.3mu#1\mkern-1.3mu}\mkern 1.3mu}
\newcommand{\Co}{\beta}
\newcommand{\DCo}{\overline{\beta}}
\newcommand{\E}{\ensuremath{\mathbb{E}}}
\newcommand{\D}{\ensuremath{\mathbb{D}}}
\newcommand{\Borel}{\ensuremath{\mathscr{B}}}
\newcommand{\Ind}{\mathbbm{1}}
\newcommand{\di}{\ensuremath{\mathrm{d}}}
\newcommand{\Z}{\mathbb{Z}}
\newcommand{\p}{\overbar{p}}
\newcommand{\q}{\overbar{q}}
\newcommand{\tauv}{\overline{\tau}}
\newcommand{\sigmav}{\ensuremath{\overbar{\sigma}}}
\newcommand{\theGame}[1][n]{\ensuremath{G_{#1}}}
\newcommand{\High}[1][\ensuremath{\infty}]{\ensuremath{H_{#1}}}
\newcommand{\Low}[1][\ensuremath{\infty}]{\ensuremath{L_{#1}}}

\newtheorem{thm}{Теорема}[section]
\newtheorem{prop}{Утверждение}[section]
\newtheorem{lem}{Лемма}[section]

\setbeamercolor{footline}{fg=blue}
\setbeamertemplate{footline}{
  \leavevmode%
  \hbox{%
  \begin{beamercolorbox}[wd=.363333\paperwidth,ht=2.25ex,dp=1ex,center]{}%
    % И. О. Фамилия, Организация кратко
    \thesisAuthorShort, \thesisOrganizationShort
  \end{beamercolorbox}%
  \begin{beamercolorbox}[wd=.333333\paperwidth,ht=2.25ex,dp=1ex,center]{}%
    % Город, 20XX
    \thesisCity, \thesisYear
  \end{beamercolorbox}%
  \begin{beamercolorbox}[wd=.303333\paperwidth,ht=2.25ex,dp=1ex,right]{}%
  Стр. \insertframenumber{} из \inserttotalframenumber \hspace*{2ex}
  \end{beamercolorbox}}%
  \vskip0pt%
}

\newcommand{\itemi}{\item[\checkmark]}

%\title{\small{Название презентации}}
\title{\small{\thesisTitle}}
\author{\small{%
\emph{Выступающий:}~\thesisAuthorShort\\%
\emph{Руководитель:}~\supervisorRegaliaShort~\supervisorFioShort}\\%
\vspace{30pt}%
\thesisOrganization%
\vspace{20pt}%
}
\date{\small{\thesisCity, \thesisYear}}

\begin{document}

\maketitle

\section{Введение}

\begin{frame}
  \frametitle{Описание модели}
  \begin{itemize}
  \item
    Между двумя игроками в течение $n \leqslant \infty$ шагов происходят
    торги за однотипные акции.
  \item
    Цена акции $s$ определяется ходом случая в соответствии с распределением $\p$.
  \item
    Первый игрок (инсайдер) знает цену $s$, второй "--- знает только вероятностное распределение $\p$.
  \item
    На каждом шаге торгов игроки делают ставки $i,\ j$.
    Игрок, предложивший большую ставку, покупает у другого акцию по цене $Tr(i, j)$.
    При равных ставках сделка не состоится.
  \end{itemize}
\end{frame}

\begin{frame}
  \frametitle{Непрерывная и дискретная постановки}
  
  \begin{tabular}{c|c|c}
    & Непрерывная модель & Дискретная модель \\
    \midrule
    Ставки & вещественные из $[0, 1]$ & вида $i/m,\ i = \overline{0, m}$ \\
    \hline \\
    Цена акции& \multicolumn{2}{c}{$s \in \{0, 1\};\ P(s = 1) = p$} \\
    \hline \\
    Цена сделки& \multicolumn{2}{c}{$Tr(i, j) = \max(i, j)$} \\
    \hline \\
    $\{V_n(p)\}$ & растет как $\sqrt{n}$ & ограничена \\
    \bottomrule \\
  \end{tabular}

  \textbf{Цель:} показать возможность стратегического происхождения броуновского движения (случайного блуждания) в динамике цен активов.
\end{frame}

\begin{frame}
  \frametitle{Мотивация}
  
  Рассмотрение более общего симметричного механизма торгов $Tr(i, j)$ обозначено в работе [De Meyer,~Saley,~2002] как одно из актуальных направлений дальнейших исследований.
  
  В качестве такого механизма был выбран
  \[
    Tr(i, j) = \beta \max(i, j) + (1-\beta) \min(i, j),\ \beta \in [0, 1],
  \]
  предложенный в [Chatterjee,~Samuelson,~1983].
  
  Параметр $\beta$ можно интерпретировать как переговорную силу продавца.
\end{frame}

\section{Глава 1. Дискретная модель}

\begin{frame}
  \frametitle{Глава 1. Теоретико-игровая модель биржевых торгов с дискретными ставками}
  
  \begin{itemize}
  \item 
  Множество состояний $S = \{L, H\}$.
  \item
  В состоянии $L$ цена $s=0$, в состоянии $H$ цена $s=m$.
  \item
  Игроки могут делать ставки из множества $\{0, 1, \ldots, m\}$. Обозначим множества действий первого и второго игроков через $I$ и $J$ соответственно.
  \end{itemize}
\end{frame}

\begin{frame}
  Модель отвечает повторяющейся игре $G^{m, \Co}_n(p)$ с платежными матрицами
  \begin{equation*}\footnotesize
    A^{L,\Co}(i, j) = \begin{cases}
      \DCo i + \Co j, &\, i < j, \\
      0, &\, i = j, \\
      -\Co i - \DCo j, &\, i > j,
    \end{cases}
    \qquad
    A^{H,\Co}(i, j) = \begin{cases}
      \DCo i + \Co j - m, &\, i < j, \\
      0, &\, i = j, \\
      m - \Co i - \DCo j, &\, i > j.
    \end{cases}
  \end{equation*}
  Стратегией первого игрока является последовательность ходов 
  \[
  \sigma = (\sigma_1, \sigma_2, \ldots, \sigma_t, \ldots),\ \sigma_t: S \times I^{t-1} \rightarrow \Delta(I).
  \]

  Стратегией второго игрока --- последовательность ходов 
  \[
  \tau = (\tau_1, \tau_2, \ldots, \tau_t, \ldots),\ \tau_t: I^{t-1} \rightarrow \Delta(J).
  \]

  Выигрыш задается как
  \begin{equation*}
    K^{m,\Co}_n(p, \sigma, \tau) = \E_{\Pi[\sigma,\tau]} \sum_{t=1}^n
    \left(
      pA^{H,\Co}(i_t^H, j_t) + (1 - p)A^{L,\Co}(i_t^L, j_t)
    \right).
  \end{equation*}
\end{frame}

\begin{frame}
  \frametitle{Оптимальная стратегия второго игрока}
  
  Следующая чистая стратегия $\tau^k$ предложена В.~К.~Доманским в [Domansky,~2007]:
  \[
    \tau^k_1 = k, \quad \tau^k_t(i_{t-1}, j_{t-1}) = \begin{cases}
      j_{t-1} - 1, & \, i_{t-1} < j_{t-1}, \\
      j_{t-1},     & \, i_{t-1} = j_{t-1}, \\
      j_{t-1} + 1, & \, i_{t-1} > j_{t-1}.
    \end{cases}
  \]
  
  В работе показано, что при $p \in (k - 1 + \Co, k + \Co],\ k = \overline{0, m}$ использование $\tau^k$ оптимально для второго игрока в игре $G^{m, \Co}_\infty(p)$.
\end{frame}

\begin{frame}
  \frametitle{Рекурсивная структура игры $G^{m, \Co}_n(p)$}
  
  Представим $\sigma$ как $(\sigma_1, \sigma(i),\ i \in I)$, а $\tau$ как $(\tau_1, \tau(i),\ i \in I)$.

  Параметризуем ход $\sigma_1$ инсайдера с помощью 
  \begin{itemize}
  \item
    полной вероятности $q(i)$ сделать ставку $i$
  \item
    апостериорной вероятности $p(H|i)$ состояния $H$ при использовании ставки $i$.
  \end{itemize}
  По формуле Байеса вероятность сделать ставку $i \in I$ в состоянии $s \in S$ выражается как
  \[
    \sigma^s_{1,i} = \frac{p(s|i)q_i}{p(s)}.
  \]

  Формулу выигрыша можно переписать в виде
  \begin{equation*}
    K^m_n(p, \sigma, \tau) = 
    K^m_1(p, \sigma_1, \tau_1) + 
    \sum_{i \in I} q(i) K^m_{n-1}(p(H|i), \sigma(i), \tau(i)).
  \end{equation*}
\end{frame}

\begin{frame}
  \frametitle{Оптимальная стратегия первого игрока}
  
  При $p = k/m,\ k = \overline{1, m-1}$ первый игрок рандомизирует выбор ставок $k, \, k+1$ с параметрами
  \begin{gather*}
    q_k = \Co, \quad p(H|k) = (k-1+\Co)/m,\\
    q_{k+1} = \DCo, \quad p(H|k + 1) = (k + \Co)/m.
  \end{gather*}
  При $p = (k+\Co)/m,\ k = \overline{0, m-1}$ --- с параметрами
  \begin{gather*}
    q_k = \DCo, \quad p(H|k) = k/m,\\
    q_{k+1} = \Co, \quad p(H|k + 1) = (k+1)/m.
  \end{gather*}
  
  Для остальных значений $p$ используется стратегия, дающая линейную комбинацию соответствующих выигрышей.
\end{frame}

\begin{frame}
  Графически оптимальную стратегию первого игрока можно представить следующим образом:
  \begin{figure}[tb]
    \centering
    \begin{tikzpicture}
      [
      auto,node distance=1.5cm,
      trans/.style={->,shorten >=1pt,>=stealth',semithick},
      state/.style={shape=circle,draw,minimum size=1mm},
      ]
      \node[state,label={$0$}] (p0) {};
      \node[state,right=of p0,label={$\frac{\Co}{m}$}] (p1) {}; 
      \node[state,right=of p1,label={$\frac{1}{m}$}] (p2) {};
      \node[right=of p2] (others) {$\ldots$};
      \node[state,right=of others,label={$\frac{m-\DCo}{m}$}] (p2mm1) {};
      \node[state,right=of p2mm1,label={$1$}] (p2m) {};
      
      \path [trans]
      (p0) edge [loop left,min distance=10mm,out=225,in=135] node {$1$} (p0)
      (p1) edge[bend right] node[above] {$\DCo$} (p0)
      (p1) edge[bend left] node[above] {$\Co$} (p2)
      (p2) edge[bend left] node[below] {$\Co$} (p1)
      (p2) edge[bend right] node[below] {$\DCo$} (others)
      (p2mm1) edge[bend left] node[below] {$\DCo$} (others)
      (p2mm1) edge[bend right] node[below] {$\Co$} (p2m)
      (p2m) edge[loop right,min distance=10mm,out=45,in=-45] node {$1$} (p2m)
      ;
    \end{tikzpicture}
  \end{figure}

  Для сравнения, ниже представлена стратегия из базовой работы [Domansky,~2007], которая отвечает значению $\Co = 1$
  \begin{figure}[tb]
    \centering
    \begin{tikzpicture}
      [
      auto,node distance=1.5cm,
      trans/.style={->,shorten >=1pt,>=stealth',semithick},
      state/.style={shape=circle,draw,minimum size=1mm},
      ]
      \node[state,label={$0$}] (p0) {};
      \node[state,right=of p0,label={$\frac{1}{m}$}] (p1) {}; 
      \node[state,right=of p1,label={$\frac{2}{m}$}] (p2) {};
      \node[right=of p2] (others) {$\ldots$};
      \node[state,right=of others,label={$\frac{m-1}{m}$}] (p2mm1) {};
      \node[state,right=of p2mm1,label={$1$}] (p2m) {};
      
      \path [trans]
      (p0) edge [loop left,min distance=10mm,out=225,in=135] node {$1$} (p0)
      (p1) edge[bend right] node[above] {$1/2$} (p0)
      (p1) edge[bend left] node[above] {$1/2$} (p2)
      (p2) edge[bend left] node[below] {$1/2$} (p1)
      (p2) edge[bend right] node[below] {$1/2$} (others)
      (p2mm1) edge[bend left] node[below] {$1/2$} (others)
      (p2mm1) edge[bend right] node[below] {$1/2$} (p2m)
      (p2m) edge[loop right,min distance=10mm,out=45,in=-45] node {$1$} (p2m)
      ;
    \end{tikzpicture}
  \end{figure}
\end{frame}

\begin{frame}
  \frametitle{Значение игры}

  \begin{thm}
    Игра $G^m_\infty(p)$ имеет значение $V^m_\infty{p}$. Данная функция является кусочно-линейной, состоит из $m~+~1$ линейных сегментов, и полностью определяется своими значениями в следующих точках:
    \begin{gather*}
      V^m_\infty((k+\Co)/m) = \frac{1}{2} \left( (m - (k + \Co))(k + \Co) + \DCo\Co
      \right),\enskip
      k = \overline{0, m - 1},\\
      V^m_\infty{0} = V^m_\infty{1} = 0.
    \end{gather*}
  \end{thm}
\end{frame}

\begin{frame}
  \begin{figure}[thb]
    \centering
    \begin{tikzpicture}[yscale=1.3,xscale=8]
      \draw[thick,->,>=stealth'] (-0.1,0) -- (1.1,0) node[right] {$p$};
      \draw[thick] (0,-0.1) -- (0,0.1);
      \node[anchor=north east] at (0,0) {$0$};
      \draw[thick] (1,-0.1) -- (1,0.1);
      \node[anchor=north west] at (1,0) {$1$};

      \draw[thick] plot file {../plots/ch1-v0.75.dat};
      
      \draw[thick,dashed] (0.15, -0.1) -- (0.15, 3.2);
      \node[anchor=north] at (0.15, -0.1) {$\Co/m$};

      \draw[thick,dashed] (0.75, -0.1) -- (0.75, 3.2);
      \node[anchor=north] at (0.75, -0.1) {$(k+\Co)/m$};
      
      \node at (0.55, 2.6) {$V^\Co_ m(p)$};
    \end{tikzpicture}
  \end{figure}

  \begin{center}
    График функции $V^{m,\Co}_\infty(p)$
  \end{center}
\end{frame}

\begin{frame}
  \begin{prop}
    При любом значении $p \in [0,1]$, $\Co \in (0,1)$ и $m \geq 3$ справедливо неравенство
    \begin{equation*}
      V^{m, \Co}_\infty(p) \geq V^{m,1}_\infty(p) = V^{m,0}_\infty(p),
    \end{equation*}
    причем равенство достигается только при $p = k/m, k = \overline{0,m}$.
  \end{prop}

  Таким образом, из всех рассматриваемых механизмов торгов, те механизмы, которые предписывают продавать акцию по наибольшей или наименьшей предложенной цене, гарантируют инсайдеру наименьший возможный выигрыш.
\end{frame}

\begin{frame}
  \begin{figure}[b]
    \centering
    \begin{tikzpicture}[yscale=1.5,xscale=8]
      \draw[thick,->,>=stealth'] (-0.1,0) -- (1.1,0) node[right] {$p$};
      \draw[thick] (0,-0.1) -- (0,0.1);
      \node[anchor=north east] at (0,0) {$0$};
      \draw[thick] (1,-0.1) -- (1,0.1);
      \node[anchor=north west] at (1,0) {$1$};

      \draw[thick] plot file {../plots/ch1-v0.5.dat};
      \draw[very thick,dashed] plot file {../plots/ch1-v1.dat};
      
      \node at (0.86, 2.6) {$V^{m,1/2}_\infty(p)$};
      \node at (0.55, 2.6) {$V^{m,1}_\infty(p)$};
    \end{tikzpicture}
    \label{ch1:fig:value-comparison}
  \end{figure}

  \begin{center}
    Графики функции $V^{m,\Co}_\infty(p)$ при значениях $\Co = 1/2$ и $\Co = 1$
  \end{center}
\end{frame}

\begin{frame}
  \frametitle{Продолжительность игры}

  \begin{prop}
    Игра $G^{m, \Co}_\infty(p)$ в среднем заканчивается за конечное количество шагов.
    При $p \in \{k/m, (k+\Co)/m\}$ ее ожидаемая продолжительность выражается формулами
    \begin{equation*} 
      \tau((k+\Co)/m) = \frac{(m-k-\Co)(k+\Co)}{\Co \DCo},\quad
      \tau(k/m) = \frac{k(m-k)}{\Co \DCo}.
    \end{equation*}
  \end{prop}

  \begin{figure}[tb]
    \centering
    \begin{tikzpicture}
      [
      auto,node distance=1.5cm,
      trans/.style={->,shorten >=1pt,>=stealth',semithick},
      state/.style={shape=circle,draw,minimum size=1mm},
      ]
      \node[state,label={$0$}] (p0) {};
      \node[state,right=of p0,label={$\frac{\Co}{m}$}] (p1) {}; 
      \node[state,right=of p1,label={$\frac{1}{m}$}] (p2) {};
      \node[right=of p2] (others) {$\ldots$};
      \node[state,right=of others,label={$\frac{m-\DCo}{m}$}] (p2mm1) {};
      \node[state,right=of p2mm1,label={$1$}] (p2m) {};
      
      \path [trans]
      (p0) edge [loop left,min distance=10mm,out=225,in=135] node {$1$} (p0)
      (p1) edge[bend right] node[above] {$\DCo$} (p0)
      (p1) edge[bend left] node[above] {$\Co$} (p2)
      (p2) edge[bend left] node[below] {$\Co$} (p1)
      (p2) edge[bend right] node[below] {$\DCo$} (others)
      (p2mm1) edge[bend left] node[below] {$\DCo$} (others)
      (p2mm1) edge[bend right] node[below] {$\Co$} (p2m)
      (p2m) edge[loop right,min distance=10mm,out=45,in=-45] node {$1$} (p2m)
      ;
    \end{tikzpicture}
  \end{figure}
\end{frame}

\section{Глава 2. Непрерывная модель}

\begin{frame}
  \frametitle{Глава 2. Теоретико-игровая  модель биржевых торгов с непрерывными ставками}
  
  \begin{itemize}
  \item 
  Множество состояний $S = \{L, H\}$.
  \item
  В состоянии $L$ цена $s=0$, в состоянии $H$ цена $s=1$.
  \item
  Игроки могут делать вещественные ставки из отрезка $[0, 1]$.
  \end{itemize}
\end{frame}

\begin{frame}
  \frametitle{Прямая игра}
  
    \begin{itemize}
    \item Обозначим $y_t = (y^R_t, y^N_t)$ портфель инсайдера на $t$-м шаге
      торгов, где $y^R_t$ и $y^N_t$ --- количество единиц рискового актива и
      денег соответственно.
    \item Если на $t$-м шаге игроки делают ставки $i_{t}, j_{t} \in [0,1]$,
      то портфель $y_t$ = $y_{t-1} + t(i_{t}, j_{t})$, где при $\DCo = 1 - \Co$
      \[
        t(i, j) =
        \Ind_{i > j}(1, -(\Co i + \DCo j) +
        \Ind_{i < j}(-1, \DCo i + \Co j).
      \]
    \item Стоимость портфеля равна
      \[
        V(y_t) = \Ind_{s = H} y^R_t + y^N_t.
      \]
      Цель игроков --- максимизировать стоимость итогового портфеля $y_n$.
    \end{itemize}
\end{frame}

\begin{frame}
  Выигрыш первого игрока равен
  \begin{equation*}
    g_n(p, \sigma, \tau) = \E_{p, \sigma, \tau} V(y_n).
  \end{equation*}
  Обозначим данную игру $G_n(p)$. Ее верхнее и нижнее значения даются
  формулами
  \begin{equation*}
    V_{1,n}(p) = \sup_\sigma \inf_\tau g_n(p, \sigma, \tau), \:
    V_{2,n}(p) = \inf_\tau \sup_\sigma g_n(p, \sigma, \tau).
  \end{equation*}
  Если $V_{1,n}(p) = V_{2,n}(p) = V_n(p)$, то игра имеет значение $V_n(p)$.
\end{frame}
  
\begin{frame}
  \frametitle{Двойственная игра}
  
  Двойственная игра $G_n^*(x)$ определяется следующим образом:
  \begin{itemize}
  \item 
    Перед началом торгов игрок 1 выбирает $s \in S$.
  \item
    В том случае, если он выбрал $H$, то по окончании игры он платит игроку 2 штраф размера $x$.
  \item
    Игрок 1 также контролирует значение $p$.
  \item
    В остальном правила двойственной игры аналогичны правилам прямой игры.
  \end{itemize}
\end{frame}

\begin{frame}
  Функция выигрыша двойственной игры задается формулой
  \begin{equation*}
    g^*_n(x, (p, \sigma), \tau) = x p - g_n(p, \sigma, \tau).
  \end{equation*}
  Игрок 2 стремиться максимизировать ее значение.

  Верхнее и нижнее значение игры $G^*_n(x)$ даются формулами
  \begin{equation*}
    W_{1,n}(x) = \inf_{(p, \sigma)} \sup_\tau g^*_n(x, (p, \sigma), \tau), \:
    W_{2,n}(x) = \sup_\tau \inf_{(p, \sigma)} g^*_n(x, (p, \sigma), \tau).
  \end{equation*}
  Если $W_{1,n}(x) = W_{2,n}(x) = W_n(x)$, то игра имеет значение $W_n(x)$.
\end{frame}

\begin{frame}
  \frametitle{Параметризация стратегии первого игрока в прямой игре}
  
  Для стратегии первого игрока имеет место декомпозиция 
  \[
    \sigma = (\sigma_1, \sigma(i),\ i \in I).
  \]
  Ход $\sigma_1$ параметризуем при помощи функций
  \begin{gather*}
    f(u) = F^{-1}_{\sigma^M_1}(u) := \inf \{ x\ |\ F_{\sigma^M_1}(x) \geqslant u \},\\
    Q(u) = P(s = H | i = f(u)),
  \end{gather*}
  где $\sigma^M_1$ --- маргинальное распределение ставки $i$.
  
  Восстановить $\sigma_1$ можно из следующего равенства:
  \begin{align*}
    P(i \in B \; | \; s = H) 
    = \frac{P(i \in B, \; s = H)}{P(s = H)}
    = \int_0^1 \Ind_{f(u) \in B} \frac{Q(u)}{p} \di u,
  \end{align*}
\end{frame}

\begin{frame}
  \frametitle{Параметризация стратегии второго игрока в двойственной игре}
  
  Для стратегии второго игрока имеет место декомпозиция 
  \[
    \tau = (\tau_1, \tau(i),\ i \in I).
  \]
  Ход $\tau_1$ параметризуем при помощи функции
  \begin{gather*}
    h(u) = F^{-1}_{\tau^M_1}(u) := \inf \{ x\ |\ F_{\tau^M_1}(x) \geqslant u \},\\
  \end{gather*}
  где $\tau^M_1$ --- маргинальное распределение ставки $j$.
\end{frame}

\begin{frame}
  \frametitle{Анализ прямой и двойственной игры}
  
  Следуя схеме, приведенной в [De~Meyer,~Saley,~2002],
  \begin{itemize}
  \item найдены функции $f, Q$, выравнивающие выигрыш первого игрока при $p_2 \in
    [f(0), f(1)]$ в прямой игре;
  \item найдена функция $h$, выравнивающая выигрыш второго игрока при $p_1 \in
    [h(0), h(1)]$ в двойственной игре;
  \item из соотношений двойственности между верхними и нижними значениями
    прямой и двойственной игр показано, что данные стратегии оптимальны.
  \end{itemize}
\end{frame}
  
    
\begin{frame}
  \frametitle{Результаты}
  
  Значение двойственной игры $G^*_n(x)$ дается формулой:
  \begin{equation*}
    W_{n+1}(x) = \int_0^1 W_n(x - 2u + 1), \, W_0(x) = \min(x, 0).
  \end{equation*}
  
  \begin{figure}[H!]
    \centering
    \includegraphics[width=0.8\textwidth]{images/wGraphs}
  \end{figure}
\end{frame} 

\begin{frame}
  Значение прямой игры $G_n(P)$ дается формулой:
  \begin{equation*}
    V_n(P) = W_n^*(P),
  \end{equation*}
  где $W^*(P) = \inf_x \{ x P - W_n(x) \}$ --- сопряженная в смысле Фенхеля функция.

  \begin{figure}[H!]
    \centering
    \includegraphics[width=0.8\textwidth]{images/vGraphs.eps}
  \end{figure}
\end{frame}

\begin{frame}
  Функция $Q(u)$ определяется как
  \begin{equation*}
    Q(n, p)(u) = W'_n(x + 1 - 2u),
  \end{equation*}
  где параметр двойственной игры $x$ удовлетворяет уравнению 
  \begin{equation*}
    W'_n(x) = p.
  \end{equation*}

  \begin{figure}[H!]
    \centering
    \includegraphics[width=0.8\textwidth]{images/qGraphs.eps}
  \end{figure}
\end{frame}

\begin{frame}
  Функция $f(u)$ определяется как
  \begin{equation*}
    f(n, p, \Co)(u) = 2 (u - \DCo)^{-2} \int_{\DCo}^u (x - \DCo) Q(n,p)(x) \di x.
  \end{equation*}
  
  При этом $h(u) = f(u)$.

  \begin{figure}[H!]
    \centering
    \includegraphics[width=0.8\textwidth]{images/fGraphs.eps}
  \end{figure}
\end{frame}

\begin{frame}
  Сравнивая полученные результаты с существующими, получаем:
  \begin{itemize}
  \item
    Введение механизма заключения сделки с параметром $\Co$ влияет на оптимальные стратегии инсайдера и неосведомленного игрока.
  \item
    При этом значение игры не зависит от значения $\Co$ и совпадает с таковым в [De~Meyer,~Saley,~2002].
    В этом смысле непрерывная модель отличается от дискретной, в которой и стратегии и значение игры зависят от значения $\Co$.
  \item
    В силу того, что функции значения прямой и двойственной игр совпадают с [De~Meyer,~Saley,~2002], справедливы все утверждения об асимптотике значений прямой игры и динамике апостериорных вероятностей.
  \end{itemize}
\end{frame}

\section{Глава 3. Обобщения дискретной модели}

\begin{frame}
  \frametitle{Глава 3. Дальнейшие обобщения модели биржевых торгов с дискретными ставками}

  \begin{itemize}
  \item 
    Множество состояний $S = \Z_+$.
    Рассматриваются только такие распределения $\p$, что дисперсия цены $\D \p$ конечна.
  \item
    Каждое состояние из $S$ отождествляется с соответствующей ценой акции.
  \item
    Игроки могут делать произвольные целочисленные ставки.
    Выплата первому игроку в состоянии $s$ равна
    \begin{equation*}
      a^s(i, j) =
      \begin{cases}
        \DCo i + \Co j - s, &\; i < j, \\
        0, &\; i = j, \\
        s - \Co i - \DCo j, &\; i > j.
      \end{cases}
    \end{equation*}
  \end{itemize}

  Базовая модель с $Tr(i, j) = \max(i, j)$ была рассмотрена в [Доманский,~Крепс,~2011].
\end{frame}

\begin{frame}
  \frametitle{Оптимальная стратегия второго игрока}
  Введем обозначения для множества распределений на $S$ с заданным математическим ожиданием:
  \begin{gather*}
    \Theta(x) = \left\{ \p' \in \Delta(S): \E \p' = x \right\},\\
    \Lambda(x, y) = \left\{ \p' \in \Delta(S): x < \E \p' \leqslant y \right\}.
  \end{gather*}

  Стратегия $\tauv^*$, заключающаяся в применении $\tauv^k$ 
  \begin{equation*}
    \tau^k_1 = k, \quad
    t^k_t = \begin{cases}
      j_{t-1}, &\; i_{t-1} < j_{t-1},\\
      j_t, &\; i_{t-1} = j_{t-1},\\
      j_{t+1}, &\; i_{t-1} > j_{t-1}.
    \end{cases}
  \end{equation*}
  при $\p \in \Lambda(k-1+\Co, k+\Co)$, является оптимальной. 
\end{frame}

\begin{frame}
  \begin{thm}
    При использовании игроком 2 стратегии $\tauv^*$, выигрыш игрока 1 в игре
    $\theGame[\infty](\p)$ ограничен сверху функцией $\High(\p)$.

    Функция $\High(\p)$ является кусочно-линейной с областями линейности $\Lambda(k - 1 + \beta, k + \beta)$ и областями недифференцируемости $\Theta(k+\beta)$ при $k \in S$.
    
    Для распределений $\p$ таких, что $\E \p = k - 1 + \beta + \xi, \; \xi \in (0, 1]$, ее значение равно
    \begin{equation}
      \label{upper-bound:eq:H(p)}
      \High(\p) = \frac{1}{2} \left( \D \p + \beta(1-\beta) - \xi(1-\xi) \right).
    \end{equation}
  \end{thm}
\end{frame}

\begin{frame}
  \begin{lem}
    Пусть $\p_k \in \Delta(S)$, $\sigmav_k$ --- стратегия игрока 1, которая гарантирует ему выигрыш $\Low[n](\p_k)$ в $\theGame[n](\p_k)$, и $\q_k = (q^1_k, \ldots, q^n_k),\ \p^i_k = ( p^{1|i}_k, p^{2|i}_k, \ldots)$ --- параметры первого хода стратегии $\sigmav_k,\ k = 1,2$.

    Тогда для $\p = \lambda \p_1 + (1-\lambda) \p_2, \; \lambda \in [0, 1],$ стратегия $\sigmav_c$, первый ход которой задается параметрами
    \begin{equation*}
      q^i = \lambda q^i_1 + (1-\lambda) q^i_2, \quad
      p^{s|i} = \left(\lambda q^i_1 p^{s|i}_1 + (1-\lambda) q^i_2 p^{s|i}_2\right)/q^i,
    \end{equation*}
    гарантирует игроку 1 выигрыш $\lambda \Low[n](\p_1) + (1-\lambda) \Low[n](\p_2)$.
  \end{lem}

  \begin{figure}[tb]
    \centering
    \begin{tikzpicture}
      [
      auto,node distance=3.5cm,
      trans/.style={->,shorten >=1pt,>=stealth',semithick},
      state/.style={shape=circle,draw,minimum size=0.1mm},
      ]
      \node[state,label={$\p_1, \sigmav_1, L_n(\p_1)$}] (p1) {};
      \node[state,right=of p1,label={$\lambda \p_1 + (1-\lambda) \p_2, \sigmav_c$}] (pc) {}; 
      \node[state,right=of pc,label={$\p_2,\ \sigmav_2, L_n(\p_2)$}] (p2) {};
      
      \draw (p1) -- (pc);
      \draw (pc) -- (p2);
    \end{tikzpicture}
  \end{figure}
\end{frame}

\begin{frame}
  \frametitle{Оптимальная стратегия первого игрока}
  \begin{itemize}
  \item 
    В [Доманский,~Крепс,~2011] показано, что любое распределение $\p$ может быть представлено в виде выпуклой комбинации распределений с одноточечным и двухточечным носителем.
  \item
    Обозначим $\p^x(l, r) \in \Theta(x)$ распределение с математическим ожиданием $x$ и носителем $\{l, r\}$.
    Достаточно найти стратегию первого игрока, гарантирующую ему $\High(\p)$ для $\p = \p^{k+\beta}(l,r), \; k \in S$.
  \item
    Перенося результаты главы 1 на распределения $\p^{k+\Co}(l, r)$, получим стратегию, которая гарантирует первому игроку
    \begin{equation*}
      \frac{1}{2}((r-k-\beta)(k+\beta-l) + \beta(1-\beta)) = \High[\infty](p^{k+\Co}(l, r)).
    \end{equation*}
  \end{itemize}
\end{frame}

\begin{frame}
\frametitle{Результаты, выносимые на защиту}
\begin{itemize}
\item
  Решение бесконечной повторяющейся игры биржевых торгов с дискретными ставками и более общим механизмом торгов.
\item
  Решение конечношаговой повторяющейся игры биржевых торгов с непрерывными ставками и более общим механизмом торгов.
\item
  Обобщение результатов, полученных для модели с дискретными ставками, на случай рынка со счетным множеством возможных цен рискового актива.
\end{itemize}
\end{frame}
  
\begin{frame}
  \begin{center}
    Спасибо за внимание!
  \end{center}
\end{frame}

\end{document} 
