\chapter*{Список условных обозначений}             % Заголовок
\addcontentsline{toc}{chapter}{Список условных обозначений}  % Добавляем его в оглавление

% \nomenclature{$\mathbb{Z}_+$}{множество неотрицательных целых чисел}

% \printnomenclature

\begin{tabular}{p{3.5cm}p{12.0cm}}
  $\Borel(X)$ & борелевская $\sigma$-алгебра подмножеств множества $X$ \\
  $\Delta(X)$ & совокупность вероятностных распределений на~$(X, \Borel(X))$ \\
  $\N_+$ & множество положительных натуральных чисел \\
  $\Z_+$ & множество неотрицательных целых чисел \\
  $\R_+$ & множество неотрицательных действительных чисел \\
  $\E_\sigma$ & символ математического ожидания по мере $\sigma$ \\
  $\eqdef$ & символ равенства <<по определению>> \\
  $\Ind{ }$ & индикаторная функция \\
  $f^*(x^*)$ & функция, сопряженная к $f$ в смысле Фенхеля \\
  $\dom f$ & эффективное множество функции $f$ \\
  $\range f$ & множество значений функции $f$ \\
  $\inter X$ & внутренность множества $X$ \\
  $\int f(x)\ \sigma(\di x)$ & символ интеграл Лебега функции $f$ по мере $\sigma$ \\
  $\partial f(x)$ & субдифференциал функции $f$ в точке $x$ \\
  $x^+$ & $\max\{0, x\}$, где $x \in \R$ \\
  $[a]$ & целая часть числа $a$ \\
\end{tabular}

%%% Local Variables:
%%% mode: latex
%%% TeX-master: "../dissertation"
%%% End:
