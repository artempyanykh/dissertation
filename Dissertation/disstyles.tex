%%% Изображения %%%
\graphicspath{{images/}{Dissertation/images/}}         % Пути к изображениям

%%% Макет страницы %%%
% Выставляем значения полей (ГОСТ 7.0.11-2011, 5.3.7)
\geometry{a4paper,top=2cm,bottom=2cm,left=2.5cm,right=1cm,nofoot,nomarginpar} %,showframe
\setlength{\topskip}{0pt}   %размер дополнительного верхнего поля

%%% Интервалы %%%
%% По ГОСТ Р 7.0.11-2011, пункту 5.3.6 требуется полуторный интервал
%% Реализация средствами класса (на основе setspace) ближе к типографской классике.
%% И правит сразу и в таблицах (если со звёздочкой) 
%\DoubleSpacing*     % Двойной интервал
\OnehalfSpacing*    % Полуторный интервал
\setSpacing{1.42}   % Полуторный интервал, подобный Ворду (возможно, стоит включать вместе с предыдущей строкой)

%%% Выравнивание и переносы %%%
%% http://tex.stackexchange.com/questions/241343/what-is-the-meaning-of-fussy-sloppy-emergencystretch-tolerance-hbadness
%% http://www.latex-community.org/forum/viewtopic.php?p=70342#p70342
\tolerance 1414
\hbadness 1414
\emergencystretch 1.5em % В случае проблем регулировать в первую очередь
\hfuzz 0.3pt
\vfuzz \hfuzz
%\raggedbottom
%\sloppy                 % Избавляемся от переполнений
\clubpenalty=10000      % Запрещаем разрыв страницы после первой строки абзаца
\widowpenalty=10000     % Запрещаем разрыв страницы после последней строки абзаца

%%% Блок управления параметрами для выравнивания заголовков в тексте %%%
\newlength{\otstuplen}
\setlength{\otstuplen}{\theotstup\parindent}
\ifnumequal{\value{headingalign}}{0}{% выравнивание заголовков в тексте
    \newcommand{\hdngalign}{\centering}                % по центру
    \newcommand{\hdngaligni}{}% по центру
    \setlength{\otstuplen}{0pt}
}{%
    \newcommand{\hdngalign}{}                 % по левому краю
    \newcommand{\hdngaligni}{\hspace{\otstuplen}}      % по левому краю
} % В обоих случаях вроде бы без переноса, как и надо (ГОСТ Р 7.0.11-2011, 5.3.5)

%%% Оглавление %%%
\renewcommand{\cftchapterdotsep}{\cftdotsep}                % отбивка точками до номера страницы начала главы/раздела

%% Переносить слова в заголовке не допускается (ГОСТ Р 7.0.11-2011, 5.3.5). Заголовки в оглавлении должны точно повторять заголовки в тексте (ГОСТ Р 7.0.11-2011, 5.2.3). Прямого указания на запрет переносов в оглавлении нет, но по той же логике невнесения искажений в смысл, лучше в оглавлении не переносить:
\setrmarg{2.55em plus1fil}                             %To have the (sectional) titles in the ToC, etc., typeset ragged right with no hyphenation
\renewcommand{\cftchapterpagefont}{\normalfont}        % нежирные номера страниц у глав в оглавлении
\renewcommand{\cftchapterleader}{\cftdotfill{\cftchapterdotsep}}% нежирные точки до номеров страниц у глав в оглавлении
%\renewcommand{\cftchapterfont}{}                       % нежирные названия глав в оглавлении

\ifnumgreater{\value{headingdelim}}{0}{%
    \renewcommand\cftchapteraftersnum{.\space}       % добавляет точку с пробелом после номера раздела в оглавлении
}{}
\ifnumgreater{\value{headingdelim}}{1}{%
    \renewcommand\cftsectionaftersnum{.\space}       % добавляет точку с пробелом после номера подраздела в оглавлении
    \renewcommand\cftsubsectionaftersnum{.\space}    % добавляет точку с пробелом после номера подподраздела в оглавлении
    \renewcommand\cftsubsubsectionaftersnum{.\space} % добавляет точку с пробелом после номера подподподраздела в оглавлении
    \AtBeginDocument{% без этого polyglossia сама всё переопределяет
        \setsecnumformat{\csname the#1\endcsname.\space}
    }
}{%
    \AtBeginDocument{% без этого polyglossia сама всё переопределяет
        \setsecnumformat{\csname the#1\endcsname\quad}
    }
}

\ifnumequal{\value{pgnum}}{1}{%
    \addtocontents{toc}{~\hfill{Стр.}\par}% добавить Стр. над номерами страниц
}{}

\renewcommand*{\cftappendixname}{\appendixname\space} % Слово Приложение в оглавлении

%%% Колонтитулы %%%
% Порядковый номер страницы печатают на середине верхнего поля страницы (ГОСТ Р 7.0.11-2011, 5.3.8)
\makeevenhead{plain}{}{\thepage}{}
\makeoddhead{plain}{}{\thepage}{}
\makeevenfoot{plain}{}{}{}
\makeoddfoot{plain}{}{}{}
\pagestyle{plain}

%%% Оформление заголовков глав, разделов, подразделов %%%
%% Работа должна быть выполнена ... размером шрифта 12-14 пунктов (ГОСТ Р 7.0.11-2011, 5.3.8). То есть не должно быть надписей шрифтом более 14. Так и поставим.
%% Эти установки будут давать одинаковый результат независимо от выбора базовым шрифтом 12 пт или 14 пт
\newcommand{\basegostsectionfont}{\fontsize{14pt}{16pt}\selectfont\bfseries}

\makechapterstyle{thesisgost}{%
    \chapterstyle{default}
    \setlength{\beforechapskip}{0pt}
    \setlength{\midchapskip}{0pt}
    \setlength{\afterchapskip}{\theintvl\curtextsize}
    \renewcommand*{\chapnamefont}{\basegostsectionfont}
    \renewcommand*{\chapnumfont}{\basegostsectionfont}
    \renewcommand*{\chaptitlefont}{\basegostsectionfont}
    \renewcommand*{\chapterheadstart}{}
    \ifnumgreater{\value{headingdelim}}{0}{%
        \renewcommand*{\afterchapternum}{.\space}   % добавляет точку с пробелом после номера раздела
    }{%
        \renewcommand*{\afterchapternum}{\quad}     % добавляет \quad после номера раздела
    }
    \renewcommand*{\printchapternum}{\hdngaligni\hdngalign\chapnumfont \thechapter}
    \renewcommand*{\printchaptername}{}
    \renewcommand*{\printchapternonum}{\hdngaligni\hdngalign}
}

\makeatletter
\makechapterstyle{thesisgostchapname}{%
    \chapterstyle{thesisgost}
    \renewcommand*{\printchapternum}{\chapnumfont \thechapter}
    \renewcommand*{\printchaptername}{\hdngaligni\hdngalign\chapnamefont \@chapapp} %
}
\makeatother

\chapterstyle{thesisgost}

\setsecheadstyle{\basegostsectionfont\hdngalign}
\setsecindent{\otstuplen}

\setsubsecheadstyle{\basegostsectionfont\hdngalign}
\setsubsecindent{\otstuplen}

\setsubsubsecheadstyle{\basegostsectionfont\hdngalign}
\setsubsubsecindent{\otstuplen}

\sethangfrom{\noindent #1} %все заголовки подразделов центрируются с учетом номера, как block 

\ifnumequal{\value{chapstyle}}{1}{%
    \chapterstyle{thesisgostchapname}
    \renewcommand*{\cftchaptername}{\chaptername\space} % будет вписано слово Глава перед каждым номером раздела в оглавлении
}{}%

%%% Интервалы между заголовками
\setbeforesecskip{\theintvl\curtextsize}% Заголовки отделяют от текста сверху и снизу тремя интервалами (ГОСТ Р 7.0.11-2011, 5.3.5).
\setaftersecskip{\theintvl\curtextsize}
\setbeforesubsecskip{\theintvl\curtextsize}
\setaftersubsecskip{\theintvl\curtextsize}
\setbeforesubsubsecskip{\theintvl\curtextsize}
\setaftersubsubsecskip{\theintvl\curtextsize}

%%% Блок дополнительного управления размерами заголовков
\ifnumequal{\value{headingsize}}{1}{% Пропорциональные заголовки и базовый шрифт 14 пт
    \renewcommand{\basegostsectionfont}{\large\bfseries}
    \renewcommand*{\chapnamefont}{\Large\bfseries}
    \renewcommand*{\chapnumfont}{\Large\bfseries}
    \renewcommand*{\chaptitlefont}{\Large\bfseries}
}{}

%%% Счётчики %%%

%% Упрощённые настройки шаблона диссертации: нумерация формул, таблиц, рисунков
\ifnumequal{\value{contnumeq}}{1}{%
    \counterwithout{equation}{chapter} % Убираем связанность номера формулы с номером главы/раздела
}{}
\ifnumequal{\value{contnumfig}}{1}{%
    \counterwithout{figure}{chapter}   % Убираем связанность номера рисунка с номером главы/раздела
}{}
\ifnumequal{\value{contnumtab}}{1}{%
    \counterwithout{table}{chapter}    % Убираем связанность номера таблицы с номером главы/раздела
}{}


%%http://www.linux.org.ru/forum/general/6993203#comment-6994589 (используется totcount)
\makeatletter
\def\formbytotal#1#2#3#4#5{%
    \newcount\@c
    \@c\totvalue{#1}\relax
    \newcount\@last
    \newcount\@pnul
    \@last\@c\relax
    \divide\@last 10
    \@pnul\@last\relax
    \divide\@pnul 10
    \multiply\@pnul-10
    \advance\@pnul\@last
    \multiply\@last-10
    \advance\@last\@c
    \total{#1}~#2%
    \ifnum\@pnul=1#5\else%
    \ifcase\@last#5\or#3\or#4\or#4\or#4\else#5\fi
    \fi
}
\makeatother

\AtBeginDocument{
%% регистрируем счётчики в системе totcounter
    \regtotcounter{totalcount@figure}
    \regtotcounter{totalcount@table}       % Если иным способом поставить в преамбуле то ошибка в числе таблиц
    \regtotcounter{TotPages}               % Если иным способом поставить в преамбуле то ошибка в числе страниц
}

%%% Правильная нумерация приложений %%%
%% По ГОСТ 2.105, п. 4.3.8 Приложения обозначают заглавными буквами русского алфавита,
%% начиная с А, за исключением букв Ё, З, Й, О, Ч, Ь, Ы, Ъ.
%% Здесь также переделаны все нумерации русскими буквами.
\ifxetexorluatex
    \makeatletter
    \def\russian@Alph#1{\ifcase#1\or
       А\or Б\or В\or Г\or Д\or Е\or Ж\or
       И\or К\or Л\or М\or Н\or
       П\or Р\or С\or Т\or У\or Ф\or Х\or
       Ц\or Ш\or Щ\or Э\or Ю\or Я\else\xpg@ill@value{#1}{russian@Alph}\fi}
    \def\russian@alph#1{\ifcase#1\or
       а\or б\or в\or г\or д\or е\or ж\or
       и\or к\or л\or м\or н\or
       п\or р\or с\or т\or у\or ф\or х\or
       ц\or ш\or щ\or э\or ю\or я\else\xpg@ill@value{#1}{russian@alph}\fi}
    \makeatother
\else
    \makeatletter
    \if@uni@ode
      \def\russian@Alph#1{\ifcase#1\or
        А\or Б\or В\or Г\or Д\or Е\or Ж\or
        И\or К\or Л\or М\or Н\or
        П\or Р\or С\or Т\or У\or Ф\or Х\or
        Ц\or Ш\or Щ\or Э\or Ю\or Я\else\@ctrerr\fi}
    \else
      \def\russian@Alph#1{\ifcase#1\or
        \CYRA\or\CYRB\or\CYRV\or\CYRG\or\CYRD\or\CYRE\or\CYRZH\or
        \CYRI\or\CYRK\or\CYRL\or\CYRM\or\CYRN\or
        \CYRP\or\CYRR\or\CYRS\or\CYRT\or\CYRU\or\CYRF\or\CYRH\or
        \CYRC\or\CYRSH\or\CYRSHCH\or\CYREREV\or\CYRYU\or
        \CYRYA\else\@ctrerr\fi}
    \fi
    \if@uni@ode
      \def\russian@alph#1{\ifcase#1\or
        а\or б\or в\or г\or д\or е\or ж\or
        и\or к\or л\or м\or н\or
        п\or р\or с\or т\or у\or ф\or х\or
        ц\or ш\or щ\or э\or ю\or я\else\@ctrerr\fi}
    \else
      \def\russian@alph#1{\ifcase#1\or
        \cyra\or\cyrb\or\cyrv\or\cyrg\or\cyrd\or\cyre\or\cyrzh\or
        \cyri\or\cyrk\or\cyrl\or\cyrm\or\cyrn\or
        \cyrp\or\cyrr\or\cyrs\or\cyrt\or\cyru\or\cyrf\or\cyrh\or
        \cyrc\or\cyrsh\or\cyrshch\or\cyrerev\or\cyryu\or
        \cyrya\else\@ctrerr\fi}
    \fi
    \makeatother
\fi