\chapter{Теоретико-игровая модель биржевых торгов с дискретными ставками и счетным множеством состояний} \label{chapt2}
{
\newcommand{\s}{\ensuremath{s}}
\newcommand{\norm}[1]{\left\lVert#1\right\rVert}
\newcommand{\LL}{L^1(\{s^2\})}
\newcommand{\PM}{\ensuremath{\overline{P}}}

В разделе~\ref{ch2:sec:intro} дано формальное описание модели рынка со счетным множеством состояний.
Раздел~\ref{ch2:sec:upper-bound} посвящен анализу стратегии неосведомленного игрока и получению оценки сверху на выигрыш инсайдера.
Построение стратегии инсайдера и получение оценки снизу его выигрыша проведено в разделе~\ref{ch2:sec:lower-bound}.
Рассматриваемая стратегия основана на представлении вероятностных распределений в виде суммы распределений с двухточечным носителем из~\cite{domansky11}.
Вопросы сходимости по норме рядов в разложении распределений с бесконечным носителем также исследованы в разделе~\ref{ch2:sec:lower-bound}.
В разделе~\ref{ch2:sec:game-solution} дана теорема о значении игры неограниченной продолжительности и приведена динамика апостериорных вероятностей при применении игроками оптимальных стратегий.
Раздел~\ref{ch2:sec:optimal-strategy2} посвящен построению второй оптимальной стратегии инсайдера, анализу случайных блужданий апостериорных вероятностей, порождаемых данной стратегией, а также сравнению результатов с результатами из~\cite{domansky11}.

Основные результаты данной главы опубликованы в работе~\cite{pyanykh:orm2016}.

\section{Описание модели рынка со счетным множеством состояний}
\label{ch2:sec:intro}

Рассмотрим модель рынка с дискретными ставками и множеством состояний $S = \Z_+$.
Перед началом игры случай выбирает состояние рынка $\s \in S$ в соответствии с вероятностным распределением $\pd = (\pc{s}, \; s \in S)$, имеющим конечную дисперсию состояния $\D \pd < \infty$.
Множество всех таких распределений обозначим $\PM$.

На каждом шаге игры $t = \overline{1,n}, \; n \leqslant \infty$, игроки делают ставки $i_t \in I, \, j_t \in J$, где $I = J = \Z_+$.
В силу того, что игрок, предложивший б\'{о}льшую ставку, покупает акцию у другого по цене, равной выпуклой комбинации предложенных ставок, выплата первому игроку в состоянии $s$ равна
\begin{equation*}
  \as(i_t, j_t) =
  \begin{cases}
    (1-\beta) i_t + \beta j_t - s, &\; i_t < j_t, \\
    0, &\; i_t = j_t, \\
    s - \beta i_t - (1-\beta)j_t, &\; i_t > j_t.
  \end{cases}
\end{equation*}

На шаге $t$ обоим игрокам достаточно принимать в расчет лишь последовательность $(i_1, i_2, \ldots, i_{t-1})$ действий первого игрока на предыдущих ходах.
Это связано с тем, что информация, получаемая вторым игроком относительно состояния $s$, может передаваться лишь посредством действий первого игрока.
Подробное обсуждение данного факта можно найти в \cite{mertens15}.

Обозначим через $\Delta(X)$ совокупность всех вероятностных распределений на множестве $X$.

Стратегией первого игрока является последовательность ходов (отображений)
$\sigmar = (\sigmas{1}, \ldots, \sigmas{n})$, где $\sigmas{t}: S \times I^{t-1} \rightarrow \Delta(I)$.
Множество стратегий первого игрока обозначим $\Sigma$.

Стратегией второго игрока является последовательность ходов (отображений)
$\taur = (\taus{1}, \ldots, \taus{n})$, где $\taus{t}: I^{t-1} \rightarrow \Delta(J)$.
Множество стратегий второго игрока обозначим $\Tau$.

При использовании игроками стратегий $\sigmar$ и $\taur$ ожидаемый выигрыш первого игрока равен
\begin{equation*}
  \K{n}(\pd, \sigmar, \taur) =
  \E_{(\pd, \sigmar, \taur)} \sum_{t=1}^n \as(i_t, j_t),
\end{equation*}
где математическое ожидание берется по мере, индуцированной $\pd$, $\sigmar$ и $\taur$ на множестве $S \times I^n \times J^n$.
Заданную таким образом игру обозначим $\theG{n}(\pd)$.

Если для некоторых стратегий $\sigmar^* \in \Sigma,$ $\taur^* \in \Tau$ выполняются равенства
\begin{equation*}
  \inf_{\taur \in \Tau} \K{n}(\pd, \sigmar^*, \taur) =
  \K{n}(\pd, \sigmar^*, \taur^*) =
  \sup_{\sigmar \in \Sigma} \K{n}(\pd, \sigmar, \taur^*) \eqdef
  \V{n}(\pd),
\end{equation*}
то говорят, что игра $\theG{n}(\pd)$ имеет значение $\V{n}(\pd)$, а стратегии $\sigmar^*$ и $\taur^*$
называются оптимальными.

Нижнее и верхнее значения игры $\theG{n}(\pd)$ обозначим соответственно
\begin{equation*}
  \LowV{n}(\pd) =
    \sup_{\sigmar \in \Sigma}
    \inf_{\taur \in \Tau}
    \K{n}(\pd, \sigmar, \taur), \quad
  \HighV{n}(\pd) =
    \inf_{\taur \in \Tau}
    \sup_{\sigmar \in \Sigma}
    \K{n}(\pd, \sigmar, \taur).
\end{equation*}
Данные функции являются вогнутыми на $\PM$.
Доказательство этого утверждения проводится аналогично~\cite{domansky07}.

Аналогично тому, как это было сделано в главе~\ref{chapt1}, опишем опишем рекурсивную структуру игры $\theG{n}(\pd)$.
Представим стратегию первого игрока в виде $\sigmar = (\sigmas{1}, \sigmara{i}, i \in I)$, где $\sigmas{1}$ "--- его ход на первом шаге, а $\sigmara{i}$ --- его стратегия в игре продолжительности $n-1$ в зависимости от ставки $i$, выбранной им на первом шаге.

Аналогично, стратегию второго игрока представим в виде $\taur = (\taus{1}, \taura{i}, \; i \in I)$.
%
Далее, обозначим $\qc{i}$ полную вероятность, с которой первый игрок делает ставку $i \in I$, а $\qd = (\qc{i}, \; i \in I)$ --- соответствующее распределение.
Также обозначим $\pac{s}{i}$ апостериорную вероятность состояния $s$ в зависимости от ставки $i$ первого игрока, и $\pad{i} = (\pac{s}{i}, \, s \in S)$ --- соответствующее апостериорное распределение.
Тогда для функции выигрыша первого игрока в игре $\theG{n}(\pd)$ справедлива формула
\begin{equation*}
  \K{n}(\pd, \sigmar, \taur) =
  \K{1}(\pd, \sigmas{1}, \taus{1}) +
  \sum_{i \in I} \qc{i} \K{n-1}(\pad{i}, \sigmara{i}, \taura{i}).
\end{equation*}

\section{Оценка сверху выигрыша первого игрока в игре~$\mathbf{G^\beta_\infty(\overline{p})}$}
\label{ch2:sec:upper-bound}

В~\cite{domansky11} определена следующая чистая стратегия второго игрока $\taur^k = (\tau^k_t,\ t = \overline{1, \infty})$:
\begin{equation*}
  \tau^k_1 = k, \quad
  \tau^k_t(i_{t-1}, j_{t-1}) = \begin{cases}
    j_{t-1} - 1, &\; i_{t-1} < j_{t-1},\\
    j_{t-1}, &\; i_{t-1} = j_{t-1},\\
    j_{t-1}+1, &\; i_{t-1} > j_{t-1}.
  \end{cases}
\end{equation*}
Другими словами, второй игрок делает ставку равную $k$ на первом шаге, а далее либо подражает инсайдеру, либо смещается на единицу к ставке инсайдера предыдущего шага.

\begin{lemma}
  \label{ch2:upper-bound:lemma:vector-payoffs}
  При применении стратегии $\taur^k$ в игре $\theG{n}(\pd)$ второй игрок в состоянии $s$ гарантирует себе проигрыш не более
  \begin{gather*}
    h^s_n(\taur^k) = \begin{cases}
      \displaystyle \sum_{t=0}^{n-1} (k-s-t-1+\beta)^+, \; s \leqslant k, \\
      \displaystyle \sum_{t=0}^{n-1} (s-k-t-\beta)^+, \; s > k.
    \end{cases} 
  \end{gather*}
  Для любого $s \in S$ последовательность $\left\{ h^s_n(\taur^k) \right\}_{n = 1}^\infty$ не убывает, ограничена сверху и сходится к %
  \begin{equation}
    \label{ch2:upper-bound:eq:max-payoff}
    h^s_\infty(\taur^k) = (s - k + 1 - 2\beta)(s-k)/2.
  \end{equation}
\end{lemma}
\begin{proof}
  Проведем доказательство индукцией по $n$ для случая $s > k$.
  При $n = 1$ оптимальный ответ первого игрока на $\taur^k$ будет $i = k + 1$.
  Тогда его выигрыш в игре $\theG{1}(\pd)$ равен
  \begin{equation*}
    h^s_1(\taur^k) = s - \beta(k+1) - (1-\beta)k = s - k - \beta.
  \end{equation*}
  База индукции проверена.
  Предположим, что утверждение верно при $n \leqslant N$.
  При $n = N + 1$ первый игрок имеет два разумных ответа на $\taur^k$: ставка $i=k+1$, что соответствует покупке акции по наименьшей возможной цене, и ставка $i=k-1$, что соответствует продаже акции за наибольшую возможную цену.
  Найдем оценки выигрыша в каждом из случаев.
  Для $i=k+1$ выигрыш первого игрока не превосходит величины
  \begin{equation*}
    s - k - \beta + h^s_N(\taur^{k+1}) = \sum_{t=0}^N(s-k-t-\beta)^+.
  \end{equation*}
  Аналогично для $i = k - 1$ тот же выигрыш не превосходит
  \begin{equation*}
    \beta k + (1-\beta)(k-1) - s + h^s_N(\taur^{k-1}) = \sum_{t=0}^{N-2}(s-k-t-\beta)^+.
  \end{equation*}
  При $s \leqslant k$ формула для $h^s_n(\taur^k)$ доказывается аналогично.
  Сходимость последовательности $\left\{  h^s_n(\taur^k) \right\}_{n=1}^\infty$ к $h^s_\infty(\taur^k)$ следует из равенств
  $h^s_n(\taur^k) = h^s_{n+1}(\taur^k)$ при $n \geqslant s - k$.
\end{proof}

Следующие множества распределений зададим ограничениями на математическое ожидание состояния:
\begin{gather*}
  \Theta(x) = \left\{ \pd \in \PM: \E \pd = x \right\}, \\
  \Lambda(x, y) = \left\{ \pd \in \PM: x < \E \pd \leqslant y \right\}.
\end{gather*}

Пусть $\taur^*$ --- стратегия второго игрока, состоящая в применении $\taur^k$ при $\pd \in \Lambda(k-1+\beta,k+\beta)$.
Отметим, что при заданном распределении $\pd$ выбор $k$ зависит от значения $\beta$.
Отсюда следует, что стратегия $\taur^*$ также зависит от $\beta$.

\begin{theorem}
  \label{ch2:upper-bound:theorem}
  При использовании вторым игроком стратегии $\taur^*$, выигрыш первого игрока в игре
  $\theG{\infty}(\pd)$ ограничен сверху функцией
  \begin{equation*}
    \H{\infty}(\pd) = \inf_{k \in J} \sum_{s \in S} \pc{s}  h^s_\infty(\taur^k).
  \end{equation*}
  Функция $\H{\infty}(\pd)$ является кусочно-линейной вогнутой с областями линейности $\Lambda(k - 1 + \beta, k + \beta)$ и областями недифференцируемости $\Theta(k+\beta)$ при $k \in S$.
  Для распределений $\pd$ с $\E \pd = k - 1 + \beta + \eta, \; \eta \in (0, 1]$, ее значение равно
  \begin{equation}
    \label{ch2:upper-bound:eq:H(p)}
    \H{\infty}(\pd) = \left( \D \pd + \beta(1-\beta) - \eta(1-\eta) \right)/2.
  \end{equation}
\end{theorem}
\begin{proof}
  Воспользовавшись \eqref{ch2:upper-bound:eq:max-payoff}, получим
  \begin{equation}
    \label{ch2:upper-bound:theorem:eq:1}
    \begin{gathered}
    \sum_{s \in S} \pc{s} h^s_\infty(\taur^j) = \bigl(
      j^2 + (2\beta - 1 - 2 \E \pd)j - (2\beta - 1) \E \pd + \E \pd^2 
    \bigr)/2.
    \end{gathered}
  \end{equation}
  
  Квадратичная функция $\omega(x, \pd) = x^2 + (2\beta - 1 - 2\E \pd)x$ достигает минимума по $x$ в точке
  $\E \pd - \beta + 1/2$.
  Отсюда при $\pd \in \Lambda(k - 1 + \beta, k + \beta)$ выражение \eqref{ch2:upper-bound:theorem:eq:1} достигает минимума по $j$ при $j = k$.
  Равенство \eqref{ch2:upper-bound:eq:H(p)} проверяется непосредственной подстановкой $\E \pd = k - 1 + \beta + \eta$ в \eqref{ch2:upper-bound:theorem:eq:1}.
\end{proof}

Заметим, что в данном случае наблюдается сдвиг областей линейности на $\beta$ относительно $\E \pd$ в сравнении с областями из \cite{domansky11}.

\section{Оценка снизу выигрыша первого игрока в игре~$\mathbf{G^\beta_\infty(\overline{p})}$}
\label{ch2:sec:lower-bound}

Перейдем к описанию стратегии первого игрока, гарантирующей ему выигрыш не менее $\H{\infty}(\pd)$.
Пусть $\sigmasc[s]{i}$ --- компонента хода $\sigmas{1}$ первого игрока, т.е. вероятность сделать ставку $i$ в состоянии $s$.
По правилу Байеса $\sigmasc[s]{i} = \pac{s}{i} \qc{i} / \pc{s}$.
В частности, справедливы равенства $\sum_{s \in S} \sigmasc[s]{i} \pc{s} = \qc{i},\ i \in I$.
Таким образом, ход $\sigmas{1}$ первого игрока можно определить, задав следующие параметры: полные вероятности $\qc{i}$ сделать ставку $i$ и апостериорные вероятности $\pac{s}{i}$ для $i \in I$.
Тогда его одношаговый выигрыш выражается следующим образом:
\begin{equation}
  \label{ch2:lower-bound:eq:K1(q,pi)}
  \K{1}(\pd, \sigmas{1}, j) = \sum_{i \in I} \sum_{s \in S} \qc{i} \pac{s}{i} \as(i, j).
\end{equation}

Обозначим $\L{n}(\pd, \sigmar)$ гарантированный выигрыш первого игрока, использующего стратегию $\sigmar$ в игре $\theG{n}(\pd)$, т.е.
\[
  \L{n}(\pd, \sigmar) = \inf_{\taur \in \Tau} \K{n}(\pd, \sigmar, \taur).
\]

\begin{lemma}
  \label{ch2:lower-bound:lemma:convex-combination}
  Пусть $\pd_1, \pd_2 \in \PM$, $\sigmar^1, \sigmar^2 \in \Sigma$ --- стратегии первого игрока.
  Тогда для $\pd = \lambda \pd_1 + (1-\lambda) \pd_2,\ \lambda \in [0, 1],$ найдется такая стратегия $\sigmar^c \in \Sigma$, что
  \[
    \L{n}(\pd, \sigmar^c) \geqslant
    \lambda \L{n}(\pd_1, \sigmar^1) + (1-\lambda) \L{n}(\pd_2, \sigmar^2).
  \]
\end{lemma}
\begin{proof}
  Без потери общности будем считать, что второй игрок использует стратегии $\taur$, в которых распределения на множестве ставок имеют конечные математические ожидания.
  В противном случае ожидаемый выигрыш первого игрока будет равен бесконечности.
  Множество таких стратегий обозначим $\Tau'$.
  
  Докажем следующее утверждение: найдется такая стратегия $\sigmar^c = (\sigmas[c]{1}, \sigmaran{c}{i}) \in \Sigma$, что при всех $\taur \in \Tau'$ справедливо равенство
  \begin{equation}
    \label{ch2:lower-bound:lemma:convex-combination:eq:1}
    \K{n}(\pd, \sigmar^c, \taur) =
    \lambda \K{n}(\pd_1, \sigmar^1, \taur) +
    (1-\lambda)\K{n}(\pd_2, \sigmar^2, \taur).
  \end{equation}
  Доказательство проведем по индукции.
  Пусть $\qdn{h} = (\qcn{h}{i},\ i \in I)$ и $\padn{h}{i} = (\pacn{h}{s}{i},\ s \in S)$ --- векторы полных и апостериорных вероятностей, соответствующие первому ходу $\sigmas[h]{1}$ стратегии $\sigmar^h,\ h = 1, 2$.
  Определим первый ход $\sigmas[c]{1}$ стратегии $\sigmar^c$ параметрами
  \begin{equation*}
    \begin{gathered}
      \qc{i} = \lambda \qcn{1}{i} + (1-\lambda) \qcn{2}{i},\ i \in I,                               \\
      \pac{s}{i} = \left(
        \lambda \qcn{1}{i} \pacn{1}{s}{i} + (1-\lambda) \qcn{2}{i} \pacn{2}{s}{i}
      \right) / \qc{i},\ i \in I,\ s \in S.
    \end{gathered}
  \end{equation*}
  Подставив эти выражения в \eqref{ch2:lower-bound:eq:K1(q,pi)}, для любого $j \in J$ имеем:
  \begin{multline*}
    \K{1}(\pd, \sigmas[c]{1}, j) =
    \sum_{i \in I} \sum_{s \in S} \qc{i} \pac{s}{i} \as(i, j) =                               \\
    = \sum_{i \in I, \, s \in S}
    \qc{i} \frac{
      (\lambda \qcn{1}{i} \pacn{1}{s}{i} + (1-\lambda) \qcn{2}{i} \pacn{2}{s}{i})
    }{\qc{i}} \as(i,j) = \\
    = \lambda \K{1}(\pd_1, \sigmas[1]{1}, j) +
    (1-\lambda)\K{1}(\pd_2, \sigmas[2]{1}, j).
  \end{multline*}
  Осредняя это равенство по произвольному $\tau \in \Delta(J)$, получим \eqref{ch2:lower-bound:lemma:convex-combination:eq:1} при $n = 1$.
  Предположим, что утверждение имеет место при любых $n \leqslant N$.
  Поскольку для каждого $i \in I$
  \[
    \pad{i}
    = \frac{
      \lambda \qcn{1}{i}
    }{\qc{i}} \padn{1}{i}
    + \frac{
      (1-\lambda) \qcn{2}{i}
    }{\qc{i}} \padn{2}{i},
  \]
  для $\sigmaran{1}{i},\ \sigmaran{2}{i}$ найдется такая стратегия первого игрока $\sigmaran{c}{i}$ в игре $\theG{N}(\pad{i})$, что для любой стратегии $\taura{i}$
  \begin{equation*}
    \begin{aligned}
      \K{N}(\pad{i}, \sigmaran{c}{i}, \taura{i})
      = \frac{\lambda \qcn{1}{i}}{\qc{i}} &\K{N}(\padn{1}{i}, \sigmaran{1}{i}, \taura{i}) + \\
      + \frac{(1-\lambda) \qcn{2}{i}}{\qc{i}} &\K{N}(\padn{2}{i}, \sigmaran{2}{i}, \taura{i}).
    \end{aligned}
  \end{equation*}
  В результате для $\sigmar^c = (\sigmas[c]{1}, \sigmaran{c}{i}) \in \Sigma$ и любой стратегии $\taur = (\taus{1}, \taura{i}) \in \Tau'$ в игре $\theG{N+1}(\pd)$ справедливо равенство
  \begin{align*}
    &\K{N+1}(\pd, \sigmar^c, \taur) =
    \K{1}(\pd, \sigmas[c]{1}, \taus{1}) +
    \sum_{i \in I} \qc{i} \K{N}(\pad{i}, \sigmaran{c}{i}, \taura{i}) =                             \\
    &= \lambda \K{1}(\pd_1, \sigmas[1]{1}, \taus{1}) +
    (1-\lambda) \K{1}(\pd_2, \sigmas[2]{1}, \taus{2}) +                                          \\
    &+ \sum_{i \in I} \qc{i} \left(
      \frac{\lambda \qcn{1}{i}}{\qc{i}} \K{N}(\padn{1}{i}, \sigmaran{1}{i}, \taura{i}) +
      \frac{(1-\lambda) \qcn{2}{i}}{\qc{i}} \K{N}(\padn{2}{i}, \sigmaran{2}{i}, \taura{i})
    \right) =                                                                          \\
    &= \lambda \K{N+1}(\pd_1, \sigmar^1, \taur) +
    (1-\lambda)\K{N+1}(\pd_2, \sigmar^2, \taur).
  \end{align*}
  Утверждение доказано.
  Из него получаем, что
  \begin{align*}
    \L{n}(\pd, \sigmar^c)
    &= \inf_{\taur \in \Tau'} \K{n}(\pd, \sigmar^c, j) \geqslant\\
    &\geqslant \lambda \min_{\taur \in \Tau'} \K{n}(\pd, \sigmar^1, j)
      + (1-\lambda) \min_{\taur \in \Tau'} \K{n}(\pd, \sigmar^2, j) =\\
    &= \lambda \L{n}(\pd_1, \sigmar^1) + (1-\lambda) \L{n}(\pd_2, \sigmar^2).
  \end{align*}
  Отсюда следует справедливость основного утверждения леммы.
\end{proof}

Обозначим $e^s$ вырожденное вероятностное распределение с носителем в точке $s$.
Пусть $\pd^x(l, r) \in \Theta(x)$ --- распределение с носителем $\{l, r\},\ l<r$.
При этом распределении вероятности реализации состояний $l$ и $r$ равны $(r-x)/(r-l)$ и $(x-l)/(r-l)$ соответственно, а дисперсия
\[
  \D \pd^x(l, r) = (x - l)(r - x).
\]
Как показано в \cite{domansky11}, любое распределение $\pd = (\pc{s},\ s \in S) \in \Theta(x)$ может быть представлено в виде выпуклой комбинации распределений с двухточечными носителями следующим образом:
\begin{gather}
  \label{ch2:lower-bound:eq:prob-decomp-sum}
  \pd = \begin{cases}
    \displaystyle
    \pc{x} e^x + \sum_{r=x+1}^\infty \sum_{l=0}^{x-1} \alpha_{l,r}(\pd) \pd^x(l, r),\ & x \in S,\\
    \displaystyle
    \sum_{r=\floor{x+1}}^\infty \sum_{l=0}^{\ceil{x-1}} \alpha_{l,r}(\pd) \pd^x(l, r),\ & x \notin S,\\
  \end{cases}\\
  \alpha_{l,r}(\pd) = (r-l) \pc{l} \pc{r} / \sum_{t=0}^{\ceil{x-1}} \pc{t} (x-t). \nonumber
\end{gather}

Обозначим через $\LL$ банахово пространство последовательностей $(l^s,\ s \in S)$ с нормой $\norm{l} = \sum_{s = 0}^\infty s^2 |l^s|$.
Множества $\PM$ и $\Theta(x)$ являются выпуклыми замкнутыми подмножествами пространства $\LL$.

\begin{lemma}
  \label{ch2:lower-bound:lemma:subseq-convergence}
  Пусть последовательность $\{l_n\} \subset \LL$ такая, что для любых $s \in S$ и $n \geqslant 1$ верно $l_n^s \geqslant 0,\ l_n^s \leqslant l_{n+1}^s$.
  Тогда если существует ее сходящаяся по норме подпоследовательность $\{l_{j_n}\}$, то и сама последовательность $\{l_n\}$ сходится по норме.
\end{lemma}
\begin{proof}
  Пусть подпоследовательность $\{l_{j_n}\}$ сходится.
  Тогда для любого $\varepsilon > 0$ существует $M$ такое, что для любых $f, g \geqslant M$ выполнено 
  $\norm{l_{j_f} - l_{j_g}} \leqslant \varepsilon$.
  Положим $N$ равное $j_M$.
  
  Для любых $m \geqslant n \geqslant N$ можно найти $q$ такое, что $j_q \geqslant m$.
  В силу покомпонентной монотонности последовательности $\{l_n\}$ выполнено неравенство
  \begin{gather*}
    \norm{l_m - l_n} =
    \sum_{s=0}^\infty s^2 |l_m^s - l_n^s| \leqslant
    \sum_{s=0}^\infty s^2 |l_{j_q}^s - l_{j_M}^s| =
    \norm{l_{j_q} - l_{j_M}} \leqslant \varepsilon.
  \end{gather*}
  Отсюда последовательность $\{l_n\}$ фундаментальна и в силу полноты пространства $\LL$ сходится.
\end{proof}

\begin{lemma}
  \label{ch2:lower-bound:lemma:decomp-convergence}
  Для любого распределения $\pd \in \Theta(x)$ ряд в разложении (\ref{ch2:lower-bound:eq:prob-decomp-sum}) сходится к $\pd$ по норме.
\end{lemma}
\begin{proof}
  Проведем доказательство для $x \in S$.
  Рассмотрим последовательность
  \begin{equation*}
    s_n = \pc{x} e^x + \sum_{r=x+1}^n \sum_{l=0}^{x-1} \alpha_{l,r}(\pd) \pd^x(l, r).
  \end{equation*}

  Тогда для $m \geqslant n$ справедливо
  \begin{align*}
    \norm{s_m-s_n}
    &= \norm{
      \sum_{r=n+1}^m \left(
      \pc{r} (r-x) \frac{\sum_{l=0}^{x-1} \pc{l} e^l}{\sum_{t=0}^{x-1} \pc{t} (x-t)} + \pc{r} e^r
      \right)} =\\
    &= \sum_{r=n+1}^m \pc{r} \left(
      r^2 + (r-x) \frac{\sum_{t=0}^{x-1} \pc{t} t^2}{\sum_{t=0}^{x-1} \pc{t} (x-t)}
      \right).
  \end{align*}
  Так как $\D \pd < \infty$, последовательность $s_n$ --- фундаментальна.
  Ее сходимость к $\pd$ следует из покомпонентного равенства векторов вероятностных распределений:
  \begin{gather*}
    \sum_{r=x+1}^\infty \alpha_{l,r}(\pd) \frac{r-x}{r-l} = \pc{l},\ l = \overline{0, x-1}, \\
    \sum_{l=0}^{x-1} \alpha_{l,r}(\pd) \frac{x-l}{r-l} = \pc{r},\ r = \overline{x+1, \infty}.
  \end{gather*}
  Отсюда в силу леммы~\ref{ch2:lower-bound:lemma:subseq-convergence} получаем, что ряд в разложении~\eqref{ch2:lower-bound:eq:prob-decomp-sum} сходится к $\pd$ по норме.

  Доказательство для нецелых значений $x$ проводится аналогично.
\end{proof}

В силу того, что функционал $\LowV{n}(\pd)$ вогнут на $\PM$ и по теореме~\ref{ch2:upper-bound:theorem} ограничен на данном множестве, то он непрерывен на $\PM$ (см.~\cite[Теорема 1.7.1]{polovinkin04}).
Отсюда и из леммы~\ref{ch2:lower-bound:lemma:decomp-convergence} следует, что для распределений $\pd \in \Theta(x)$ выполнено
\begin{equation*}
  \left\{
  \begin{aligned}
    \LowV{n}(\pd) &\geqslant
      \pc{x} \LowV{n}(e^x) + \sum_{r=x+1}^\infty \sum_{l=0}^{x-1} \alpha_{l,r}(\pd) \LowV{n} \left( \pd^x(l, r) \right),&\ x \in S,\\
    \LowV{n}(\pd) &\geqslant 
      \sum_{r=\floor{x+1}}^\infty \sum_{l=0}^{\ceil{x-1}} \alpha_{l,r}(\pd) \LowV{n} \left( \pd^x(l, r) \right),&\ x \notin S.
  \end{aligned}
  \right.
\end{equation*}

Из данных неравенств, теоремы~\ref{ch2:upper-bound:theorem} и леммы~\ref{ch2:lower-bound:lemma:convex-combination} следует, что для доказательства совпадения верхней и нижней оценок выигрыша в игре $\theG{\infty}(\pd)$ можно ограничиться рассмотрением только распределений %
$\pd = \pd^{k+\beta}(l, r) \in \Theta(k + \beta), \; k \in S,\ l = \overline{0, k},\ r = \overline{k+1, \infty}$.
Для таких распределений мы построим стратегию первого игрока $\sigmar^*$, для которой $\L{\infty}(\pd, \sigmar^*) = \H{\infty}(\pd)$.
Отсюда будет следовать, что $\V{\infty}(\pd) = \H{\infty}(\pd)$, а $\sigmar^*$ и $\taur^*$ --- оптимальные стратегии игроков в игре $\theG{\infty}(\pd)$.

Обозначим $\sigmak$ ход первого игрока, состоящий в выборе ставки из множества $\{k, k+1\}$.
Ход $\sigmak$ определяется заданием полных вероятностей $\qc{k}, \qc{k+1}$ и апостериорных распределений $\pad{k}, \pad{k+1}$, причем $\qc{k} + \qc{k+1} = 1$.
Следующая лемма является обобщением утверждения~\ref{ch1:prop:stage-payoff} из главы~\ref{chapt1}.
\begin{lemma}
  \label{ch2:lower-bound:lemma:stage-payoff}
  При использовании $\sigmak$ одношаговый выигрыш первого игрока равен
  \begin{equation*}
    \K{1}(\pd, \sigmak, j) = \begin{cases}
      \E \pd - \beta k - (1-\beta) j - \beta \qc{k+1}, &\; j < k, \\
      (\E \pad{k+1} - k - \beta) \qc{k+1}, &\; j = k, \\
      (k + \beta - \E \pad{k}) \qc{k}, &\; j = k+1, \\
      (1-\beta) k + \beta j - \E \pd + (1-\beta) \qc{k+1}, &\; j > k + 1.
    \end{cases}
  \end{equation*}
\end{lemma}
\begin{proof}
  Можно показать, что
  \begin{equation*}
    \as(\sigmak, j) = \begin{cases}
      s - \beta k - (1-\beta) j - \beta \sigmasc[s]{k+1},     & \; j < k,   \\
      (s - k - \beta) \sigmasc[s]{k+1},                       & \; j = k,   \\
      (k + \beta - s) \sigmasc[s]{k},                           & \; j = k+1, \\
      (1-\beta) k + \beta j - s + (1-\beta) \sigmasc[s]{k+1}, & \; j > k + 1.
    \end{cases}
  \end{equation*}
  Осредняя это равенство по произвольному $\tau \in \Delta(J)$, устанавливаем справедливость утверждения леммы.
\end{proof}

Определим стратегию $\sigmar^*$ первого игрока в игре $\theG{\infty}(\pd)$.
Введем множество распределений
\begin{equation*}
  P(l,r) = \left\{
    \pd^k(l, r), \, \pd^{s+\beta}(l, r), \, k = \overline{l,r}, s = \overline{l,r-1}
  \right\}.
\end{equation*}
При $\pd \in P(l,r)$ первый ход $\sigmas[*]{1}$ стратегии $\sigmar^*$ определяется следующим образом.
Если $\pd = \pd^l(l,r)$ или $\pd = \pd^r(l,r)$, то первый игрок использует ставки $l$ и $r$, соответственно, с вероятностью 1.
В противном случае он использует $\sigmak$ с параметрами из таблицы~\ref{ch2:tab:insider-strategy}.

\begin{table}[htb]
  \centering
  \renewcommand{\arraystretch}{1.5}
  \captionsetup{width=17cm}
  \caption{Параметры хода $\sigmas[*]{1}$ при $\pd \in P(l, r)$}
  \label{ch2:tab:insider-strategy}
  \begin{tabular}{|P{3cm}||P{3cm}|P{3cm}|P{3cm}|P{3cm}|}
    \hline
    \hline
    $\pd$                 & $\qc{k}$     & $\qc{k+1}$ & $\pad{k}$               & $\pad{k+1}$           \\ \hline
    $\pd^k(l, r)$         & $\beta$   & $1-\beta$ & $\pd^{k-1+\beta}(l, r)$ & $\pd^{k+\beta}(l, r)$ \\ \hline
    $\pd^{k+\beta}(l, r)$ & $1-\beta$ & $\beta$   & $\pd^k(l, r)$           & $\pd^{k+1}(l, r)$     \\
    \hline
    \hline
    \multicolumn{1}{c}{}
    \vspace{-2.5em}
  \end{tabular}
\end{table}

На последующих шагах игры ход $\sigmas[*]{1}$ применяется рекурсивно для соответствующих значений апостериорных вероятностей.
В результате определили стратегию $\sigmar^*$ для распределения $\pd \in P(l,r)$.

Обозначим $L^x_n(\sigmar) = \L{n}(\pd^x(l,r), \sigmar),\ \sigmar \in \Sigma$.
Следующая теорема является обобщением утверждения~\ref{ch1:prop:lower:recurrence-solution}.
\begin{theorem}
  \label{ch2:lower-bound:theorem}
  Пусть $\beta \in (0, 1)$.
  При использовании стратегии $\sigmar^*$ в игре $\theG{\infty}(\pd)$ для
  распределения %
  $\pd \in P(l,r)$ %
  гарантированный выигрыш первого игрока удовлетворяет следующей системе:
  \begin{subequations}
    \label{ch2:lower-bound:eq:Linf-recurrence}
    \begin{equation}
      L^k_\infty(\sigmar^*) =
      \beta L^{k-1+\beta}_\infty(\sigmar^*) + (1-\beta) L^{k+\beta}_\infty(\sigmar^*),\ k = \overline{l + 1, r - 1},
    \end{equation}
    \begin{equation}
      L^{k+\beta}_\infty(\sigmar^*) =
      \beta(1-\beta) + (1-\beta) L^k_\infty(\sigmar^*) + \beta L^{k+1}_\infty(\sigmar^*),\ k = \overline{l, r - 1}.
    \end{equation}
    \begin{equation}
      L^l_\infty(\sigmar^*) = L^r_\infty(\sigmar^*) = 0.
    \end{equation}
  \end{subequations}
  Ее решение дает нижнюю оценку выигрыша первого игрока, равную
  \begin{equation*}
    \label{ch2:lower-bound:eq:Linf-recurrence-solution}
    \L{\infty}(\pd^{k+\beta}(l, r), \sigmar^*) = \frac{(r-k-\beta)(k+\beta-l) + \beta(1-\beta)}{2}.
  \end{equation*}
\end{theorem}
\begin{proof}
  Для $\pd \in P(l,r)$ стратегия $\sigmar^*$ определяется аналогично оптимальной стратегии из главы~\ref{chapt1} с заменой $0$ и $m$ на $l$ и $r$ соответственно.
  Из леммы~\ref{ch2:lower-bound:lemma:stage-payoff} нетрудно вывести, что
  \begin{equation*}
    L^k_1(\sigmar^*) = 0,\quad
    L^{k+\beta}_1(\sigmar^*) = \beta(1-\beta).
  \end{equation*}
  Для $k = \overline{l+1,r-1}$ имеем
  \begin{align*}
    L^k_\infty(\sigmar^*)
    &= L^k_1(\sigmar^*) + \qc{k} L^{k-1+\beta}_\infty(\sigmar^*) + \qc{k+1} L^{k+\beta}_\infty(\sigmar^*) = \\
    &= \beta L^{k-1+\beta}_\infty(\sigmar^*) + (1-\beta) L^{k+\beta}_\infty(\sigmar^*).
  \end{align*}
  Аналогично для $k = \overline{l, r-1}$ получаем
  \begin{align*}
    L^{k+\beta}_\infty(\sigmar^*)
    &= L^{k+\beta}_1(\sigmar^*) + \qc{k} L^k_\infty(\sigmar^*) + \qc{k+1} L^{k+1}_\infty(\sigmar^*) = \\
    &= \beta(1-\beta) + (1-\beta) L^k_\infty(\sigmar^*) + \beta L^{k+1}_\infty(\sigmar^*).
  \end{align*}
  Полученная система \eqref{ch2:lower-bound:eq:Linf-recurrence} является системой с трехдиагональной матрицей и решается методом прогонки.
\end{proof}

Поскольку
$
  \D \pd^{k+\beta}(l, r) = (r-k-\beta)(k+\beta-l),
$
а распределение $\pd^{k+\beta}(l, r)$ удовлетворяет условию теоремы~\ref{ch2:upper-bound:theorem} с $\eta = 1$, выражения для $\H{\infty}(\pd^{k+\beta}(l, r))$ и $\L{\infty}(\pd^{k+\beta}(l, r), \sigmar^*),\ k = \overline{l,r-1}$ совпадают.
Из теоремы~\ref{ch2:upper-bound:theorem} также следует, что
\begin{equation*}
  \H{\infty}(\pd^l(l,r)) = \H{\infty}(\pd^r(l,r)) = 0.
\end{equation*}
В самом деле, распределения $\pd^l(l,r)$ и $\pd^r(l,r)$ удовлетворяют условию теоремы~\ref{ch2:upper-bound:theorem} с $\eta = 1-\beta$ и имеют нулевую дисперсию.

Для произвольного распределения $\pd \in \Theta(k),\ k \in S$, стратегию $\sigmar^*$ определим следующим образом.
Если реализуется состояние $s = k$, то гарантированный выигрыш первого игрока не превышает $0$ и он прекращает игру.
Таким образом, первый игрок, следуя стратегии $\sigmar^*$, прекращает игру с вероятностью $\pc{k}$.
В противном случае игрок использует конструкцию леммы~\ref{ch2:lower-bound:lemma:convex-combination} для построения стратегии, соответствующей выпуклой комбинации распределений $\pd^k(l, r)$ в разложении $\pd$.
Первый ход такой стратегии использует две ставки $k$ и $k+1$ с полными вероятностями $(1-\pc{k})\beta$ и $(1-\pc{k})(1-\beta)$ соответственно.
Апостериорные вероятностные распределения являются выпуклыми комбинациями соответствующих апостериорных двухточечных распределений и даются следующими формулами:
\begin{gather*}
  \pad{k} = \frac{1}{1-\pc{k}} \sum_{r=k+1}^\infty \sum_{l=0}^{k-1} \alpha_{l,r}(\pd) \pd^{k-1+\beta}(l, r), \\
  \pad{k+1} = \frac{1}{1-\pc{k}} \sum_{r=k+1}^\infty \sum_{l=0}^{k-1} \alpha_{l,r}(\pd) \pd^{k+\beta}(l, r).
\end{gather*}
Для распределений $\pd$ со счетным носителем сходимость по норме данных рядов устанавливается аналогично доказательству леммы~\ref{ch2:lower-bound:lemma:decomp-convergence}.
Аналогичные рассуждения справедливы и для распределений $\pd \in \Theta(k+\beta) \cup \Lambda(k, k+\beta),\ k \in S$.

\section{Решение игры $\mathbf{G^\beta_\infty(\p)}$}
\label{ch2:sec:game-solution}

Подчеркнем, что приведенная в п.~\ref{ch2:sec:lower-bound} стратегия инсайдера $\sigmar^*$ определена только при $\beta \in (0, 1)$.

Нетрудно проверить справедливость следующего равенства:
\begin{equation*}
  \as[r](i, j) = \as[l][1-\beta](r+l-i, r+l-j).
\end{equation*}
Из него вытекает, что решение игры $\theG{\infty}(\pd^x(l, r))$ сводится к решению игры $\theG[1-\beta]{\infty}(\pd^{l+r-x}(l,r))$.
При этом ставки, используемые в соответствующих смешанных стратегиях инсайдера, симметричны относительно точки $(l+r)/2$.
Аналогичные рассуждения справедливы для игры $\theG{\infty}(\pd)$ при любом распределении $\pd \in \PM$.

Оптимальная стратегия $\sigmar^*$ инсайдера в игре $\theG{\infty}(\pd)$ при $\beta = 1$ найдена в~\cite{domansky11}.
Решение $\theG{\infty}(\pd)$ при $\beta = 0$ может быть получено при помощи описанной выше конструкции из решения $\theG{\infty}(\pd)$ при $\beta = 1$.
Таким образом, при любом $\beta \in [0, 1]$ справедлива следующая
\begin{theorem}
  \label{ch2:solution:theorem}
  Игра $\theG{\infty}(\pd)$ имеет значение
  \[
    \V{\infty}(\pd) = \H{\infty}(\pd) = \L{\infty}(\pd, \sigmar^*),
  \]
  а $\sigmar^*$ и $\taur^*$ --- оптимальные стратегии игроков.
\end{theorem}

\begin{figure}[tbh]
  \centering
  \begin{tikzpicture}
    [
    auto,node distance=2.0cm,
    trans/.style={->,shorten >=1pt,>=stealth',semithick},
    state/.style={shape=circle,draw,minimum size=2mm}
    ]
    \node[state,label={$\pd^l(l,r)$}] (p0) {};
    \node[state,right=of p0,label={$\pd^{l+\beta}(l,r)$}] (p1) {}; 
    \node[state,right=of p1,label={$\pd^{l+1}(l,r)$}] (p2) {};
    \node[right=of p2] (others) {$\ldots$};
    \node[state,right=of others,label={$\pd^{r-1+\beta}(l,r)$}] (p2mm1) {};
    \node[state,right=of p2mm1,label={$\pd^r(l,r)$}] (p2m) {};
    
    \path [trans]
    (p0) edge [loop left,min distance=10mm,out=205,in=155] node {$1$} (p0)
    (p1) edge[bend right] node[below] {$1-\beta$} (p0)
    (p1) edge[bend left] node[below] {$\beta$} (p2)
    (p2) edge[bend left] node[below] {$\beta$} (p1)
    (p2) edge[bend right] node[below] {$1-\beta$} (others)
    (others) edge[bend left] node[below] {$\beta$} (p2mm1)
    (p2mm1) edge[bend left] node[below] {$1-\beta$} (others)
    (p2mm1) edge[bend right] node[below] {$\beta$} (p2m)
    (p2m) edge[loop right,min distance=10mm,out=25,in=-25] node {$1$} (p2m)
    ;
  \end{tikzpicture}
  \caption[Последовательность апостериорных вероятностей]{Случайное блуждание последовательности апостериорных вероятностей, порожденное $\sigmar^*$}
  \label{ch2:fig:posterior-1}
\end{figure}

Применение первым игроком стратегии $\sigmar^*$ при $\pd \in P(l, r)$ порождает случайное блуждание последовательности апостериорных вероятностей, изображенное на рисунке~\ref{ch2:fig:posterior-1}, которое в отличие от \cite{domansky11} происходит по более широкому множеству и уже не является симметричным, за исключением случая $\beta = 1/2$.

\section{Вторая оптимальная стратегия инсайдера в игре~$\mathbf{G^\beta_\infty(\p)}$}
\label{ch2:sec:optimal-strategy2}
В дополнение к стратегии $\sigmar^*$ построим еще одну оптимальную стратегию $\xi^*$ инсайдера.
Введем множество распределений %
\begin{equation*}
  P'(l,r) =
  \{\pd^l(l, r), \pd^r(l, r)\}
  \cup
  \left\{
    \pd^{k+\beta}(l, r), \, k = \overline{l, r-1}
  \right\}.
\end{equation*}
При $\pd \in P'(l,r)$ первый ход $\xi^*_1$ стратегии $\xi^*$ определяется следующим образом.
Если $\pd = \pd^l(l,r)$ или $\pd = \pd^r(l,r)$, то первый игрок использует ставки $l$ и $r$ соответственно с вероятностью 1.
В противном случае он использует $\sigmak$ с параметрами из таблицы~\ref{ch2:tab:insider-strategy2}.
\begin{table}[htb]
  \centering
  \renewcommand{\arraystretch}{1.5}
  \captionsetup{width=17cm}
  \caption{Параметры хода $\xi^*_1$ при $\pd \in P'(l, r)$}
  \label{ch2:tab:insider-strategy2}
  \begin{tabular}{|P{3cm}||P{3cm}|P{3cm}|P{3cm}|P{3cm}|}
    \hline
    \hline
    $\pd$                   & $\qc{k}$ & $\qc{k+1}$                 & $\pad{k}$                & $\pad{k+1}$                                      \\
    \hline
    $\pd^{l+\beta}(l, r)$           & $\frac{1}{1+\beta}$       & $\frac{\beta}{1+\beta}$ & $\pd^l(l, r)$           & $\pd^{l+1+\beta}(l, r)$ \\
    \hline
    $\pd^{r-1+\beta}(l, r)$         & $\frac{1-\beta}{2-\beta}$ & $\frac{1}{2-\beta}$     & $\pd^{r-2+\beta}(l, r)$ & $\pd^r(l, r)$           \\
    \hline
    $\pd^{k+\beta}(l, r)$   & $\frac{1}{2}$             & $\frac{1}{2}$           & $\pd^{k-1+\beta}(l, r)$ & $\pd^{k+1+\beta}(l, r)$  \\
    \hline
    \hline
    \multicolumn{1}{c}{}
    \vspace{-2.5em}
  \end{tabular}
\end{table}

Для остальных распределений $\pd$ стратегия $\xi^*$ определяется аналогично тому, как это было сделано для стратегии $\sigmar^*$.

Использование стратегии $\xi^*$ при $\pd \in P'(l, r)$ порождает случайное блуждание последовательности апостериорных вероятностей, изображенное на рисунке~\ref{ch2:fig:posterior-2}.
Данное блуждание симметрично с вероятностями перехода в соседние состояния равными $1/2$, симметрия нарушается только в крайних и соседних к ним состояниях.

\begin{figure}[tbh]
  \centering
  \begin{tikzpicture}
    [
    auto,yscale=1.1,node distance=2cm,
    trans/.style={->,shorten >=1pt,>=stealth',semithick},
    state/.style={shape=circle,draw,minimum size=2mm}
    ]
    \node[state,label={$\pd^l(l,r)$}] (p0) {};
    \node[state,right=of p0,label={$\pd^{l+\beta}(l,r)$}] (p1) {}; 
    \node[state,right=of p1,label={$\pd^{l+1+\beta}(l,r)$}] (p2) {};
    \node[right=of p2] (others) {$\ldots$};
    \node[state,right=of others,label={$\pd^{r-1+\beta}(l,r)$}] (p2mm1) {};
    \node[state,right=of p2mm1,label={$\pd^r(l,r)$}] (p2m) {};
    
    \path [trans]
    (p0) edge [loop left,min distance=10mm,out=205,in=155] node {$1$} (p0)
    (p1) edge[bend right] node[below] {$\frac{1}{1+\beta}$} (p0)
    (p1) edge[bend left] node[below] {$\frac{\beta}{1+\beta}$} (p2)
    (p2) edge[bend left] node[below] {$\frac{1}{2}$} (p1)
    (p2) edge[bend right] node[below] {$\frac{1}{2}$} (others)
    (others) edge[bend left] node[below] {$\frac{1}{2}$} (p2mm1)
    (p2mm1) edge[bend left] node[below] {$\frac{1-\beta}{2-\beta}$} (others)
    (p2mm1) edge[bend right] node[below] {$\frac{1}{2-\beta}$} (p2m)
    (p2m) edge[loop right,min distance=10mm,out=25,in=-25] node {$1$} (p2m)
    ;
  \end{tikzpicture}
  \caption[Последовательность апостериорных вероятностей]{Случайное блуждание последовательности апостериорных вероятностей, порожденное $\xi^*$}
  \label{ch2:fig:posterior-2}
\end{figure}

Из леммы~\ref{ch2:lower-bound:lemma:stage-payoff} можно вывести, что
\begin{gather*}
  L^{l+\beta}_1(\xi^*_1) = \frac{\beta}{1+\beta},\quad
  L^{r-1+\beta}_1(\xi^*_1) = \frac{1-\beta}{2-\beta},           \\
  L^{k+\beta}_1(\xi^*_1) = \frac{1}{2},\ k = \overline{l+1,r-2}.
\end{gather*}
Отсюда следует, что при использовании стратегии $\xi^*$ в игре $\theG{\infty}(\pd)$ для распределения %
$\pd \in P'(l,r)$ %
гарантированный выигрыш первого игрока удовлетворяет следующей системе:
\begin{subequations}
  \label{ch2:lower-bound:eq:Linf-recurrence-2}
  \begin{equation}
    L^{l+\beta}_{\infty}(\xi^*) =
    \frac{\beta}{1+\beta} + \frac{1}{1+\beta} L^l_{\infty}(\xi^*) + \frac{\beta}{1+\beta} L^{l+1+\beta}_{\infty}(\xi^*),
  \end{equation}
  \begin{equation}
    L^{r-1+\beta}_{\infty}(\xi^*) =
    \frac{1-\beta}{2-\beta} + \frac{1-\beta}{2-\beta} L^{r-2+\beta}_{\infty}(\xi^*) + \frac{1}{2-\beta} L^r_{\infty}(\xi^*),
  \end{equation}
  \begin{equation}
    L^{k+\beta}_{\infty}(\xi^*) =
    \frac{1}{2} + \frac{1}{2} L^{k-1+\beta}_{\infty}(\xi^*) + \frac{1}{2} L^{k+1+\beta}_{\infty}(\xi^*),\ k = \overline{l+1, r-2},
  \end{equation}
  \begin{equation}
    L^l_{\infty}(\xi^*) = L^r_{\infty}(\xi^*) = 0.
  \end{equation}
\end{subequations}

Нетрудно проверить, что подстановкой
\begin{gather*}
  \H{\infty}(\pd^{k+\beta}(l,r)) = \frac{(r-k-\beta) (k+\beta-l) + \beta(1-\beta)}{2}, \\
  \H{\infty}(\pd^l(l,r)) = \H{\infty}(\pd^r(l,r)) = 0,
\end{gather*}
вместо $L^{k+\beta}_{\infty}(\xi^*),\ k = \overline{l,r-1}$, $L^l_\infty(\xi^*)$ и $L^r_\infty(\xi^*)$ соответственно данные равенства обращаются в тождества.
Отсюда, как и для стратегии $\sigmar^*$, следует, что стратегия $\xi^*$ является оптимальной.

Отметим, что в отличие от стратегии $\sigmar^*$ стратегия $\xi^*$ определена при $\beta \in [0, 1]$ и совпадает с оптимальной стратегией инсайдера из \cite{domansky11} при $\beta = 1$.
При этом обе стратегии $\sigmar^*$ и $\xi^*$ порождают существенно различные случайные блуждания апостериорных вероятностей.

\clearpage
}

%%% Local Variables:
%%% mode: latex
%%% TeX-master: "../dissertation"
%%% End:
