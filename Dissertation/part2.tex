\chapter{Теоретико-игровая модель биржевых торгов с дискретными ставками и счетным множеством состояний} \label{chapt2}
{
\newcommand{\as}[1][\beta]{\ensuremath{a^{s,#1}}}
\newcommand{\s}{\ensuremath{s}}
\newcommand{\q}{\ensuremath{\overbar{q}}}
\newcommand{\theG}[1][n]{\ensuremath{G^\beta_{#1}}}
\newcommand{\K}[1][n]{\ensuremath{K^\beta_{#1}}}
\newcommand{\V}[1][n]{\ensuremath{V^\beta_{#1}}}
\newcommand{\High}[1][\ensuremath{\infty}]{\ensuremath{H^\beta_{#1}}}
\newcommand{\sigmav}{\ensuremath{\overbar{\sigma}}}
\newcommand{\tauv}{\ensuremath{\overbar{\tau}}}
\newcommand{\xiv}{\ensuremath{\overbar{\xi}}}
\newcommand{\sigmak}{\ensuremath{\hat{\sigma}}}
\newcommand{\Low}[1][\ensuremath{\infty}]{\ensuremath{L^\beta_{#1}}}
\newcommand{\norm}[1]{\left\lVert#1\right\rVert}
\newcommand{\LL}{L^1(\{s^2\})}
\newcommand{\LowV}[1][n]{\ensuremath{\underbar{V}^\beta_{#1}}}
\newcommand{\HighV}[1][n]{\ensuremath{\overbar{V}^\beta_{#1}}}
\newcommand{\MM}{\ensuremath{\overline{P}}}

В разделе~\ref{ch2:sec:intro} дано формальное описание модели рынка со счетным множеством состояний.
Раздел~\ref{ch2:sec:upper-bound} посвящен анализу стратегии неосведомленного игрока и получению оценки сверху на выигрыш инсайдера.
Построение стратегии инсайдера и получение оценки снизу его выигрыша проведено в разделе~\ref{ch2:sec:lower-bound}.
Рассматриваемая стратегия основана на представлении вероятностных распределений в виде суммы распределений с двухточечным носителем из~\cite{domansky11}.
Вопросы сходимости по норме рядов в разложении распределений с бесконечным носителем также исследованы в разделе~\ref{ch2:sec:lower-bound}.
В разделе~\ref{ch2:sec:game-solution} дана теорема о значении игры неограниченной продолжительности и приведена динамика апостериорных вероятностей при применении игроками оптимальных стратегий.
Раздел~\ref{ch2:sec:optimal-strategy2} посвящен построению второй оптимальной стратегии инсайдера, анализу случайных блужданий апостериорных вероятностей, порождаемых данной стратегией, а также сравнению результатов с результатами из~\cite{domansky11}.

Основные результаты данной главы опубликованы в работе~\cite{pyanykh:orm2016}.

\section{Описание модели рынка со счетным множеством состояний}
\label{ch2:sec:intro}

Рассмотрим модель рынка с дискретными ставками и множеством состояний $S = \Z_+$.
Перед началом игры случай выбирает состояние рынка $\s \in S$ в соответствии с вероятностным распределением $\p = (p^s, \; s \in S)$, имеющим конечную дисперсию состояния $\D \p < \infty$.
Множество всех таких распределений обозначим $\MM$.

На каждом шаге игры $t = \overline{1,n}, \; n \leqslant \infty$, игроки делают ставки $i_t \in I, \, j_t \in J$, где $I = J = \Z_+$.
В силу того, что игрок, предложивший б\'{о}льшую ставку, покупает акцию у другого по цене, равной выпуклой комбинации предложенных ставок, выплата первому игроку в состоянии $s$ равна
\begin{equation*}
  \as(i_t, j_t) =
  \begin{cases}
    (1-\beta) i_t + \beta j_t - s, &\; i_t < j_t, \\
    0, &\; i_t = j_t, \\
    s - \beta i_t - (1-\beta)j_t, &\; i_t > j_t.
  \end{cases}
\end{equation*}

На шаге $t$ обоим игрокам достаточно принимать в расчет лишь последовательность $(i_1, i_2, \ldots, i_{t-1})$ действий первого игрока на предыдущих ходах.
Это связано с тем, что информация, получаемая вторым игроком относительно состояния $s$, может передаваться лишь посредством действий первого игрока.
Подробное обсуждение данного факта можно найти в \cite{mertens15}.

Обозначим через $\Delta(X)$ совокупность всех вероятностных распределений на множестве $X$.

Стратегией первого игрока является последовательность ходов (отображений) $\sigmav = (\sigma_1, \ldots, \sigma_n)$, где $\sigma_t: S \times I^{t-1} \rightarrow \Delta(I)$.
Множество стратегий первого игрока обозначим $\Sigma$.

Стратегией второго игрока является последовательность ходов (отображений) $\tauv = (\tau_1, \ldots, \tau_n)$, где $\tau_t: I^{t-1} \rightarrow \Delta(J)$.
Множество стратегий второго игрока обозначим $\Tau$.

При использовании игроками стратегий $\sigmav$ и $\tauv$ ожидаемый выигрыш первого игрока равен
\begin{equation*}
  \K(\p, \sigmav, \tauv) =
  \E_{(\p, \sigmav, \tauv)} \sum_{t=1}^n \as(i_t, j_t),
\end{equation*}
где математическое ожидание берется по мере, индуцированной $\p$, $\sigmav$ и $\tauv$ на множестве $S \times I^n \times J^n$.
Заданную таким образом игру обозначим $\theG(\p)$.

Если для некоторых стратегий $\sigmav^* \in \Sigma,$ $\tauv^* \in \Tau$ выполняются равенства
\begin{equation*}
  \inf_{\tauv \in \Tau} \K(\p, \sigmav^*, \tauv) =
  \K(\p, \sigmav^*, \tauv^*) =
  \sup_{\sigmav \in \Sigma} \K(\p, \sigmav, \tauv^*) \eqdef
  \V(\p),
\end{equation*}
то говорят, что игра $\theG(\p)$ имеет значение $\V(\p)$, а стратегии $\sigmav^*$ и $\tauv^*$
называются оптимальными.

Нижнее и верхнее значения игры $\theG(\p)$ обозначим соответственно
\begin{equation*}
  \LowV(\p) =
    \sup_{\sigmav \in \Sigma}
    \inf_{\tauv \in \Tau}
    \K(\p, \sigmav, \tauv), \quad
  \HighV(\p) =
    \inf_{\tauv \in \Tau}
    \sup_{\sigmav \in \Sigma}
    \K(\p, \sigmav, \tauv).
\end{equation*}
Данные функции являются вогнутыми на $\MM$.
Доказательство этого утверждения проводится аналогично~\cite{domansky07}.

Аналогично тому, как это было сделано в главе~\ref{chapt1}, опишем опишем рекурсивную структуру игры $\theG(\p)$.
Представим стратегию первого игрока в виде $\sigmav = (\sigma, \sigmav^i, i \in I)$, где $\sigma$~--- его ход на первом шаге, а $\sigmav^i$ --- его стратегия в игре продолжительности $n-1$ в зависимости от ставки $i$, выбранной им на первом шаге.
Аналогично, стратегию второго игрока представим в виде $\tauv = (\tau, \tauv^i, \; i \in I)$.
%
Далее, обозначим $q^i$ полную вероятность, с которой первый игрок делает ставку $i \in I$, а $\q = (q^i, \; i \in I)$ --- соответствующее распределение.
Также обозначим $p^{s|i}$ апостериорную вероятность состояния $s$ в зависимости от ставки $i$ первого игрока, и $\p^i = (p^{s|i}, \, s \in S)$ --- соответствующее апостериорное распределение.
Тогда для функции выигрыша первого игрока в игре $\theG[n](\p)$ справедлива формула
\begin{equation*}
  \K[n](\p, \sigmav, \tauv) =
  \K[1](\p, \sigma, \tau) +
  \sum_{i \in I} q^i \K[n-1](\p^i, \sigmav^i, \tauv^i).
\end{equation*}

\section{Оценка сверху выигрыша первого игрока в игре~$\mathbf{G^\beta_\infty(\p)}$}
\label{ch2:sec:upper-bound}

В~\cite{domansky11} определена следующая чистая стратегия второго игрока $\tauv^k = (\tau^k_t,\ t = \overline{1, \infty})$:
\begin{equation*}
  \tau^k_1 = k, \quad
  \tau^k_t(i_{t-1}, j_{t-1}) = \begin{cases}
    j_{t-1} - 1, &\; i_{t-1} < j_{t-1},\\
    j_{t-1}, &\; i_{t-1} = j_{t-1},\\
    j_{t-1}+1, &\; i_{t-1} > j_{t-1}.
  \end{cases}
\end{equation*}
Другими словами, второй игрок делает ставку равную $k$ на первом шаге, а далее либо подражает инсайдеру, либо смещается на единицу к ставке инсайдера предыдущего шага.

\begin{lemma}
  \label{ch2:upper-bound:lemma:vector-payoffs}
  При применении стратегии $\tauv^k$ в игре $\theG(\p)$ второй игрок в состоянии $s$ гарантирует себе проигрыш не более
  \begin{gather*}
    h^s_n(\tauv^k) = \begin{cases}
      \displaystyle \sum_{t=0}^{n-1} (k-s-t-1+\beta)^+, \; s \leqslant k, \\
      \displaystyle \sum_{t=0}^{n-1} (s-k-t-\beta)^+, \; s > k.
    \end{cases} 
  \end{gather*}
  Для любого $s \in S$ последовательность $\left\{ h^s_n(\tauv^k) \right\}_{n = 1}^\infty$ не убывает, ограничена сверху и сходится к %
  \begin{equation}
    \label{ch2:upper-bound:eq:max-payoff}
    h^s_\infty(\tauv^k) = (s - k + 1 - 2\beta)(s-k)/2.
  \end{equation}
\end{lemma}
\begin{proof}
  Проведем доказательство индукцией по $n$ для случая $s > k$.
  При $n = 1$ оптимальный ответ первого игрока на $\tauv^k$ будет $i = k + 1$.
  Тогда его выигрыш в игре $\theG[1](\p)$ равен
  \begin{equation*}
    h^s_1(\tauv^k) = s - \beta(k+1) - (1-\beta)k = s - k - \beta.
  \end{equation*}
  База индукции проверена.
  Предположим, что утверждение верно при $n \leqslant N$.
  При $n = N + 1$ первый игрок имеет два разумных ответа на $\tauv^k$: ставка $i=k+1$, что соответствует покупке акции по наименьшей возможной цене, и ставка $i=k-1$, что соответствует продаже акции за наибольшую возможную цену.
  Найдем оценки выигрыша в каждом из случаев.
  Для $i=k+1$ выигрыш первого игрока не превосходит величины
  \begin{equation*}
    s - k - \beta + h^s_N(\tauv^{k+1}) = \sum_{t=0}^N(s-k-t-\beta)^+.
  \end{equation*}
  Аналогично для $i = k - 1$ тот же выигрыш не превосходит
  \begin{equation*}
    \beta k + (1-\beta)(k-1) - s + h^s_N(\tauv^{k-1}) = \sum_{t=0}^{N-2}(s-k-t-\beta)^+.
  \end{equation*}
  При $s \leqslant k$ формула для $h^s_n(\tauv^k)$ доказывается аналогично.
  Сходимость последовательности $\left\{  h^s_n(\tauv^k) \right\}_{n=1}^\infty$ к $h^s_\infty(\tauv^k)$ следует из равенств
  $h^s_n(\tauv^k) = h^s_{n+1}(\tauv^k)$ при $n \geqslant s - k$.
\end{proof}

Следующие множества распределений зададим ограничениями на математическое ожидание состояния:
\begin{gather*}
  \Theta(x) = \left\{ \p \in \MM: \E \p = x \right\}, \\
  \Lambda(x, y) = \left\{ \p \in \MM: x < \E \p \leqslant y \right\}.
\end{gather*}

Пусть $\tauv^*$ --- стратегия второго игрока, состоящая в применении $\tauv^k$ при $\p \in \Lambda(k-1+\beta,k+\beta)$.
Отметим, что при заданном распределении $\p$ выбор $k$ зависит от значения $\beta$.
Отсюда следует, что стратегия $\tauv^*$ также зависит от $\beta$.

\begin{theorem}
  \label{ch2:upper-bound:theorem}
  При использовании вторым игроком стратегии $\tauv^*$, выигрыш первого игрока в игре
  $\theG[\infty](\p)$ ограничен сверху функцией
  \begin{equation*}
    \High(\p) = \inf_{k \in J} \sum_{s \in S} p^s  h^s_\infty(\tauv^k).
  \end{equation*}
  Функция $\High(\p)$ является кусочно-линейной вогнутой с областями линейности $\Lambda(k - 1 + \beta, k + \beta)$ и областями недифференцируемости $\Theta(k+\beta)$ при $k \in S$.
  Для распределений $\p$ с $\E \p = k - 1 + \beta + \eta, \; \eta \in (0, 1]$, ее значение равно
  \begin{equation}
    \label{ch2:upper-bound:eq:H(p)}
    \High(\p) = \left( \D \p + \beta(1-\beta) - \eta(1-\eta) \right)/2.
  \end{equation}
\end{theorem}
\begin{proof}
  Воспользовавшись \eqref{ch2:upper-bound:eq:max-payoff}, получим
  \begin{equation}
    \label{ch2:upper-bound:theorem:eq:1}
    \begin{gathered}
    \sum_{s \in S} p^s h^s_\infty(\tauv^j) = \bigl(
      j^2 + (2\beta - 1 - 2 \E \p)j - (2\beta - 1) \E \p + \E \p^2 
    \bigr)/2.
    \end{gathered}
  \end{equation}
  
  Квадратичная функция $f(x) = x^2 + (2\beta - 1 - 2\E \p)x$ достигает минимума
  при $x = \E \p - \beta + 1/2$. Отсюда при $\p \in \Lambda(k - 1 + \beta, k +
  \beta)$ выражение \eqref{ch2:upper-bound:theorem:eq:1} достигает минимума при $j =
  k$. Равенство \eqref{ch2:upper-bound:eq:H(p)} проверяется непосредственной
  подстановкой $\E \p = k - 1 + \beta + \eta$ в \eqref{ch2:upper-bound:theorem:eq:1}.
\end{proof}

Заметим, что в данном случае наблюдается сдвиг областей линейности на $\beta$ относительно $\E \p$ в сравнении с областями из \cite{domansky11}.

\section{Оценка снизу выигрыша первого игрока в игре~$\mathbf{G^\beta_\infty(\p)}$}
\label{ch2:sec:lower-bound}

Перейдем к описанию стратегии первого игрока, гарантирующей ему выигрыш не менее $\High(\p)$.
Пусть $\sigma^s_i$ --- компонента хода $\sigma$ первого игрока, т.е. вероятность сделать ставку $i$ в состоянии $s$.
По правилу Байеса $\sigma^s_i = p^{s|i} q^i / p^s$.
В частности, справедливы равенства $\sum_{s \in S} \sigma^s_i p^s = q^i,\ i \in I$.
Таким образом, ход $\sigma$ первого игрока можно определить, задав следующие параметры: полные вероятности $q^i$ сделать ставку $i$ и апостериорные вероятности $p^{s|i}$ для $i \in I$.
Тогда его одношаговый выигрыш выражается следующим образом:
\begin{equation}
  \label{ch2:lower-bound:eq:K1(q,pi)}
  \K[1](\p, \sigma, j) = \sum_{i \in I} \sum_{s \in S} q^i p^{s|i} \as(i, j).
\end{equation}

Обозначим $\Low[n](\p, \sigmav)$ гарантированный выигрыш первого игрока, использующего стратегию $\sigmav$ в игре $\theG[n](\p)$, т.е.
\[
  \Low[n](\p, \sigmav) = \inf_{\tauv \in \Tau} \K[n](\p, \sigmav, \tauv).
\]

\begin{lemma}
  \label{ch2:lower-bound:lemma:convex-combination}
  Пусть $\p_1, \p_2 \in \MM$, $\sigmav_1, \sigmav_2 \in \Sigma$ --- стратегии первого игрока.
  Тогда для $\p = \lambda \p_1 + (1-\lambda) \p_2,\ \lambda \in [0, 1],$ найдется такая стратегия $\sigmav_c \in \Sigma$, что
  \[
    \Low[n](\p, \sigmav_c) \geqslant
    \lambda \Low[n](\p_1, \sigmav_1) + (1-\lambda) \Low[n](\p_2, \sigmav_2).
  \]
\end{lemma}
\begin{proof}
  Без потери общности будем считать, что второй игрок использует стратегии $\tauv$, в которых распределения на множестве ставок имеют конечные математические ожидания.
  В противном случае ожидаемый выигрыш первого игрока будет равен бесконечности.
  Множество таких стратегий обозначим $\Tau'$.
  
  Докажем следующее утверждение: найдется такая стратегия $\sigmav_c = (\sigma_c, \sigmav_c^i) \in \Sigma$, что при всех $\tauv \in \Tau'$ справедливо равенство
  \begin{equation}
    \label{ch2:lower-bound:lemma:convex-combination:eq:1}
    \K[n](\p, \sigmav_c, \tauv) =
    \lambda \K[n](\p_1, \sigmav_1, \tauv) +
    (1-\lambda)\K[n](\p_2, \sigmav_2, \tauv).
  \end{equation}
  Доказательство проведем по индукции.
  Пусть $\q_h = (q_h^i,\ i \in I)$ и $\p_h^i = (p_h^{s|i},\ s \in S)$ --- векторы полных и апостериорных вероятностей, соответствующие первому ходу $\sigma_h$ стратегии $\sigmav_h,\ h = 1, 2$.
  Определим первый ход $\sigma_c$ стратегии $\sigmav_c$ параметрами
  \begin{equation*}
    \begin{gathered}
      q^i = \lambda q^i_1 + (1-\lambda) q^i_2,\ i \in I,                               \\
      p^{s|i} = \left(
        \lambda q^i_1 p^{s|i}_1 + (1-\lambda) q^i_2 p^{s|i}_2
      \right) / q^i,\ i \in I,\ s \in S.
    \end{gathered}
  \end{equation*}
  Подставив эти выражения в \eqref{ch2:lower-bound:eq:K1(q,pi)}, для любого $j \in J$ имеем:
  \begin{multline*}
    \K[1](\p, \sigma_c, j) =
    \sum_{i \in I}\sum_{s \in S} q^i p^{s|i} \as(i, j) =                               \\
    = \sum_{i \in I, \, s \in S}
    q^i \frac{(\lambda q^i_1 p^{s|i}_1 + (1-\lambda) q^i_2 p^{s|i}_2)}{q^i} \as(i,j) = \\
    = \lambda \K[1](\p_1, \sigma_1, j) +
    (1-\lambda)\K[1](\p_2, \sigma_2, j).
  \end{multline*}
  Осредняя это равенство по произвольному $\tau \in \Delta(J)$, получим \eqref{ch2:lower-bound:lemma:convex-combination:eq:1} при $n = 1$.
  Предположим, что утверждение имеет место при любых $n \leqslant N$.
  Поскольку для каждого $i \in I$
  \[
    \p^i = \frac{\lambda q_1^i}{q^i} \p_1^i + \frac{(1-\lambda)q_2^i}{q^i} \p_2^i,
  \]
  для $\sigmav_1^i,\ \sigmav_2^i$ найдется такая стратегия первого игрока $\sigmav_c^i$ в игре $\theG[N](\p^i)$, что для любой стратегии $\tauv^i$
  \begin{equation*}
    \begin{gathered}
      \K[N](\p^i, \sigmav^i_c, \tauv^i) =
      \frac{\lambda q^i_1}{q^i} \K[N](\p^i_1, \sigmav^i_1, \tauv^i) +
      \frac{(1-\lambda) q^i_2}{q^i} \K[N](\p^i_2, \sigmav^i_2, \tauv^i).
    \end{gathered}
  \end{equation*}
  В результате для $\sigmav_c = (\sigma_c, \sigmav_c^i) \in \Sigma$ и любой стратегии $\tauv = (\tau, \tauv^i) \in \Tau'$ в игре $\theG[N+1](\p)$ справедливо равенство
  \begin{align*}
    \K[N+1](\p, \sigmav_c, \tauv) &=
    \K[1](\p, \sigma_c, \tau) +
    \sum_{i \in I} q_i \K[N](\p^i, \sigmav^i_c, \tauv^i) =                             \\
    &= \lambda \K[1](\p_1, \sigma_1, \tau) +
    (1-\lambda) \K[1](\p_2, \sigma_2, \tau) +                                          \\
    &+ \sum_{i \in I} q^i \left(
      \frac{\lambda q^i_1}{q^i} \K[N](\p^i_1, \sigmav^i_1, \tauv^i) +
      \frac{(1-\lambda) q^i_2}{q^i} \K[N](\p^i_2, \sigmav^i_2, \tauv^i)
    \right) =                                                                          \\
    &= \lambda \K[N+1](\p_1, \sigmav_1, \tauv) +
    (1-\lambda)\K[N+1](\p_2, \sigmav_2, \tauv).
  \end{align*}
  Утверждение доказано.
  Из него получаем, что
  \begin{align*}
    \Low[n](\p, \sigmav_c)
    &= \inf_{\tauv \in \Tau'} \K[n](\p, \sigmav_c, j) \geqslant\\
    &\geqslant \lambda \min_{\tauv \in \Tau'} \K[n](\p, \sigmav_1, j) + (1-\lambda) \min_{\tauv \in \Tau'} \K[n](\p, \sigmav_2, j) =\\
    &= \lambda \Low[n](\p_1, \sigmav_1) + (1-\lambda) \Low[n](\p_2, \sigmav_2).
  \end{align*}
  Отсюда следует справедливость основного утверждения леммы.
\end{proof}

Обозначим $e^s$ вырожденное вероятностное распределение с носителем в точке $s$.
Пусть $\p^x(l, r) \in \Theta(x)$ --- распределение с носителем $\{l, r\},\ l<r$.
При этом распределении вероятности реализации состояний $l$ и $r$ равны $(r-x)/(r-l)$ и $(x-l)/(r-l)$ соответственно, а дисперсия
\[
  \D \p^x(l, r) = (x - l)(r - x).
\]
Как показано в \cite{domansky11}, любое распределение $\p = (p^s,\ s \in S) \in \Theta(x)$ может быть представлено в виде выпуклой комбинации распределений с двухточечными носителями следующим образом:
\begin{gather}
  \label{ch2:lower-bound:eq:prob-decomp-sum}
  \p = \begin{cases}
    \displaystyle
    p^x e^x + \sum_{r=x+1}^\infty \sum_{l=0}^{x-1} \alpha_{l,r}(\p) \p^x(l, r),\ & x \in S,\\
    \displaystyle
    \sum_{r=\floor{x+1}}^\infty \sum_{l=0}^{\ceil{x-1}} \alpha_{l,r}(\p) \p^x(l, r),\ & x \notin S,\\
  \end{cases}\\
  \alpha_{l,r}(\p) = (r-l) p^l p^r / \sum_{t=0}^{\ceil{x-1}} p^t (x-t). \nonumber
\end{gather}

Обозначим через $\LL$ банахово пространство последовательностей $(l^s,\ s \in S)$ с нормой $\norm{l} = \sum_{s = 0}^\infty s^2 |l^s|$.
Множества $\MM$ и $\Theta(x)$ являются выпуклыми замкнутыми подмножествами пространства $\LL$.

\begin{lemma}
  \label{ch2:lower-bound:lemma:subseq-convergence}
  Пусть последовательность $\{l_n\} \subset \LL$ такая, что для любых $s \in S$ и $n \geqslant 1$ верно $l_n^s \geqslant 0,\ l_n^s \leqslant l_{n+1}^s$.
  Тогда если существует ее сходящаяся по норме подпоследовательность $\{l_{j_n}\}$, то и сама последовательность $\{l_n\}$ сходится по норме.
\end{lemma}
\begin{proof}
  Пусть подпоследовательность $\{l_{j_n}\}$ сходится.
  Тогда для любого $\varepsilon > 0$ существует $M$ такое, что для любых $f, g \geqslant M$ выполнено 
  $\norm{l_{j_f} - l_{j_g}} \leqslant \varepsilon$.
  Положим $N$ равное $j_M$.
  
  Для любых $m \geqslant n \geqslant N$ можно найти $q$ такое, что $j_q \geqslant m$.
  В силу покомпонентной монотонности последовательности $\{l_n\}$ выполнено неравенство
  \begin{gather*}
    \norm{l_m - l_n} =
    \sum_{s=0}^\infty s^2 |l_m^s - l_n^s| \leqslant
    \sum_{s=0}^\infty s^2 |l_{j_q}^s - l_{j_M}^s| =
    \norm{l_{j_q} - l_{j_M}} \leqslant \varepsilon.
  \end{gather*}
  Отсюда последовательность $\{l_n\}$ фундаментальна и в силу полноты пространства $\LL$ сходится.
\end{proof}

\begin{lemma}
  \label{ch2:lower-bound:lemma:decomp-convergence}
  Для любого распределения $\p \in \Theta(x)$ ряд в разложении (\ref{ch2:lower-bound:eq:prob-decomp-sum}) сходится к $\p$ по норме.
\end{lemma}
\begin{proof}
  Проведем доказательство для $x \in S$.
  Рассмотрим последовательность
  \begin{equation*}
    s_n = p^x e^x + \sum_{r=x+1}^n \sum_{l=0}^{x-1} \alpha_{l,r}(\p) \p^x(l, r).
  \end{equation*}

  Тогда для $m \geqslant n$ справедливо
  \begin{align*}
    \norm{s_m-s_n}
    &= \norm{
      \sum_{r=n+1}^m \left(
      p^r (r-x) \frac{\sum_{l=0}^{x-1} p^l e^l}{\sum_{t=0}^{x-1} p^t (x-t)} + p^r e^r
      \right)} =\\
    &= \sum_{r=n+1}^m p^r \left(
      r^2 + (r-x) \frac{\sum_{t=0}^{x-1} p^t t^2}{\sum_{t=0}^{x-1} p^t (x-t)}
      \right).
  \end{align*}
  Так как $\D \p < \infty$, последовательность $s_n$ --- фундаментальна.
  Ее сходимость к $\p$ следует из покомпонентного равенства векторов вероятностных распределений:
  \begin{gather*}
    \sum_{r=x+1}^\infty \alpha_{l,r}(\p) \frac{r-x}{r-l} = p^l,\ l = \overline{0, x-1}, \\
    \sum_{l=0}^{x-1} \alpha_{l,r}(\p) \frac{x-l}{r-l} = p^r,\ r = \overline{x+1, \infty}.
  \end{gather*}
  Отсюда в силу леммы~\ref{ch2:lower-bound:lemma:subseq-convergence} получаем, что ряд в разложении~\eqref{ch2:lower-bound:eq:prob-decomp-sum} сходится к $\p$ по норме.

  Доказательство для нецелых значений $x$ проводится аналогично.
\end{proof}

В силу того, что функционал $\LowV(\p)$ вогнут на $\MM$ и по теореме~\ref{ch2:upper-bound:theorem} ограничен на данном множестве, то он непрерывен на $\MM$ (см.~\cite[Теорема 1.7.1]{polovinkin04}).
Отсюда и из леммы~\ref{ch2:lower-bound:lemma:decomp-convergence} следует, что для распределений $\p \in \Theta(x)$ выполнено
\begin{equation*}
  \left\{
  \begin{aligned}
    \LowV(\p) &\geqslant
      p^x \LowV(e^x) + \sum_{r=x+1}^\infty \sum_{l=0}^{x-1} \alpha_{l,r}(\p) \LowV \left( \p^x(l, r) \right),&\ x \in S,\\
    \LowV(\p) &\geqslant 
      \sum_{r=\floor{x+1}}^\infty \sum_{l=0}^{\ceil{x-1}} \alpha_{l,r}(\p) \LowV \left( \p^x(l, r) \right),&\ x \notin S.
  \end{aligned}
  \right.
\end{equation*}

Из данных неравенств, теоремы~\ref{ch2:upper-bound:theorem} и леммы~\ref{ch2:lower-bound:lemma:convex-combination} следует, что для доказательства совпадения верхней и нижней оценок выигрыша в игре $\theG[\infty](\p)$ можно ограничиться рассмотрением только распределений %
$\p = \p^{k+\beta}(l, r) \in \Theta(k + \beta), \; k \in S,\ l = \overline{0, k},\ r = \overline{k+1, \infty}$.
Для таких распределений мы построим стратегию первого игрока $\sigmav^*$, для которой $\Low[\infty](\p, \sigmav^*) = \High[\infty](\p)$.
Отсюда будет следовать, что $\V[\infty](\p)~=~\High[\infty](\p)$, а $\sigmav^*$ и $\tauv^*$ --- оптимальные стратегии игроков в игре $\theG[\infty](\p)$.

Обозначим $\sigmak_k$ ход первого игрока, состоящий в выборе ставки из множества $\{k, k+1\}$.
Ход $\sigmak_k$ определяется заданием полных вероятностей $q^k, q^{k+1}$ и апостериорных распределений $\p^k, \p^{k+1}$, причем $q^k + q^{k+1} = 1$.
Следующая лемма является обобщением утверждения~\ref{ch1:prop:stage-payoff} из главы~\ref{chapt1}.
\begin{lemma}
  \label{ch2:lower-bound:lemma:stage-payoff}
  При использовании $\sigmak_k$ одношаговый выигрыш первого игрока равен
  \begin{equation*}
    \K[1](\p, \sigmak_k, j) = \begin{cases}
      \E \p - \beta k - (1-\beta) j - \beta q^{k+1}, &\; j < k, \\
      (\E \p^{k+1} - k - \beta) q^{k+1}, &\; j = k, \\
      (k + \beta - \E \p^k) q^k, &\; j = k+1, \\
      (1-\beta) k + \beta j - \E \p + (1-\beta) q^{k+1}, &\; j > k + 1.
    \end{cases}
  \end{equation*}
\end{lemma}
\begin{proof}
  Можно показать, что
  \begin{equation*}
    \as(\sigmak_k, j) = \begin{cases}
      s - \beta k - (1-\beta) j - \beta \sigma^s_{k+1},     & \; j < k,   \\
      (s - k - \beta) \sigma^s_{k+1},                       & \; j = k,   \\
      (k + \beta - s) \sigma^s_k,                           & \; j = k+1, \\
      (1-\beta) k + \beta j - s + (1-\beta) \sigma^s_{k+1}, & \; j > k + 1.
    \end{cases}
  \end{equation*}
  Осредняя это равенство по произвольному $\tau \in \Delta(J)$, устанавливаем справедливость утверждения леммы.
\end{proof}

Определим стратегию $\sigmav^*$ первого игрока в игре $\theG[\infty](\p)$.
Введем множество распределений
\begin{equation*}
  P(l,r) = \left\{
    \p^k(l, r), \, \p^{s+\beta}(l, r), \, k = \overline{l,r}, s = \overline{l,r-1}
  \right\}.
\end{equation*}
При $\p \in P(l,r)$ первый ход $\sigma^*$ стратегии $\sigmav^*$ определяется следующим образом.
Если $\p = \p^l(l,r)$ или $\p = \p^r(l,r)$, то первый игрок использует ставки $l$ и $r$, соответственно, с вероятностью 1.
В противном случае он использует $\sigmak_k$ с параметрами из таблицы~\ref{ch2:tab:insider-strategy}.

\begin{table}[htb]
  \centering
  \renewcommand{\arraystretch}{1.5}
  \captionsetup{width=17cm}
  \caption{Параметры хода $\sigma^*$ при $\p \in P(l, r)$}
  \label{ch2:tab:insider-strategy}
  \begin{tabular}{|P{3cm}||P{3cm}|P{3cm}|P{3cm}|P{3cm}|}
    \hline
    \hline
    $\p$                 & $q^k$     & $q^{k+1}$ & $\p^{k}$               & $\p^{k+1}$           \\ \hline
    $\p^k(l, r)$         & $\beta$   & $1-\beta$ & $\p^{k-1+\beta}(l, r)$ & $\p^{k+\beta}(l, r)$ \\ \hline
    $\p^{k+\beta}(l, r)$ & $1-\beta$ & $\beta$   & $\p^k(l, r)$           & $\p^{k+1}(l, r)$     \\
    \hline
    \hline
    \multicolumn{1}{c}{}
    \vspace{-2.5em}
  \end{tabular}
\end{table}

На последующих шагах игры ход $\sigma^*$ применяется рекурсивно для соответствующих значений апостериорных вероятностей.
В результате определили стратегию $\sigmav^*$ для распределения $\p \in P(l,r)$.

Обозначим $L^x_n(\sigmav) = \Low[n](\p^x(l,r), \sigmav),\ \sigmav \in \Sigma$.
Следующая теорема является обобщением утверждения~\ref{ch1:prop:lower:recurrence-solution}.
\begin{theorem}
  \label{ch2:lower-bound:theorem}
  Пусть $\beta \in (0, 1)$.
  При использовании стратегии $\sigmav^*$ в игре $\theG[\infty](\p)$ для
  распределения %
  $\p \in P(l,r)$ %
  гарантированный выигрыш первого игрока удовлетворяет следующей системе:
  \begin{subequations}
    \label{ch2:lower-bound:eq:Linf-recurrence}
    \begin{equation}
      L^k_\infty(\sigmav^*) =
      \beta L^{k-1+\beta}_\infty(\sigmav^*) + (1-\beta) L^{k+\beta}_\infty(\sigmav^*),\ k = \overline{l + 1, r - 1},
    \end{equation}
    \begin{equation}
      L^{k+\beta}_\infty(\sigmav^*) =
      \beta(1-\beta) + (1-\beta) L^k_\infty(\sigmav^*) + \beta L^{k+1}_\infty(\sigmav^*),\ k = \overline{l, r - 1}.
    \end{equation}
    \begin{equation}
      L^l_\infty(\sigmav^*) = L^r_\infty(\sigmav^*) = 0.
    \end{equation}
  \end{subequations}
  Ее решение дает нижнюю оценку выигрыша первого игрока, равную
  \begin{equation*}
    \label{ch2:lower-bound:eq:Linf-recurrence-solution}
    \Low(\p^{k+\beta}(l, r), \sigmav^*) = \frac{(r-k-\beta)(k+\beta-l) + \beta(1-\beta)}{2}.
  \end{equation*}
\end{theorem}
\begin{proof}
  Для $\p \in P(l,r)$ стратегия $\sigmav^*$ определяется аналогично стратегии $\sigma^*$ из главы~\ref{chapt1} с заменой $0$ и $m$ на $l$ и $r$ соответственно.
  Из леммы~\ref{ch2:lower-bound:lemma:stage-payoff} нетрудно вывести, что
  \begin{equation*}
    L^k_1(\sigmav^*) = 0,\quad
    L^{k+\beta}_1(\sigmav^*) = \beta(1-\beta).
  \end{equation*}
  Для $k = \overline{l+1,r-1}$ имеем
  \begin{align*}
    L^k_\infty(\sigmav^*)
    &= L^k_1(\sigmav^*) + q^k L^{k-1+\beta}_\infty(\sigmav^*) + q^{k+1} L^{k+\beta}_\infty(\sigmav^*) = \\
    &= \beta L^{k-1+\beta}_\infty(\sigmav^*) + (1-\beta) L^{k+\beta}_\infty(\sigmav^*).
  \end{align*}
  Аналогично для $k = \overline{l, r-1}$ получаем
  \begin{align*}
    L^{k+\beta}_\infty(\sigmav^*)
    &= L^{k+\beta}_1(\sigmav^*) + q^k L^k_\infty(\sigmav^*) + q^{k+1} L^{k+1}_\infty(\sigmav^*) = \\
    &= \beta(1-\beta) + (1-\beta) L^k_\infty(\sigmav^*) + \beta L^{k+1}_\infty(\sigmav^*).
  \end{align*}
  Полученная система \eqref{ch2:lower-bound:eq:Linf-recurrence} является системой с трехдиагональной матрицей и решается методом прогонки.
\end{proof}

Поскольку
$
  \D \p^{k+\beta}(l, r) = (r-k-\beta)(k+\beta-l),
$
а распределение $\p^{k+\beta}(l, r)$ удовлетворяет условию теоремы~\ref{ch2:upper-bound:theorem} с $\eta = 1$, выражения для $\High[\infty](\p^{k+\beta}(l, r))$ и $\Low[\infty](\p^{k+\beta}(l, r), \sigmav^*),\ k = \overline{l,r-1}$ совпадают.
Из теоремы~\ref{ch2:upper-bound:theorem} также следует, что
\begin{equation*}
  \High[\infty](\p^l(l,r)) = \High[\infty](\p^r(l,r)) = 0.
\end{equation*}
В самом деле, распределения $\p^l(l,r)$ и $\p^r(l,r)$ удовлетворяют условию теоремы~\ref{ch2:upper-bound:theorem} с $\eta = 1-\beta$ и имеют нулевую дисперсию.

Для произвольного распределения $\p \in \Theta(k),\ k \in S$, стратегию $\sigmav^*$ определим следующим образом.
Если реализуется состояние $s = k$, то гарантированный выигрыш первого игрока не превышает $0$ и он прекращает игру.
Таким образом, первый игрок, следуя стратегии $\sigmav^*$, прекращает игру с вероятностью $p^k$.
В противном случае игрок использует конструкцию леммы~\ref{ch2:lower-bound:lemma:convex-combination} для построения стратегии, соответствующей выпуклой комбинации распределений $\p^k(l, r)$ в разложении $\p$.
Первый ход такой стратегии использует две ставки $k$ и $k+1$ с полными вероятностями $(1-p^k)\beta$ и $(1-p^k)(1-\beta)$ соответственно.
Апостериорные вероятностные распределения являются выпуклыми комбинациями соответствующих апостериорных двухточечных распределений и даются следующими формулами:
\begin{gather*}
  \p^k = \frac{1}{1-p^k} \sum_{r=k+1}^\infty \sum_{l=0}^{k-1} \alpha_{l,r}(\p) \p^{k-1+\beta}(l, r), \\
  \p^{k+1} = \frac{1}{1-p^k} \sum_{r=k+1}^\infty \sum_{l=0}^{k-1} \alpha_{l,r}(\p) \p^{k+\beta}(l, r).
\end{gather*}
Для распределений $\p$ со счетным носителем сходимость по норме данных рядов устанавливается аналогично доказательству леммы~\ref{ch2:lower-bound:lemma:decomp-convergence}.
Аналогичные рассуждения справедливы и для распределений $\p \in \Theta(k+\beta) \cup \Lambda(k, k+\beta),\ k \in S$.

\section{Решение игры $\mathbf{G^\beta_\infty(\p)}$}
\label{ch2:sec:game-solution}

Подчеркнем, что приведенная в п.~\ref{ch2:sec:lower-bound} стратегия инсайдера $\sigmav^*$ определена только при $\beta \in (0, 1)$.

Нетрудно проверить справедливость следующего равенства:
\begin{equation*}
  a^{r, \beta}(i, j) = a^{l, 1-\beta}(r+l-i, r+l-j).
\end{equation*}
Из него вытекает, что решение игры $\theG[\infty](\p^x(l, r))$ сводится к решению игры $G^{1-\beta}_\infty(\p^{l+r-x}(l,r))$.
При этом ставки, используемые в соответствующих смешанных стратегиях инсайдера, симметричны относительно точки $(l+r)/2$.
Аналогичные рассуждения справедливы для игры $\theG[\infty](\p)$ при любом распределении $\p \in \MM$.

Оптимальная стратегия $\sigmav^*$ инсайдера в игре $\theG[\infty](\p)$ при $\beta~=~1$ найдена в~\cite{domansky11}.
Решение $\theG[\infty](\p)$ при $\beta = 0$ может быть получено при помощи описанной выше конструкции из решения $\theG[\infty](\p)$ при $\beta = 1$.
Таким образом, при любом $\beta \in [0, 1]$ справедлива следующая
\begin{theorem}
  \label{ch2:solution:theorem}
  Игра $\theG[\infty](\p)$ имеет значение
  \[
    \V[\infty](\p) = \High(\p) = \Low(\p),
  \]
  а $\sigmav^*$ и $\tauv^*$ --- оптимальные стратегии игроков.
\end{theorem}

\begin{figure}[tbh]
  \centering
  \begin{tikzpicture}
    [
    auto,node distance=2.0cm,
    trans/.style={->,shorten >=1pt,>=stealth',semithick},
    state/.style={shape=circle,draw,minimum size=2mm}
    ]
    \node[state,label={$\p^l(l,r)$}] (p0) {};
    \node[state,right=of p0,label={$\p^{l+\beta}(l,r)$}] (p1) {}; 
    \node[state,right=of p1,label={$\p^{l+1}(l,r)$}] (p2) {};
    \node[right=of p2] (others) {$\ldots$};
    \node[state,right=of others,label={$\p^{r-1+\beta}(l,r)$}] (p2mm1) {};
    \node[state,right=of p2mm1,label={$\p^r(l,r)$}] (p2m) {};
    
    \path [trans]
    (p0) edge [loop left,min distance=10mm,out=205,in=155] node {$1$} (p0)
    (p1) edge[bend right] node[below] {$1-\beta$} (p0)
    (p1) edge[bend left] node[below] {$\beta$} (p2)
    (p2) edge[bend left] node[below] {$\beta$} (p1)
    (p2) edge[bend right] node[below] {$1-\beta$} (others)
    (others) edge[bend left] node[below] {$\beta$} (p2mm1)
    (p2mm1) edge[bend left] node[below] {$1-\beta$} (others)
    (p2mm1) edge[bend right] node[below] {$\beta$} (p2m)
    (p2m) edge[loop right,min distance=10mm,out=25,in=-25] node {$1$} (p2m)
    ;
  \end{tikzpicture}
  \caption[Последовательность апостериорных вероятностей]{Случайное блуждание последовательности апостериорных вероятностей, порожденное $\sigmav^*$}
  \label{ch2:fig:posterior-1}
\end{figure}

Применение первым игроком стратегии $\sigmav^*$ при $\p \in P(l, r)$ порождает случайное блуждание последовательности апостериорных вероятностей, изображенное на рисунке~\ref{ch2:fig:posterior-1}, которое в отличие от \cite{domansky11} происходит по более широкому множеству и уже не является симметричным, за исключением случая $\beta = 1/2$.

\section{Вторая оптимальная стратегия инсайдера в игре~$\mathbf{G^\beta_\infty(\p)}$}
\label{ch2:sec:optimal-strategy2}
В дополнение к стратегии $\sigmav^*$ построим еще одну оптимальную стратегию $\xiv^*$ инсайдера.
Введем множество распределений %
\begin{equation*}
  P'(l,r) =
  \{\p^l(l, r), \p^r(l, r)\}
  \cup
  \left\{
    \p^{k+\beta}(l, r), \, k = \overline{l, r-1}
  \right\}.
\end{equation*}
При $\p \in P'(l,r)$ первый ход $\xi^*$ стратегии $\xiv^*$ определяется следующим образом.
Если $\p = \p^l(l,r)$ или $\p = \p^r(l,r)$, то первый игрок использует ставки $l$ и $r$ соответственно с вероятностью 1.
В противном случае он использует $\sigmak_k$ с параметрами из таблицы~\ref{ch2:tab:insider-strategy2}.
\begin{table}[htb]
  \centering
  \renewcommand{\arraystretch}{1.5}
  \captionsetup{width=17cm}
  \caption{Параметры хода $\xi^*$ при $\p \in P'(l, r)$}
  \label{ch2:tab:insider-strategy2}
  \begin{tabular}{|P{3cm}||P{3cm}|P{3cm}|P{3cm}|P{3cm}|}
    \hline
    \hline
    $\p$                   & $q^k$ & $q^{k+1}$                 & $\p^{k}$                & $\p^{k+1}$                                      \\
    \hline
    $\p^{l+\beta}(l, r)$           & $\frac{1}{1+\beta}$       & $\frac{\beta}{1+\beta}$ & $\p^l(l, r)$           & $\p^{l+1+\beta}(l, r)$ \\
    \hline
    $\p^{r-1+\beta}(l, r)$         & $\frac{1-\beta}{2-\beta}$ & $\frac{1}{2-\beta}$     & $\p^{r-2+\beta}(l, r)$ & $\p^r(l, r)$           \\
    \hline
    $\p^{k+\beta}(l, r)$   & $\frac{1}{2}$             & $\frac{1}{2}$           & $\p^{k-1+\beta}(l, r)$ & $\p^{k+1+\beta}(l, r)$  \\
    \hline
    \hline
    \multicolumn{1}{c}{}
    \vspace{-2.5em}
  \end{tabular}
\end{table}

Для остальных распределений $\p$ стратегия $\xiv^*$ определяется аналогично тому, как это было сделано для стратегии $\sigmav^*$.

Использование стратегии $\xiv^*$ при $\p \in P'(l, r)$ порождает случайное блуждание последовательности апостериорных вероятностей, изображенное на рисунке~\ref{ch2:fig:posterior-2}.
Данное блуждание симметрично с вероятностями перехода в соседние состояния равными $1/2$, симметрия нарушается только в крайних и соседних к ним состояниях.

\begin{figure}[tbh]
  \centering
  \begin{tikzpicture}
    [
    auto,yscale=1.1,node distance=2cm,
    trans/.style={->,shorten >=1pt,>=stealth',semithick},
    state/.style={shape=circle,draw,minimum size=2mm}
    ]
    \node[state,label={$\p^l(l,r)$}] (p0) {};
    \node[state,right=of p0,label={$\p^{l+\beta}(l,r)$}] (p1) {}; 
    \node[state,right=of p1,label={$\p^{l+1+\beta}(l,r)$}] (p2) {};
    \node[right=of p2] (others) {$\ldots$};
    \node[state,right=of others,label={$\p^{r-1+\beta}(l,r)$}] (p2mm1) {};
    \node[state,right=of p2mm1,label={$\p^r(l,r)$}] (p2m) {};
    
    \path [trans]
    (p0) edge [loop left,min distance=10mm,out=205,in=155] node {$1$} (p0)
    (p1) edge[bend right] node[below] {$\frac{1}{1+\beta}$} (p0)
    (p1) edge[bend left] node[below] {$\frac{\beta}{1+\beta}$} (p2)
    (p2) edge[bend left] node[below] {$\frac{1}{2}$} (p1)
    (p2) edge[bend right] node[below] {$\frac{1}{2}$} (others)
    (others) edge[bend left] node[below] {$\frac{1}{2}$} (p2mm1)
    (p2mm1) edge[bend left] node[below] {$\frac{1-\beta}{2-\beta}$} (others)
    (p2mm1) edge[bend right] node[below] {$\frac{1}{2-\beta}$} (p2m)
    (p2m) edge[loop right,min distance=10mm,out=25,in=-25] node {$1$} (p2m)
    ;
  \end{tikzpicture}
  \caption[Последовательность апостериорных вероятностей]{Случайное блуждание последовательности апостериорных вероятностей, порожденное $\xiv^*$}
  \label{ch2:fig:posterior-2}
\end{figure}

Из леммы~\ref{ch2:lower-bound:lemma:stage-payoff} можно вывести, что
\begin{gather*}
  L^{l+\beta}_1(\xiv^*) = \frac{\beta}{1+\beta},\quad
  L^{r-1+\beta}_1(\xiv^*) = \frac{1-\beta}{2-\beta},           \\
  L^{k+\beta}_1(\xiv^*) = \frac{1}{2},\ k = \overline{l+1,r-2}.
\end{gather*}
Отсюда следует, что при использовании стратегии $\xiv^*$ в игре $\theG[\infty](\p)$ для распределения %
$\p \in P'(l,r)$ %
гарантированный выигрыш первого игрока удовлетворяет следующей системе:
\begin{subequations}
  \label{ch2:lower-bound:eq:Linf-recurrence-2}
  \begin{equation}
    L^{l+\beta}_{\infty}(\xiv^*) =
    \frac{\beta}{1+\beta} + \frac{1}{1+\beta} L^l_{\infty}(\xiv^*) + \frac{\beta}{1+\beta} L^{l+1+\beta}_{\infty}(\xiv^*),
  \end{equation}
  \begin{equation}
    L^{r-1+\beta}_{\infty}(\xiv^*) =
    \frac{1-\beta}{2-\beta} + \frac{1-\beta}{2-\beta} L^{r-2+\beta}_{\infty}(\xiv^*) + \frac{1}{2-\beta} L^r_{\infty}(\xiv^*),
  \end{equation}
  \begin{equation}
    L^{k+\beta}_{\infty}(\xiv^*) =
    \frac{1}{2} + \frac{1}{2} L^{k-1+\beta}_{\infty}(\xiv^*) + \frac{1}{2} L^{k+1+\beta}_{\infty}(\xiv^*),\ k = \overline{l+1, r-2},
  \end{equation}
  \begin{equation}
    L^l_{\infty}(\xiv^*) = L^r_{\infty}(\xiv^*) = 0.
  \end{equation}
\end{subequations}

Нетрудно проверить, что подстановкой
\begin{gather*}
  \High(\p^{k+\beta}(l,r)) = \frac{(r-k-\beta) (k+\beta-l) + \beta(1-\beta)}{2}, \\
  \High(\p^l(l,r)) = \High(\p^r(l,r)) = 0,
\end{gather*}
вместо $L^{k+\beta}_{\infty}(\xiv^*),\ k = \overline{l,r-1}$, $L^l_\infty(\xiv^*)$ и $L^r_\infty(\xiv^*)$ соответственно данные равенства обращаются в тождества.
Отсюда, как и для стратегии $\sigmav^*$, следует, что стратегия $\xiv^*$ является оптимальной.

Отметим, что в отличие от стратегии $\sigmav^*$ стратегия $\xiv^*$ определена при $\beta \in [0, 1]$ и совпадает с оптимальной стратегией инсайдера из \cite{domansky11} при $\beta = 1$.
При этом обе стратегии $\sigmav^*$ и $\xiv^*$ порождают существенно различные случайные блуждания апостериорных вероятностей.

\clearpage
}

%%% Local Variables:
%%% mode: latex
%%% TeX-master: "../dissertation"
%%% End:
