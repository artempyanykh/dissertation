\chapter*{Обзор литературы}
\addcontentsline{toc}{chapter}{Обзор литературы}	% Добавляем его в оглавление

Теория повторяющихся игр берет начало во второй половине 60-х годов в серии отчетов~\cite{r:aumann66, r:aumann67, r:aumann68a, r:aumann68b, r:stearns67} Агенства по Контролю за Вооружением и Разоружением (ACDA) США.
Позднее данные результаты были опубликованы Ауманом и Машлером в монографии~\cite{aumann95}.
Основываясь на результатах Харсаньи~\cite{harsanyi67} о формализации игр с неполной информацией, авторы моделируют информационную асимметрию агентов путем введения множества возможных состояний природы и хода случая, который в соответствии с некоторым вероятностным распределением перед началом игры определяет текущее состояние и связанную с этим состоянием \emph{одношаговую} игру, которая впоследствии будет разыгрываться на протяжении $n$ шагов.
В простейшем случае антагонистической повторяющейся игры двух лиц с неполной информацией у одной из сторон первый игрок "--- которого мы также будем называть \emph{инсайдером} "--- осведомлен о результате хода случая, в то время как второй игрок знает только вероятностное распределение.

Ауман и Машлер рассматривают матричные игры c функцией выигрыша равной средней ожидаемой выплате за $n$ шагов.
Для игр неограниченной продолжительности с неполной информацией у одной из сторон авторами получены исчерпывающие результаты, в частности показано, что значения таких игр всегда существуют, и приведен способ построения оптимальных стратегий игроков.
Для игр с неполной информацией у обоих игроков авторами также получен критерий существования значения.
Ключевое отличие игр с неполной информацией у одного игрока от игр с неполной информацией у обоих игроков заключается в том, что последние уже не всегда имеют значение.
В работе~\cite{mertens71} Мертенсом и Замиром был рассмотрен более широкий класс повторяющихся игр с неполной информацией у обоих игроков.
Для таких игр авторами были найдены функциональные уравнения, которым должно удовлетворять значение игры, а также получена оценка отклонения значения $n$-шаговой игры от значения игры неограниченной продолжительности.

Одной из областей приложения данной теории является анализ поведения агентов на финансовых рынках.
Применение винеровских процессов и случайных блужданий для описания эволюции цен на финансовые инструменты широко распространено как в
финансовой литературе так и на практике.
Появление случайных колебаний цен на активы принято объяснять наличием множества слабых внешних факторов.
Впервые гипотеза об их эндогенном происхождении была выдвинута Кайлом в работе~\cite{kyle85}.
Ключевой момент в модели Кайла состоит в наличии на рынке фоновых игроков, которые принимают решение о покупке или продаже актива случайным образом и тем самым маскируют действия инсайдера.
Фактически броуновское движение было введено в модель извне.

Более явно идея прослеживается в работе Де~Мейера и Салей~\cite{demeyer02}, в которой они рассматривают упрощенную модель финансового рынка, на котором два
игрока ведут торговлю однотипными акциями на протяжении $n \leq \infty$ шагов.
Рынок может находится в состояниях $H$ и $L$ с вероятностями $p$ и $1-p$ соответственно.
В состоянии $H$ цена акции равна $1$, в состоянии $L$ она равна $0$.
Первый игрок знает текущее состояние рынка, второй игрок знает только вероятностное распределение и то, что первый "--- инсайдер.
На каждом шаге торгов игроки одновременно и независимо назначают некоторую цену за акцию.
Игрок, сделавший б\'{о}льшую ставку, покупает у другого акцию по названной цене; если ставки равны, то сделка не состоится.
Задачей игроков является максимизация стоимости итогового портфеля, состоящего из некоторого числа купленных акций и суммы денег, полученных в результате торгов.

Авторы формализуют данное описание в виде антагонистической повторяющейся игры с неполной информацией у одного из игроков.
Ключевые отличия полученной игры от игр, рассмотренных в~\cite{aumann95}, заключаются в следующем: множество возможных действий игроков на каждом шаге игры имеет мощность континуум, и в качестве выигрыша принимаются не усредненные, а суммарные выплаты за $n$ шагов.
Это приводит к существенно отличной технике решения. Основной результат данной работы заключается в демонстрации наличия винеровской компоненты в динамике последовательности ожидаемых цен акции.

В работе~\cite{demeyer02c} авторы рассматривают модель торгов, в рамках которой цена акции распределена в соответствии с произвольным вероятностным распределением, имеющим конечное математическое ожидание.
Результатом анализа соответствующей повторяющейся игры с континуумом состояний снова является демонстрация наличия броуновского движения в асимптотике эволюции цен, предлагаемых инсайдером.

В работе~\cite{demeyer10} Де Мейером рассмотрена непрерывная модель с достаточно общим механизмом торгов.
Автор показывает, что асимптотически последовательность ожидаемых ликвидных цен акции сходится к непрерывному мартингалу максимальной
вариации (Continuous Martingale of Maximal Variation) и не зависит от конкретного вида механизма торгов, а зависит только от априорного распределенияцены акции.
Этот результат был обобщен для торгов несколькими активами в работе~\cite{gensbittel15}.

Модели биржевых торгов с дискретными ставками были впервые были рассмотрены в работах Марино, Де~Мейера~\cite{demeyer05} и Доманского~\cite{domansky07}.
Показано, что в отличие от непрерывной модели последовательность значений $n$-шаговых игр ограничена сверху, что можно объяснить меньшей стратегической свободой инсайдера.
Одним из результатов полученных для дискретного случая является появление симметричного случайного блуждания апостериорных вероятностей высокой цены актива.
Важным отличием работы~\cite{domansky07} от работы~\cite{demeyer05} является использование \emph{разумной} стратегии инсайдера, которая оказывается оптимальной в бесконечной игре, что существенно упрощает схему доказательств. 
Нужно отметить, что дискретная модель больше соответствует действительности, в силу того, что расчеты на реальных рынках проводятся кратно минимальной денежной единице.

В работе~\cite{domansky11} была рассмотрена дискретная модель биржевых торгов, в рамках которой цена акции может принимать любые целочисленные значения.
Авторами показано, что когда дисперсия цены акции конечна, игра неограниченной продолжительности имеет значение.
Оптимальная стратегия инсайдера построена на основе представления распределений на $\Z_+$ в виде выпуклой комбинации распределений, имеющих не более двух носителей.

Обобщение дискретной модели на случай торгов несколькими активами проведено в работах~\cite{domansky13, domansky14}.
Авторами показано, что значение игры неограниченной продолжительности равно сумме значений соответствующих игр с одним активом.
Для торгов $m$ активами оптимальные стратегии инсайдера построены на основе представления $m$-мерных распределений в виде выпуклой комбинации распределений, имеющих не более $m+1$ носитель.
Также авторами отмечено, что конечная игра с $m$ активами менее выгодна для инсайдера, чем $m$ независимых игр с одним активом, в силу того, что ставки для одного актива несут информацию о настоящей стоимости других активов.

В работе~\cite{sandomirskaya14} Сандомирской была рассмотрена дискретная модель торгов с фиксированным спрэдом.
В рамках данной модели игроки назначают цены покупки $p_b$, цены продажи определяются как $p_a = p_b + x$, где $x$ "--- величина спрэда; сделка происходит в том случае, если цена продажи одного из игроков меньше или равна цены покупки другого. 
Автором получена оценка сверху не выигрыш инсайдера в бесконечной игре, а также оценка снизу при использовании стратегии из класса стратегий, порождающих простые случайные блуждания апостериорных вероятностей. 
Показано, что в сравнении с моделью без спрэда, максимальный выигрыш инсайдера уменьшается в $x$ раз.

Упомянутые выше результаты для дискретных моделей касаются в основном бесконечных игр.
Для игр с конечным количеством шагов аналитические решения получены только в частных случаях: в работе~\cite{sandomirskaya12} найдено решение одношаговой игры, в работе~\cite{kreps09} получено решение $n$-шаговой игры с минимальным нетривиальным числом ставок, равным трем.
В диссертации Сандомирской~\cite{phd:sandomirskaya} определена величина, которую может гарантировать инсайдер в конечной игре, применяя стратегию, оптимальную при бесконечношаговом взаимодействии.
Показано, что значение конечной игры стремится к значению бесконечной игры как минимум с экспоненциальной скоростью.

%%% Local Variables:
%%% mode: latex
%%% TeX-master: "../dissertation"
%%% End:
