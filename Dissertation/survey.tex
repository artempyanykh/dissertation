\chapter*{Обзор литературы}
\addcontentsline{toc}{chapter}{Обзор литературы}	% Добавляем его в оглавление

Теория повторяющихся игр берет начало во второй половине 60-х годов в серии
отчетов \cite{r:aumann66, r:aumann67, r:aumann68a, r:aumann68b, r:stearns67}
Агенства по Контролю за Вооружением и Разоружением (ACDA) США. Позднее данные
результаты были опубликованы Ауманом и Машлером в монографии \cite{aumann95}.
Основываясь на результатах Харсаньи \cite{harsanyi67} о формализации игр с
неполной информацией, авторы моделируют информационную асимметрию агентов путем
введения множества возможных состояний природы и хода случая, который в
соответстви с некоторым вероятностным распределением перед началом игры
определяет текущее состояние и связанную с этим состоянием \emph{одношаговую}
игру, которая впоследствии будет разыгрываться на протяжении $n$ шагов. В
простейшем случае антагонистической повторяющейся игры двух лиц с неполной
информацией у одной из сторон первый игрок "--- которого мы также будем называть
\emph{инсайдером} "--- осведомлен о результате хода случая, в то время как
второй игрок знает только вероятностное распределение.

Ауман и Машлер рассматривают матричные игры c функцией выигрыша равной средней
ожидаемой выплате за $n$ шагов. Для игр неограниченной продолжительности с
неполной информацией у одной из сторон авторами получены исчерпывающие
результаты, в частности показано, что значения таких игр всегда существуют, и
приведен способ построения оптимальных стратегий игроков. Для игр с неполной
информацией у обоих игроков авторами также получен критерий существования
значения. Ключевое отличие игр с неполной информацией у одного игрока от игр с
неполной информацией у обоих игроков заключается в том, что последние уже не
всегда имеют значение.

\todo{Добавить Mertens, J.F. and S. Zamir, 1971. The value of Two-Person
  Zero-Sum Repeated Games with Lack of Information on Both Sides, International
  Journal of Game Theory, vol.1, p.39-64.}

Одной из областей приложения данной теории является анализ поведения агентов на
финансовых рынках. Применение винеровских процессов и случайных блужданий для
описания эволюции цен на финансовые инструменты широко распространено как в
финансовой литературе так и на практике. Появление случайных колебаний цен на
активы принято объяснять наличием множества слабых внешних факторов. Впервые
гипотеза об их эндогенном происхождении была выдвинута Кайлом в работе
\cite{kyle85}. Ключевой момент в модели Кайла состоит в наличии на рынке фоновых
игроков, которые принимают решение о покупке или продаже актива случайным
образом и тем самым маскируют действия инсайдера. Фактически броуновское
движение было введено в модель извне.

Более явно идея прослеживается в работе Де Мейера и Салей \cite{demeyer02}, в
которой они рассматривают упрощенную модель финансового рынка, на котором два
игрока ведут торговлю однотипными акциями на протяжении $n \leq \infty$ шагов.
Рынок может находится в состояниях $H$ и $L$ с вероятностями $p$ и $1-p$
соответственно. В состоянии $H$ цена акции равна $1$, в состоянии $L$ она равна
$0$. Первый игрок знает текущее состояние рынка, второй игрок знает только
вероятностное распределение и то, что первый "--- инсайдер. На каждом шаге
торгов игроки одновременно и независимо назначают некоторую цену за акцию.
Игрок, сделавший б\'{о}льшую ставку, покупает у другого акцию по названной цене;
если ставки равны, то сделка не состоится. Задачей игроков является максимизация
стоимости итогового портфеля, состоящего из некоторого числа купленных акций и
суммы денег, полученных в результате торгов.

Авторы формализуют данное описание в виде антагонистической повторяющейся игры с
неполной информацией у одного из игроков. Ключевые отличия полученной игры от
игр, рассмотренных в \cite{aumann95}, заключаются в следующем: множество
возможных действий игроков на каждом шаге игры имеет мощность континуум, и в
качестве выигрыша принимаются не усредненные, а суммарные выплаты за $n$ шагов.
Это приводит к существенно отличной технике решения. Основной результат данной
работы заключается в демонстрации наличия винеровской компоненты в динамике
последовательности ожидаемых цен акции.

\todo{Продолжить обзор непрерывных моделей.}

В работе Марино, Де Мейера \cite{demeyer05}, а также в работе Доманского
\cite{domansky07} была исследована модель биржевых торгов с дискретными
ставками. Показано, что в отличие от непрерывной модели последовательность
значений $n$-шаговых игр ограничена сверху, что можно объяснить меньшей
стратегической свободой инсайдера. Одним из результатов полученных для
дискретного случая является появление симметричного случайного блуждания
апостериорных вероятностей высокой цены актива.

Важным отличием работы \cite{domansky07} от работы \cite{demeyer05} является
использование \emph{разумной} стратегии инсайдера, которая оказывается
оптимальной в бесконечной игре. Это существенно упрощает схему доказательств и
позволяет распространить данную схему на более широкий класс моделей. Нужно
отметить, что дискретная модель больше соответствует действительности, в силу
того, что расчеты на реальных рынках проводятся кратно минимальной денежной
единице.

\todo{Продолжить обзор дискретных бесконечных моделей. Потом переключиться на
  конечные игры.}

% Повторяющиеся игры с неполной информацией представляют собой
% естественную модель для анализа информационного аспекта в продолжительном
% стратегическом взаимодействии агентов и позволяют ответить на вопросы о том, как
% быстро происходит раскрытие приватной информации, каковы эффективные механизмы
% ее сокрытия и какую выгоду из нее могут извлечь агенты. Начиная с работ Аумана,
% Машлера и Стернса (см. \cite{aumann95}), теория повторяющихся игр с неполной
% информацией получила дальнейшее развитие в работах Харта \todo{cite}, Сорина
% \todo{cite}, Замира \todo{cite} и др.

% Наиболее полное развитие получила теория повторяющихся антагонистических игр
% двух лиц с неполной информацией у одной из сторон. В таких играх информационная
% неопределенность моделируется введением множества возможных состояний природы
% $S$. Перед началом игры ходом случая выбирается конкретное состояние $s \in S$ в
% соответствии с некоторым вероятностным распределением, известным обоим игрокам.
% При этом игрок 1 осведомлен о выбранном значении $s$, в то время как игрок 2
% знает только то, что первый обладает приватной информацией. Анализ таких игр
% является довольно сложной задачей, и аналитические решения получены лишь в
% относительно небольшом числе случаев.

% Одной из областей приложения данной теории является анализ поведения агентов на
% финансовых рынках. Начиная с работы Башелье \cite{bachelier1900}, для описания
% эволюции цен на активы используются винеровские процессы или "--- в дискретном
% случае --- случайные блуждания. Возникновение случайных колебаний цен на рынке
% принято объяснять влиянием не процесс ценообразования множества слабых
% независимых внешних факторов. Однако гипотеза о полностью экзогенном
% происхождении случайных колебаний цен не является удовлетворительной. Гипотеза
% об эндогенном происхождении колебаний цен впервые была выдвинута в работе
% \cite{kyle85}, однако броуновское движение по сути было введено в модель извне.
% Более убедительно гипотеза о стратегическом происхождении случайных колебаний
% цен была продемонстрирована в работе Де Мейер, Салей \cite{demeyer02}, где
% винеровская компонента в эволюции цен возникает в следствие асимметричной
% информированности агентов. В модели Де Мейер, Салей случайная ликвидная цена
% акции может принимать два возможных значения $0$ и $1$ и определяется перед
% началом торгов. Игрок 1 осведомлен о настоящем значении цены, игрок 2 знает
% только ее вероятностное распределение. На каждом шаге торгов игроки одновременно
% назначают цены за одну единицу рискового актива. Игрок, предложивший б\'{о}льшую
% цену, покупает у второго актив по названной цене. В рамках данной модели игроки
% могут делать произвольные вещественные ставки. Модели с дискретными ставками
% впервые были рассмотрены в работах \cite{demeyer05, domansky07}. Ключевое
% отличие дискретных моделей от непрерывных заключается в ограниченности значений
% $n$-шаговых игр в силу меньшей стратегической свободы действий игрока 1.
% Показано, что в модели с дискретными ставками ожидаемый момент раскрытия
% приватной информации конечен, в то время как в модели с непрерывными ставками
% раскрытие информации до последнего шага торгов стремиться к $0$ при стремлении
% числа шагов к бесконечности.

%%% Local Variables:
%%% mode: latex
%%% TeX-master: "../dissertation"
%%% End:
