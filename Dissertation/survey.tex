\chapter*{Обзор литературы}
\addcontentsline{toc}{chapter}{Обзор литературы}	% Добавляем его в оглавление

Повторяющиеся игры с неполной информацией представляют собой
естественную модель для анализа информационного аспекта в продолжительном
стратегическом взаимодействии агентов и позволяют ответить на вопросы о том, как
быстро происходит раскрытие приватной информации, каковы эффективные механизмы
ее сокрытия и какую выгоду из нее могут извлечь агенты. Начиная с работ Аумана,
Машлера и Стернса (см. \cite{aumann95}), теория повторяющихся игр с неполной
информацией получила дальнейшее развитие в работах Харта \todo{cite}, Сорина
\todo{cite}, Замира \todo{cite} и др.

Наиболее полное развитие получила теория повторяющихся антагонистических игр
двух лиц с неполной информацией у одной из сторон. В таких играх информационная
неопределенность моделируется введением множества возможных состояний природы
$S$. Перед началом игры ходом случая выбирается конкретное состояние $s \in S$ в
соответствии с некоторым вероятностным распределением, известным обоим игрокам.
При этом игрок 1 осведомлен о выбранном значении $s$, в то время как игрок 2
знает только то, что первый обладает приватной информацией. Анализ таких игр
является довольно сложной задачей, и аналитические решения получены лишь в
относительно небольшом числе случаев.

Одной из областей приложения данной теории является анализ поведения агентов на
финансовых рынках. Начиная с работы Башелье \cite{bachelier1900}, для описания
эволюции цен на активы используются винеровские процессы или "--- в дискретном
случае --- случайные блуждания. Возникновение случайных колебаний цен на рынке
принято объяснять влиянием не процесс ценообразования множества слабых
независимых внешних факторов. Однако гипотеза о полностью экзогенном
происхождении случайных колебаний цен не является удовлетворительной. Гипотеза
об эндогенном происхождении колебаний цен впервые была выдвинута в работе
\cite{kyle85}, однако броуновское движение по сути было введено в модель извне.
Более убедительно гипотеза о стратегическом происхождении случайных колебаний
цен была продемонстрирована в работе Де Мейер, Салей \cite{demeyer02}, где
винеровская компонента в эволюции цен возникает в следствие асимметричной
информированности агентов. В модели Де Мейер, Салей случайная ликвидная цена
акции может принимать два возможных значения $0$ и $1$ и определяется перед
началом торгов. Игрок 1 осведомлен о настоящем значении цены, игрок 2 знает
только ее вероятностное распределение. На каждом шаге торгов игроки одновременно
назначают цены за одну единицу рискового актива. Игрок, предложивший б\'{о}льшую
цену, покупает у второго актив по названной цене. В рамках данной модели игроки
могут делать произвольные вещественные ставки. Модели с дискретными ставками
впервые были рассмотрены в работах \cite{demeyer05, domansky07}. Ключевое
отличие дискретных моделей от непрерывных заключается в ограниченности значений
$n$-шаговых игр в силу меньшей стратегической свободы действий игрока 1.
Показано, что в модели с дискретными ставками ожидаемый момент раскрытия
приватной информации конечен, в то время как в модели с непрерывными ставками
раскрытие информации до последнего шага торгов стремиться к $0$ при стремлении
числа шагов к бесконечности.

%%% Local Variables:
%%% mode: latex
%%% TeX-master: "../dissertation"
%%% End:
