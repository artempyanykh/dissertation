\chapter*{Введение}							% Заголовок
\addcontentsline{toc}{chapter}{Введение}	% Добавляем его в оглавление

\newcommand{\actuality}{\textbf\actualityTXT}
\newcommand{\progress}{}
\newcommand{\aim}{{\textbf\aimTXT}}
\newcommand{\tasks}{\textbf{\tasksTXT}}
\newcommand{\researchsubject}{\textbf{\researchsubjectTXT}}
\newcommand{\novelty}{\textbf{\noveltyTXT}}
\newcommand{\influence}{\textbf{\influenceTXT}}
\newcommand{\methods}{\textbf{\methodsTXT}}
\newcommand{\defpositions}{\textbf{\defpositionsTXT}}
\newcommand{\reliability}{\textbf{\reliabilityTXT}}
\newcommand{\probation}{\textbf{\probationTXT}}
\newcommand{\contribution}{\textbf{\contributionTXT}}
\newcommand{\publications}{\textbf{\publicationsTXT}}


{\actuality}
������������� ���� � �������� ����������� ������������ ����� ������������ ������ ��� ������� ��������������� ������� � ��������������� �������������� �������������� ������� � ��������� �������� �� ������� � ���, ��� ������ ���������� ��������� ��������� ����������, ������ ����������� ��������� �� �������� � ����� ������ �� ��� ����� ������� ������.

������ �������� ���� �������� � ������������ ������� ��������~\cite{harsanyi67}, ������, ������� � �������~\cite{aumann95}. 
�������� ����� ����������� ������������� ����������������� ���� ���� ��� � �������� ����������� � ����� �� ������. 
� ����� ����� �������������� ���������������� ������������ ��������� ��������� $S$ ��������� ��������� �������. 
����� ������� ���� ����� ������ ���������� ���������� ��������� $s \in S$ � ������������ � ��������� ������������� ��������������, ��������� ����� �������.
����� ���� ������ �� ���������� $n$ ����� ������ � ����, ��������������� ��������� $s$.
��� ���� ������ ����� ���������� � ��������� �������� $s$, � �� ����� ��� ������ ����� ������ ��, ��� ������ �������� ��������� �����������.

����� �� �������� ���������� ������ ������ �������� ������ ��������� ������� �� ���������� ������.
������� � ������ �������~\cite{bachelier1900}, ��� �������� �������� ��� �� ������ ������������ ����������� �������� ��� "--- � ���������� ������ --- ��������� ���������.
������������� ��������� ��������� ��� �� ����� ������� ��������� �������� �� ������� ��������������� ��������� ������ ����������� ������� ��������. 
������ �������� � ��������� ���������� ������������� ��������� ��������� ��� �� �������� ������������������. 
�������� �� �� �������������� ������������� ���� ������������������ � ������ ��~������ � �����~\cite{demeyer02}, ��� ����������� ���������� � �������� ��� ��������� �
��������� ������������� ����������������� �������. 
� ������ ������ ������ ��� ������ �� ���������� $n$ ����� ����� �������� ����������� ��������� ��������, ������ ���� �� ��� ����� ��������� ���� ������. 
�� ������ ���� ��� ������ ������������ ������, � �����, ������������ �\'{�}����� ������, �������� �
������� ����� �� ������������ ����; ��� ��������� ������ ������ �� ���������.
� ������ ������ � ��~������~\cite{demeyer05}, � ����� � ������ �.~�.~����������~\cite{domansky07} ���� ����������� ������ �������� ������ � ����������� ��������, ��� ���� ��������, ��� ������������������ ��� ������ �������� ������� ��������� ���������.

� ��������� ������� ������������� ���� � ��� �� �������� ������, � ������ "--- ������� �� ���������� ����.
��� ���� ��������, ��� ����� ����������� ��������� ����� ������������ ������� ������ �� �������������� ��������� ������� � �� �� �������� � ���������� ��������������.
��� ���� �������� �������� � ������~\cite{demeyer02}, ����������������, �������� ������������� ������������ �������� �������� ������ ��������� ����� �������� (����� �������������� ����������).
� ���������, ������ ������ � ����� ����� ������������ ���������� ������ ������� ��� ���� �� ���������� ����������� ������������.

� ������~\cite{demeyer10} ��~������� ����������� ����������� ������ � ���������� ����� ���������� ������.
�������� ��������� ������ ������ ����������� � ���, ��� � ����������� ������� �������� ��� �� �������� ����� �� ������� �� ����������� ���������, � ������� ������ �� ���������� ������������� ���� ������.
������, ����� �� �������, ������������� �� �������� ������, � ������ ������� ������� �������� ����������, �������� ������������� �������� ��������������� ��� �������� �����������������.

� ������ �.~C.~������������~\cite{sandomirskaya14} ����������� ���������� ������ ������ � ����������, � ������ �������� ����� ��������� ���� ������� ����� $p_b$, ��� ���� ���� ������� ������������ ��� $p_a = p_b + x$, ��� ���������� �������� � ������������� ������� $x$.
������ �������� ��������� � ������ ��������� � �����������~\cite{chatterjee83}.
��� ���� ����������� ������ �������������� �������� � �������� �����������, � ������� ���� ������ ����� �������� ���������� ������������ ������ � ������������� $\beta \in [0, 1]$.
��� ���� � ������ ��������� � �����������a~\cite{myerson83} ��������, ��� ��� ������������ �������� �������� �� ��������� $\beta = 1/2$ �������� ����������� � ����� ������ ������������ ������ �� ������.

{\progress} 
� ����������� ����������� ���������� � ����������� ������ �������� ������ � ������������ ���������� ������, ���������� ��������� ��������� � �����������, �.�. ������� ������ �� ����, ������ �������� ���������� ������������ ������.

{\aim} ������������ ������������� ��� � �������� �����������, ������������ �������� ����� � ���������������� ���������� ������.

{\researchsubject} �������� ������������ �������� �������������� ������ ���������� �������������� ������� �� ���������� ������.
��������� ������������ �������� ������������� ���� � �������� �����������, ������������ �������� ����� ����� ����� �������� ���������������� ��������.

���~���������� ������������ ���� ���������� ���� ������ ��������� {\tasks}:
\begin{enumerate}
\item 
����������� ������ �������� ������ � ����������� �������� � �������������� ����������� �����.
���������� ������� ����������������� ��������� ������ �� ��������� ������� � ��������� ������.
\item 
����������� ������ �������� ������ � ������������ �������� � ������� ��������� � ������������ ���������� �����.
�������� ����������� ��������� ������� ��� ������������� ����������������� ��������� ������ � ������������ ������������ ������.
\item 
�������� ���������� ������� ������ � ����������� �������� �� ������ ����� �� ������� ���������� ��������� �������� ���� ��������� ������ \todo{� �� ������ ������ ����������� ��������� ��������}.
\end{enumerate}

{\methods} � ����������� ����������� ������ ������ ���, ��������� �������, ������ �������������� � ������������� ����������.

{\novelty}
\begin{enumerate}
\item 
������ �������� ������ � ����������� �������� � ��������� ���������� ������ ����������� �������.
������� ����������� ��������� ���������, ������������� ������������ �� ������������.
\item 
������������ �������������� ������ �������� ������ � ������������� �������� �� ������ ������������� ���������� ��������� ������ ��������� �������.
������� ��������� � ������������� �������� ��������������� ������������� ���� �� ��������� $\Co$.
\end{enumerate}

{\influence} �������� ������� ���� ������������� ��� � �������� �����������, ������������ �������� ����� � ����� ����� ���������� ������, ��� ��������� ������� ������� ����������� ���� ��������� �� ����������� ��������� ������� � ��������� ������.

{\defpositions}
\begin{enumerate}
\item
������� ����������� ������������� ���� �������� ������ � ����������� �������� � ����� ����� ���������� ������.
\item
������� �������������� ������������� ���� �������� ������ � ������������ �������� � ����� ����� ���������� ������.
\item
��������� �����������, ���������� ��� ������ � ����������� ��������, �� ������ ����� �� ������� ���������� ��������� ��� ��������� ������.
\end{enumerate}

{\reliability} ���������� � ������ ����������� ����������� ���������� ������������ ����� � �������������� �������������.
���������� ��������� � ������������ � ������������, ����������� ������� ��������.

{\probation} �������� ����������, ���������� � �����������, ���� ������������ �� ��������� ������� ������������ � ��� ��. �.�. ���������� <<����������� ������>> (2014) � <<������������� ������>> (2016).

\ifthenelse{\equal{\thebibliosel}{0}}{%% ���������� ���������� � ��������� ����� ����� ������ bibtex8
    \publications\ �� ���� ����������� ������� 7 ���������� \cite{pyanykh14, pyanykh16:discr:eng, pyanykh16:discr:ru, pyanykh16:cont, pyanykh16:countable, pyanykh:tikhon2014, pyanykh:lomonosov2016}.
        �������� ���������� ��������������� ������ ������������ � 4 ������� �� ������� ��� \cite{pyanykh14, pyanykh16:discr:ru, pyanykh16:cont, pyanykh16:countable}, 2 "--- � ������� �������� \cite{pyanykh:tikhon2014, pyanykh:lomonosov2016}.
  }{% ���������� ������� biblatex ����� ������ biber
    \begin{refsection}%
        \printbibliography[heading=countauthornotvak, env=countauthornotvak, keyword=biblioauthornotvak, section=1]
        \printbibliography[heading=countauthorvak, env=countauthorvak, keyword=biblioauthorvak, section=1]
        \printbibliography[heading=countauthorconf, env=countauthorconf, keyword=biblioauthorconf, section=1]
        \printbibliography[heading=countauthor, env=countauthor, keyword=biblioauthor, section=1]
        \publications\ �� ���� ����������� ������� \arabic{citeauthor} ����������. \nocite{pyanykh14, pyanykh16:discr:eng, pyanykh16:discr:ru, pyanykh16:cont, pyanykh16:countable, pyanykh:tikhon2014, pyanykh:lomonosov2016}
        �������� ���������� ��������������� ������ ������������ � \arabic{citeauthorvak} ������� �� ������� ���, \nocite{pyanykh14, pyanykh16:discr:ru, pyanykh16:cont, pyanykh16:countable}
        \arabic{citeauthorconf} "--- � ������� ��������\nocite{pyanykh:tikhon2014, pyanykh:lomonosov2016}.
    \end{refsection}
}

%%% Local Variables:
%%% coding: cp1251
%%% mode: latex
%%% TeX-master: "../dissertation"
%%% End:
 % Характеристика работы по структуре во введении и в автореферате не отличается (ГОСТ Р 7.0.11, пункты 5.3.1 и 9.2.1), потому её загружаем из одного и того же внешнего файла, предварительно задав форму выделения некоторым параметрам

\textbf{Объем и структура работы.} Диссертация состоит из~введения, трёх глав и
заключения. Полный объём диссертации составляет
\formbytotal{TotPages}{страниц}{у}{ы}{} и
включает~\formbytotal{totalcount@figure}{рисун}{ок}{ка}{ков}
и~\formbytotal{totalcount@table}{таблиц}{у}{ы}{}. Список литературы содержит
\formbytotal{citenum}{наименован}{ие}{ия}{ий}.

%%% Local Variables:
%%% mode: latex
%%% TeX-master: "../dissertation"
%%% End: