\chapter{Теоретико-игровая модель биржевых торгов с дискретными ставками и счетным множеством состояний} \label{chapt3}
{
% \newcommand{\s}{\ensuremath{s}}
% \newcommand{\q}{\ensuremath{\overbar{q}}}
% \newcommand{\theGame}[1][n]{\ensuremath{G_{#1}}}
% \newcommand{\K}[1][n]{\ensuremath{K_{#1}}}
% \newcommand{\V}[1][n]{\ensuremath{V_{#1}}}
% \newcommand{\High}[1][\ensuremath{\infty}]{\ensuremath{H_{#1}}}
% \newcommand{\sigmav}{\ensuremath{\overbar{\sigma}}}
% \newcommand{\tauv}{\ensuremath{\overbar{\tau}}}
% \newcommand{\sigmak}{\ensuremath{\hat{\sigma}}}
% \newcommand{\Low}[1][\ensuremath{\infty}]{\ensuremath{L_{#1}}}

\newcommand{\as}[1][\beta]{\ensuremath{a^{s,#1}}}
\newcommand{\s}{\ensuremath{s}}
\newcommand{\q}{\ensuremath{\overbar{q}}}
\newcommand{\theG}[1][n]{\ensuremath{G^\beta_{#1}}}
\newcommand{\K}[1][n]{\ensuremath{K^\beta_{#1}}}
\newcommand{\V}[1][n]{\ensuremath{V^\beta_{#1}}}
\newcommand{\High}[1][\ensuremath{\infty}]{\ensuremath{H^\beta_{#1}}}
\newcommand{\sigmav}{\ensuremath{\overbar{\sigma}}}
\newcommand{\tauv}{\ensuremath{\overbar{\tau}}}
\newcommand{\xiv}{\ensuremath{\overbar{\xi}}}
\newcommand{\sigmak}{\ensuremath{\hat{\sigma}}}
\newcommand{\Low}[1][\ensuremath{\infty}]{\ensuremath{L^\beta_{#1}}}

\section{Описание модели рынка со счетным множеством состояний}
\label{ch3:sec:model-descr}

Рассмотрим модель рынка с дискретными ставками и множеством состояний $S = \Z_+$.
Перед началом игры случай выбирает состояние рынка $\s \in S$ в соответствии с вероятностным распределением $\p = (p^s, \; s \in S)$ таким, что дисперсия состояния $\D \p < \infty$.
На каждом шаге игры $t = \overline{1,n}, \; n \leqslant \infty$ игроки делают ставки $i_t \in I, \, j_t \in J$, где $I = J = \Z_+$.
Выплата первому игроку в состоянии $s$ равна
\begin{equation*}
  a^s(i_t, j_t) =
  \begin{cases}
    (1-\beta) i_t + \beta j_t - s, &\; i_t < j_t, \\
    0, &\; i_t = j_t, \\
    s - \beta i_t - (1-\beta)j_t, &\; i_t > j_t.
  \end{cases}
\end{equation*}

Стратегией первого игрока является последовательность ходов $\sigmav = (\sigma_1, \ldots, \sigma_n)$, где $\sigma_t: S \times I^{t-1} \rightarrow \Delta(I)$.
Множество стратегий первого игрока обозначим $\Sigma$.

Стратегией второго игрока является последовательность ходов $\tauv = (\tau_1, \ldots, \tau_n)$, где $\tau_t: I^{t-1} \rightarrow \Delta(J)$.
Множество стратегий второго игрока обозначим $\Tau$.

При использовании игроками стратегий $\sigmav$ и $\tauv$, ожидаемый выигрыш первого игрока равен
\begin{equation}
  \label{ch3:eq:Kn}
  \K(\p, \sigmav, \tauv) =
  \E_{\Pi(\p, \sigmav, \tauv)} \sum_{t=1}^n a^s(i_t, j_t),
\end{equation}
где математическое ожидание берется по мере, индуцированной $\p$, $\sigmav$ и $\tauv$.
Заданную таким образом игру обозначим $\theG(\p)$.

Если для некоторых $\sigmav^* \in \Sigma, \tauv^* \in \Tau$ выполняется
\begin{equation*}
  \inf_{\tauv \in \Tau} \K(\p, \sigmav^*, \tauv) =
  \K(\p, \sigmav^*, \tauv^*) =
  \sup_{\sigmav \in \Sigma} \K(\p, \sigmav, \tauv^*) = 
  \V(\p),
\end{equation*}
то игра $\theG(\p)$ имеет значение $\V(\p)$, а стратегии $\sigmav^*$ и $\tauv^*$
называются оптимальными.

Аналогично тому, как это было сделано в главе \ref{chapt1}, опишем рекурсивную структуру игры $\theG(\p)$.
Представим стратегию первого игрока в виде $\sigmav~=~(\sigma, \sigmav^i, i \in I)$,
где $\sigma$~--- ход игрока на первом шаге, а $\sigmav^i$ --- стратегия в игре
продолжительности $n-1$ в зависимости от ставки $i$ на первом шаге.
Аналогично, стратегию второго игрока представим в виде $\tauv = (\tau, \tauv^i, \; i \in I)$.
%
Далее, обозначим $q^i$ полную вероятность, с которой первый игрок делает ставку $i \in I$, и $\q = (q^i, \; i \in I)$ --- соответствующее распределение.
Также обозначим $p^{s|i}$ апостериорную вероятность состояния $s$ в зависимости от ставки $i$ первого игрока, и $\p^i = (p^{s|i}, \, s \in S)$ --- соответствующее апостериорное распределение.
Тогда для функции выигрыша в игре $\theG[n](\p)$ будет справедлива формула
\begin{equation}
  \label{problem:eq:Kn-recurrence}
  \K[n](\p, \sigmav, \tauv) =
  \K[1](\p, \sigma, \tau) +
  \sum_{i \in I} q^i \K[n-1](\p^i, \sigmav^i, \tauv^i).
\end{equation}

\section{Оценка сверху выигрыша в игре $\mathbf{G^\beta_\infty(\p)}$}
Следуя \cite{domansky11}, рассмотрим чистую стратегию подражания инсайдеру $\tauv^k~=~(\tau^k_t,\ t = \overline{1, \infty})$ неосведомленного игрока, где
\begin{equation*}
  \tau^k_1 = k, \quad
  \tau^k_t(i_{t-1}, j_{t-1}) = \begin{cases}
    j_{t-1} - 1, &\; i_{t-1} < j_{t-1},\\
    j_{t-1}, &\; i_{t-1} = j_{t-1},\\
    j_{t-1}+1, &\; i_{t-1} > j_{t-1}.
  \end{cases}
\end{equation*}

\begin{lemma}
  \label{ch3:upper-bound:lemma:vector-payoffs}
  При применении стратегии $\tauv^k$ в игре $\theG(\p)$ второй игрок в состоянии $s$ гарантирует себе проигрыш не более
  \begin{gather*}
    h^s_n(\tauv^k) = \sum_{t=0}^{n-1} (k-s-t-1+\beta)^+, \; s \leqslant k, \\
    h^s_n(\tauv^k) = \sum_{t=0}^{n-1} (s-k-t-\beta)^+, \; s > k.
  \end{gather*}
  Последовательность $\left\{ h^s_n(\tauv^k), \; n = \overline{1, \infty}
  \right\}$ не убывает, ограничена сверху и сходится к %
  \begin{equation}
    \label{upper-bound:eq:max-payoff}
    h^s_\infty(\tauv^k) = (s - k + 1 - 2\beta)(s-k)/2.
  \end{equation}
\end{lemma}
\begin{proof}
  Проведем доказательство по индукции для случая $s > k$.
  При $n = 1$ оптимальный ответ игрока 1 на $\tauv^k$ будет $i = k + 1$.
  Тогда его выигрыш в игре $\theG[1](\p)$ равен
  \begin{equation*}
    h^s_1(\tauv^k) = s - \beta(k+1) - (1-\beta)k = s - k - \beta.
  \end{equation*}
  База индукции проверена.
  Предположим, что утверждение верно при $n \leqslant N$.
  При $n = N + 1$ игрок 1 имеет два разумных ответа на $\tauv^k$: ставка $i = k + 1$, что соответствует покупке акции по наименьшей возможной цене, и ставка $i = k - 1$, что соответствует продаже акции за наибольшую возможную цену.
  Найдем оценки выигрыша в каждом из случаев.
  Для $i = k + 1$ выигрыш игрока 1 не превосходит величины
  \begin{equation*}
    s - k - \beta + h^s_N(\tauv^{k+1}) = \sum_{t=0}^N(s-k-t-\beta)^+.
  \end{equation*}
  Аналогично для $i = k - 1$ тот же выигрыш не превосходит
  \begin{equation*}
    \beta k + (1-\beta)(k-1) - s + h^s_N(\tauv^{k-1}) = \sum_{t=0}^{N-2}(s-k-t-\beta)^+.
  \end{equation*}
  При $s \leqslant k$ формула для $h^s_n(\tauv^k)$ доказывается аналогично.
  Сходимость $h^s_n(\tauv^k)$ к $h^s_\infty(\tauv^k)$ следует из равенства $h^s_n(\tauv^k) = h^s_{n+1}(\tauv^k)$ при $n \geqslant s - k$.
\end{proof}

Введем следующие обозначения для множества распределений на $S$ с заданным математическим ожиданием состояния:
\begin{gather*}
  \Theta(x) = \left\{ \p' \in \Delta(S): \E \p' = x \right\},\\
  \Lambda(x, y) = \left\{ \p' \in \Delta(S): x < \E \p' \leqslant y \right\}.
\end{gather*}

Пусть $\tauv^*$ --- стратегия второго игрока, состоящая в применении $\tauv^k$ при $\p~\in~\Lambda(k-1+\beta, k+\beta)$.

\begin{theorem}
  \label{ch3:upper-bound:theorem}
  При использовании вторым игроком стратегии $\tauv^*$, выигрыш первого игрока в игре $\theG[\infty](\p)$ ограничен сверху функцией
  \begin{equation*}
    \High(\p) = \min_{k \in J} \sum_{s \in S} p^s  h^s_\infty(\tauv^k).
  \end{equation*}
  Функция $\High(\p)$ является кусочно-линейной с областями линейности $\Lambda(k - 1 + \beta, k + \beta)$ и областями недифференцируемости $\Theta(k+\beta)$ при $k \in S$.
  Для распределений $\p$ таких, что $\E \p = k - 1 + \beta + \xi, \; \xi \in (0, 1]$, ее значение равно
  \begin{equation}
    \label{upper-bound:eq:H(p)}
    \High(\p) = \left( \D \p + \beta(1-\beta) - \xi(1-\xi) \right)/2.
  \end{equation}
\end{theorem}
\begin{proof}
  Воспользовавшись \eqref{upper-bound:eq:max-payoff}, получим
  \begin{equation}
    \label{ch3:upper-bound:theorem:eq:1}
    \begin{gathered}
    \sum_{s \in S} p^s h^s_\infty(\tauv^j) = \bigl(
      j^2 + (2\beta - 1 - 2 \E \p)j - \\
      - (2\beta - 1) \E \p + \E \p^2 
    \bigr)/2.
    \end{gathered}
  \end{equation}
  
  Квадратичная функция $f(x) = x^2 + (2\beta - 1 - 2\E \p)x$ достигает минимума
  при $x = \E \p - \beta + 1/2$. Отсюда при $\p \in \Lambda(k - 1 + \beta, k +
  \beta)$ выражение \eqref{ch3:upper-bound:theorem:eq:1} достигает минимума при $j =
  k$. Равенство \eqref{upper-bound:eq:H(p)} проверяется непосредственной
  подстановкой $\E \p = k - 1 + \beta + \xi$ в \eqref{ch3:upper-bound:theorem:eq:1}.
\end{proof}

Заметим, что как и в \cite{pyanykh16:discr:ru}, в данном случае наблюдается сдвиг областей линейности на $\beta$ относительно $\E \p$ в сравнении с результатами из \cite{domansky11}.
Кроме того, хотя описание чистой стратегии $\tauv^k$ неосведомленного игрока повторяет описание из \cite{domansky07}, его стратегия $\tauv^*$ зависит от $\beta$ в силу того, что от $\beta$ зависит выбор конкретной $\tauv^k$ в определении $\tauv^*$.

\section{Оценка снизу выигрыша в игре $\mathbf{G^\beta_\infty(\p)}$}
\label{ch3:sec:lower-bound}

Перейдем к описанию стратегии первого игрока, которая гарантирует ему выигрыш не менее $\High(\p)$.
Пусть $\sigma^s_i$ --- компонента хода $\sigma$ первого игрока, т.е. вероятность сделать ставку $i$ в состоянии $s$.
По правилу Байеса
\[
  \sigma^s_i = p^{s|i} q^i / p^s.
\]
В частности, справедливо $\sum_{s \in S} \sigma^s_i p^s = q^i,\ i \in I$.
Таким образом, ход первого игрока можно определить, задав следующие параметры: полные вероятности $q^i$ сделать ставку $i$ и апостериорные вероятности $p^{s|i}$ для $i \in I$.
Тогда в терминах $\q$, $\p^i$ его одношаговый выигрыш выражается следующим образом:
\begin{equation}
  \label{ch3:lower-bound:eq:K1(q,pi)}
  \K[1](\p, \sigma, j) = \sum_{i \in I} \sum_{s \in S} q^i p^{s|i} \as(i, j).
\end{equation}

Обозначим через $\Low[n](\p)$ --- максимальный выигрыш, который может гарантировать себе первый игрок в игре $\theG(\p)$.
\begin{lemma}
  \label{ch3:lower-bound:lemma:convex-comb}
  Пусть $\p_k$ --- распределение из $\Delta(S)$, $\sigmav_k$ --- стратегия игрока 1, которая гарантирует ему выигрыш $\Low[n](\p_k)$ в игре $\theG[n](\p_k)$, и $\q_k = (q^1_k, \ldots, q^n_k),\ \p^i_k = ( p^{1|i}_k, p^{2|i}_k, \ldots)$ --- векторы полных вероятностей ставок и апостериорных вероятностей состояния, соответствующие первому ходу стратегии $\sigmav_k,\ k = 1,2$.
  Тогда для $\p = \lambda \p_1 + (1-\lambda) \p_2, \; \lambda \in [0, 1],$ стратегия $\sigmav_c$, первый ход которой задается параметрами
  \begin{equation}
    \label{ch3:lower-bound:eq:q-pi}
    q^i = \lambda q^i_1 + (1-\lambda) q^i_2, \quad
    p^{s|i} = \left(\lambda q^i_1 p^{s|i}_1 + (1-\lambda) q^i_2 p^{s|i}_2\right)/q^i,
  \end{equation}
  гарантирует игроку 1 выигрыш $\lambda \Low[n](\p_1) + (1-\lambda) \Low[n](\p_2)$.
\end{lemma}
\begin{proof}
  Проведем доказательство по индукции.
  Покажем, что справедливо равенство %
  \begin{equation}
    \label{ch3:lower-bound:lemma:convex-comb:eq:1}
    \K[n](\p, \sigmav_c, \tauv) =
    \lambda \K[n](\p_1, \sigmav_1, \tauv) +
    (1-\lambda)\K[n](\p_1, \sigmav_2, \tauv).
  \end{equation}
  Подставив \eqref{ch3:lower-bound:eq:q-pi} в \eqref{ch3:lower-bound:eq:K1(q,pi)}, получим
  \begin{gather*}
    \K[1](\p, \sigmav_c, j) = \sum_{i \in I, \, s \in S}
    q^i \frac{\lambda q^i_1 p^{s|i}_1 + (1-\lambda) q^i_2 p^{s|i}_2}{q^i} \as(i,j) = \\
    = \lambda \sum_{i \in I, \, s \in S} q^i_1 p^{s|i}_1 \as(i,j) +
    (1-\lambda) \sum_{i \in I, s \in S} q^i_2 p^{s|i}_2 \as(i, j) = \\
    = \lambda \K[1](\p_1, \sigmav_1, j) +
    (1-\lambda)\K(\p_2, \sigmav_2, j).
  \end{gather*}
  Таким образом, утверждение справедливо при $n = 1$.
  Пусть утверждение имеет место при $n \leqslant N$.
  Тогда из \eqref{problem:eq:Kn-recurrence} вытекает
  \begin{gather*}
    \K[N+1](\p, \sigmav_c, \tauv) =
    \K[1](\p, \sigmav_c, \tauv) +
    \sum_{i \in I} q_i \K[N](\p^i, \sigmav^i_c, \tau^i) = \\
    = \lambda \K[1](\p_1, \sigma_1, \tau) +
    (1-\lambda) \K[1](\p_2, \sigma_2, \tau) + \\
    + \sum_{i \in I} q^i \left(
      \frac{\lambda q^i_1}{q^i} \K[N](\p^i_1, \sigmav^i_1, \tauv^i) +
      \frac{(1-\lambda) q^i_2}{q^i} \K[N](\p^i_2, \sigmav^i_2, \tauv^i)
    \right) = \\
    = \lambda \K[N+1](\p_1, \sigmav_1, \tauv) +
    (1-\lambda)\K[N+1](\p_1, \sigmav_2, \tauv).
  \end{gather*}
  Справедливость равенства \eqref{ch3:lower-bound:lemma:convex-comb:eq:1} доказана.
  Отсюда получаем
  \begin{multline*}
    \Low[n](\p) = \min_{\tauv \in \Tau} \K[n](\p, \sigmav_c, j) \geqslant
    \lambda \min_{\tauv \in \Tau} \K[n](\p, \sigmav_1, j) + \\
    + (1-\lambda) \min_{\tauv \in \Tau} \K[n](\p, \sigmav_2, j) =
    \lambda \Low[n](\p_1) + (1-\lambda) \Low[n](\p_2).
  \end{multline*}
  Получили, что стратегия $\sigmav_c$ обеспечивает игроку 1 в игре $\theG[n](\p)$ соответствующую выпуклую комбинацию гарантированных выигрышей в играх $\theG[n](\p_1)$ и $\theG[n](\p_2)$.
\end{proof}

Из теоремы~\ref{ch3:upper-bound:theorem} и леммы~\ref{ch3:lower-bound:lemma:convex-comb} следует, что для доказательства совпадения верхней и нижней оценок выигрыша в игре $\theG[\infty](\p)$ можно ограничиться рассмотрением только распределений $\p~\in~\Theta(k + \beta), \; k \in S$.
Как показано в \cite{domansky11}, любое $\p$ может быть представлено в виде выпуклой комбинации распределений с двухточечным носителем.
Обозначим $\p^x(l, r) \in \Theta(x)$ распределение с математическим ожиданием $x$ и носителем $\{l, r\}$.
Таким образом, достаточно доказать выполнение равенства $\Low(\p)~=~\High(\p)$ для распределения $\p = \p^{k+\beta}(l,r), \; k \in S$.
Построим соответствующую стратегию первого игрока.

Обозначим $\sigmak_k$ ход первого игрока, состоящий в применении действий $k$ и $k+1$.
Ход $\sigmak_k$ определяется заданием полных вероятностей $q^k, q^{k+1}$ и апостериорных распределений $\p^k, \p^{k+1}$, причем $q^k + q^{k+1} = 1$.
Следующая лемма является обобщением утверждения~\ref{ch1:prop:K1-base} из главы~\ref{chapt1}.
\begin{lemma}
  \label{ch3:lower-bound:lemma:stage-payoff}
  При использовании $\sigmak_k$ одношаговый выигрыш первого игрока равен
  \begin{equation*}
    \K[1](\p, \sigmak_k, j) = \begin{cases}
      \E \p - \beta k - (1-\beta) j - \beta q^{k+1}, &\; j < k, \\
      (\E \p^{k+1} - k - \beta) q^{k+1}, &\; j = k, \\
      (k + \beta - \E \p^k) q^k, &\; j = k+1, \\
      (1-\beta) k + \beta j - \E \p + (1-\beta) q^{k+1}, &\; j > k + 1.
    \end{cases}
  \end{equation*}
\end{lemma}
\begin{proof}
  Несложно проверить, что
  \begin{equation*}
    \as(\sigmak_k, j) = \begin{cases}
      s - \beta k - (1-\beta) j - \beta \sigma^s_{k+1}, &\; j < k,\\
      (s - k - \beta) \sigma^s_{k+1}, &\; j = k,\\
      (k + \beta - s) \sigma^s_k, &\; j = k+1,\\
      (1-\beta) k + \beta j - s + (1-\beta) \sigma^s_{k+1}, &\; j > k + 1.
    \end{cases}
  \end{equation*}
  Отсюда и из \eqref{ch3:eq:Kn} следует утверждение леммы.
\end{proof}

Распространяя результаты главы~\ref{chapt1} на случай $\p^{k+\beta}(l, r)$, определим следующую стратегию первого игрока в игре $\theG[\infty](\p)$.
Введем обозначение %
\begin{equation*}
  P(l,r) = \left\{
    \p^k(l, r), \, \p^{s+\beta}(l, r), \, k = \overline{l,r}, s = \overline{l,r-1}
  \right\}.
\end{equation*}
При $\p \in P(l,r)$ первый ход стратегии $\sigmav^*$ определяется следующим образом.
Если $\p = \p^l(l,r)$ или $\p = \p^r(l,r)$ игрок 1 использует ставки $l$ и $r$, соответственно, с вероятностью $1$.
Иначе первый игрок использует $\sigmak_k$ с параметрами
\begin{equation}
  \label{ch3:eq:insider-strategy1}
  \begin{aligned}
    &\text{при } \p = \p^k(l, r),\ k \in \overline{l+1, r-1}:&
    &q^k = \beta,&
    &q^{k+1} = 1-\beta,\\
    && 
    &\p^k = \p^{k-1+\beta}(l, r),&
    &\p^{k+1} = \p^{k+\beta}(l, r);\\
    %
    &\text{при } \p = \p^{k+\beta}(l, r),\ k \in \overline{l, r-1}:&
    &q^k = 1-\beta,&
    &q^{k+1} = \beta,\\
    &&
    &\p^k = \p^k(l, r),&
    &\p^{k+1} = \p^{k+1}(l, r).&
  \end{aligned}
\end{equation}

На последующих шагах игры таким образом определенный ход применяется рекурсивно для соответствующих значений апостериорных вероятностей.
Для остальных распределений $\p$ стратегия $\sigmav^*$ определяется конструкцией леммы~\ref{ch3:lower-bound:lemma:convex-comb}.

Обозначим $L^x_{l,r} = \Low(\p^x(l,r))$.
Следующая теорема является обобщением утверждения~\ref{ch1:prop:lower:recurrence-solution}.
\begin{theorem}
  \label{lower-bound:theorem}
  При использовании стратегии $\sigmav^*$ в игре $\theG[\infty](\p)$ для распределения $\p \in P(l,r)$ гарантированный выигрыш первого игрока удовлетворяет следующей системе:
  \begin{equation}
    \label{ch3:lower-bound:eq:Linf-recurrence}
    \begin{gathered}
      L^{k+\beta}_{l,r} =
      \beta(1-\beta) + (1-\beta) L^k_{l,r} + \beta L^{k+1}_{l,r}, \;
      k \in \overline{l, r - 1}, \\
      L^k_{l,r} =
      \beta L^{k-1+\beta}_{l,r} + (1-\beta) L^{k+\beta}_{l,r}, \;
      k \in \overline{l + 1, r - 1},\\
      L^l_{l,r} = L^r_{l,r} = 0.
    \end{gathered}
  \end{equation}
  Ее решение дает нижнюю оценку выигрыша первого игрока равную
  \begin{equation*}
    \Low(\p^{k+\beta}(l, r)) = ((r-k-\beta)(k+\beta-l) + \beta(1-\beta))/2.
  \end{equation*}
\end{theorem}
\begin{proof}
  Для $\p \in P(l,r)$ определение стратегия $\sigmav^*$ аналогично определению оптимальной стратегии первого игрока из главы~\ref{chapt1} с заменой $0, m$ на $l, r$ соответственно.

  Параметры $\q$ и $\p^i$ подобраны таким образом, чтобы выполнялись равенства
  \[
    \Low[1](\p^k(l,r)) = 0, \, \Low[1](\p^{k+\beta}(l,r)) = \beta(1-\beta),
  \]
  а апостериорные распределения принадлежали тому же множеству $P(l,r)$.
  Полученная система \eqref{ch3:lower-bound:eq:Linf-recurrence} является системой с трехдиагональной матрицей и решается методом прогонки аналогично тому, как это было сделано в разделе \ref{ch1:lower-bound} данной работы.
\end{proof}

В силу справедливости равенства
\[
  \D p^{k+\beta}(l, r) = (r-k-\beta)(k+\beta-l)
\]
получаем, что выражения для $\High(\p^{k+\beta}(l,r))$ и $\Low(\p^{k+\beta}(l,r))$ совпадают.

\section{Решение игры $\mathbf{G^\beta_\infty(\p)}$}
\label{ch3:sec:game-solution}

Отметим, что приведенная в разделе~\ref{ch3:sec:lower-bound} стратегия инсайдера $\sigma^*$ определена только при $\beta \in (0, 1)$.

Нетрудно проверить справедливость следующего равенства:
\begin{equation}
  \label{ch3:eq:as-symmetric}
  a^{r, \beta}(i, j) = a^{l, 1-\beta}(r+l-i, r+l-j).
\end{equation}

Из (\ref{ch3:eq:Kn}) и (\ref{ch3:eq:as-symmetric}) видно, что решение игры $\theG[\infty](\p^x(l, r))$ сводится к решению игры $G^{1-\beta}_\infty(\p^{l+r-x}(l,r))$.
При этом ставки, используемые в соответствующих смешанных стратегиях инсайдера, симметричны относительно $(l+r)/2$.
Аналогичные рассуждения справедливы для $\theG[\infty](\p)$ при любом $\p$.

Оптимальная стратегия $\sigmav^*$ инсайдера в игре $\theG[\infty](\p)$ при $\beta = 1$ приведена в~\cite{domansky11}.
Решение $\theG[\infty](\p)$ при $\beta = 0$ может быть получено при помощи описанной выше конструкции.
Таким образом, при любом $\beta \in [0, 1]$ справедлива следующая
\begin{theorem}
  Игра $\theG[\infty](\p)$ имеет значение
  \[
    \V[\infty](\p) = \High(\p) = \Low(\p).
  \]
  Стратегии $\tauv^*$ и $\sigmav^*$, определенные ранее, являются оптимальными.
\end{theorem}

% \begin{figure}[tbh]
%   \small
%   \centering
%   \begin{tikzpicture}
%     [
%     auto,node distance=1.25cm,
%     trans/.style={->,shorten >=1pt,>=stealth',semithick},
%     state/.style={shape=circle,draw,minimum size=2mm}
%     ]
%     \node[state,label={$p^l(l,r)$}] (p0) {};
%     \node[state,right=of p0,label={$p^{l+\beta}(l,r)$}] (p1) {}; 
%     \node[state,right=of p1,label={$p^{l+1}(l,r)$}] (p2) {};
%     \node[right=of p2] (others) {$\ldots$};
%     \node[state,right=of others,label={$p^{r-1+\beta}(l,r)$}] (p2mm1) {};
%     \node[state,right=of p2mm1,label={$p^r(l,r)$}] (p2m) {};
    
%     \path [trans]
%     (p0) edge [loop left,min distance=10mm,out=205,in=155] node {$1$} (p0)
%     (p1) edge[bend right] node[below] {$1-\beta$} (p0)
%     (p1) edge[bend left] node[below] {$\beta$} (p2)
%     (p2) edge[bend left] node[below] {$\beta$} (p1)
%     (p2) edge[bend right] node[below] {$1-\beta$} (others)
%     (p2mm1) edge[bend left] node[below] {$1-\beta$} (others)
%     (p2mm1) edge[bend right] node[below] {$\beta$} (p2m)
%     (p2m) edge[loop right,min distance=10mm,out=25,in=-25] node {$1$} (p2m)
%     ;
%   \end{tikzpicture}
%   \caption[Последовательность апостериорных вероятностей]{Случайное блуждание последовательности апостериорных вероятностей, порожденное $\sigmav^*$}
%   \label{fig:posterior-1}
% \end{figure}

% Использование игроком~1 стратегии $\sigmav^*$ порождает случайное блуждание последовательности апостериорных вероятностей, изображенное на рисунке~\ref{fig:posterior-1}, которое в отличие от \cite{bib:domansky11} происходит по более широкому множеству и уже не является симметричным, кроме случая $\beta = 1/2$.

В дополнение к стратегии, определенной выражением (\ref{ch3:eq:insider-strategy1}), приведем еще одну оптимальную стратегию инсайдера.
Введем обозначение %
\begin{equation*}
  P'(l,r) =
  \{\p^l(l, r), \p^r(l, r)\}
  \cup
  \left\{
    \p^{k+\beta}(l, r), \, k = \overline{l, r-1}
  \right\}.
\end{equation*}

При $\p \in P'(l,r)$ первый ход стратегии $\xiv^*$ определяется следующим образом.
Если $\p = \p^l(l,r)$ или $\p = \p^r(l,r)$ первый игрок использует ставки $l$ и $r$, соответственно, с вероятностью $1$.
Иначе первый игрок использует $\sigmak_k$ с параметрами
\begin{equation}
  \label{ch3:eq:insider-strategy2}
  \begin{aligned}
    &\text{при } \p = \p^{l+\beta}(l, r):&
    &q^l = \frac{1}{1+\beta},&
    &q^{l+1} = \frac{\beta}{1+\beta},\\
    && 
    &\p^l = \p^l(l, r),&
    &\p^{l+1} = \p^{l+1+\beta}(l, r);\\
    %
    &\text{при } \p^{r-1+\beta}(l, r):&
    &q^{r-1} = \frac{1-\beta}{2-\beta},&
    &q^r = \frac{1}{2-\beta},\\
    && 
    &\p^{r-1} = \p^{r-1}(l, r),&
    &\p^r = \p^r(l, r);\\
    %
    &\text{при } \p^{k+\beta}(l, r),\ k \in \overline{l+1, r-2}:&
    &q^k = \frac{1}{2},&
    &q^{k+1} = \frac{1}{2},\\
    &&
    &\p^k = \p^{k-1+\beta}(l, r),&
    &\p^{k+1} = \p^{k+1+\beta}(l, r).&
  \end{aligned}
\end{equation}
Для остальных распределений $\p$ стратегия $\xiv^*$ так же определяется конструкцией леммы~\ref{ch3:lower-bound:lemma:convex-comb}.

Использование стратегии $\xiv^*$ порождает случайное блуждание последовательности апостериорных вероятностей, изображенное на рисунке~\ref{ch3:fig:posterior-2}.
Данное случайное блуждание симметрично с вероятностями перехода в соседние состояния равными $1/2$, симметрия нарушается только на краях.

\begin{figure}[tbh]
  \small
  \centering
  \begin{tikzpicture}
    [
    auto,yscale=1.0,node distance=2.50cm,
    trans/.style={->,shorten >=1pt,>=stealth',semithick},
    state/.style={shape=circle,draw,minimum size=2mm}
    ]
    \node[state,label={$p^l(l,r)$}] (p0) {};
    \node[state,right=of p0,label={$p^{l+\beta}(l,r)$}] (p1) {}; 
    \node[state,right=of p1,label={$p^{l+1+\beta}(l,r)$}] (p2) {};
    \node[right=of p2] (others) {$\ldots$};
    \node[state,right=of others,label={$p^{r-1+\beta}(l,r)$}] (p2mm1) {};
    \node[state,right=of p2mm1,label={$p^r(l,r)$}] (p2m) {};
    
    \path [trans]
    (p0) edge [loop left,min distance=10mm,out=205,in=155] node {$1$} (p0)
    (p1) edge[bend right] node[below] {$\frac{1}{1+\beta}$} (p0)
    (p1) edge[bend left] node[below] {$\frac{\beta}{1+\beta}$} (p2)
    (p2) edge[bend left] node[below] {$\frac{1}{2}$} (p1)
    (p2) edge[bend right] node[below] {$\frac{1}{2}$} (others)
    (p2mm1) edge[bend left] node[below] {$\frac{1-\beta}{2-\beta}$} (others)
    (p2mm1) edge[bend right] node[below] {$\frac{1}{2-\beta}$} (p2m)
    (p2m) edge[loop right,min distance=10mm,out=25,in=-25] node {$1$} (p2m)
    ;
  \end{tikzpicture}
  \caption[Последовательность апостериорных вероятностей]{Случайное блуждание последовательности апостериорных вероятностей, порожденное $\xiv^*$}
  \label{ch3:fig:posterior-2}
\end{figure}

При использовании стратегии $\xiv^*$ в игре $\theG[\infty](\p)$ для распределения $\p \in P(l,r)$ гарантированный выигрыш первого игрока удовлетворяет следующей системе:
\begin{equation*}
  \begin{gathered}
    L^{k+\beta}_{l,r} =
    \frac{1}{2} + \frac{1}{2} L^k_{l,r} + \frac{1}{2} L^{k+1}_{l,r}, \;
    k \in \overline{l+1, r-2}, \\
    L^{l+\beta}_{l,r} =
    \frac{\beta}{1+\beta} + \frac{1}{1+\beta} L^l_{l,r} + \frac{\beta}{1+\beta} L^{l+1+\beta}_{l,r},\\
    L^{r-1+\beta}_{l,r} =
    \frac{1-\beta}{2-\beta} + \frac{1-\beta}{2-\beta} L^{r-2+\beta}_{l,r} + \frac{1}{2-\beta} L^r_{l,r},\\
    L^l_{l,r} = L^r_{l,r} = 0.
  \end{gathered}
\end{equation*}

Нетрудно проверить, что подстановкой $\High(\p^{k+\beta}(l,r))$ вместо $L^{k+\beta}_{l,r}$ данные равенства обращаются в тождества.
Отсюда получаем, что стратегия $\xiv^*$ является оптимальной.

Отметим, что в отличие от стратегии $\sigmav^*$ стратегия $\xiv^*$ определена при $\beta~\in~[0, 1]$ и совпадает с оптимальной стратегией инсайдера из \cite{domansky11} при $\beta = 1$.
При этом стратегии $\sigmav^*$ и $\xiv^*$ порождают существенно отличные случайные блуждания апостериорных вероятностей.

\clearpage
}

%%% Local Variables:
%%% mode: latex
%%% TeX-master: "../dissertation"
%%% End:
