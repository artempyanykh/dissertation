%% Согласно ГОСТ Р 7.0.11-2011:
%% 5.3.3 В заключении диссертации излагают итоги выполненного исследования, рекомендации, перспективы дальнейшей разработки темы.
%% 9.2.3 В заключении автореферата диссертации излагают итоги данного исследования, рекомендации и перспективы дальнейшей разработки темы.
\begin{enumerate}
  \item
    Получено решение повторяющейся игры с неполной информацией неограниченной продолжительности, моделирующей биржевые торги с дискретными ставками и более общим механизмом торгов.
    На основе анализа функции значения игры в зависимости от параметра $\Co$ механизма торгов показано, что механизмы, предписывающие продажу рискового актива по наибольшей или наименьшей цене гарантируют инсайдеру наименьший выигрыш.
  \item
    Найдено решение $n$-шаговой повторяющейся игры с неполной информацией, моделирующей биржевые торги с непрерывными ставками и более общим механизмом торгов.
    Показано, что, хотя стратегии инсайдера и неосведомленного игрока зависят от значения параметра $\Co$, значение $n$-шаговой игры от него не зависит, что существенно отличает непрерывный случай от дискретного.
  \item
    Получено обобщение результатов для дискретной модели на случай рынка со счетным множеством состояний.
    Показано, что введение более общего механизма торгов с параметром $\Co$ приводит к сдвигу решетки апостериорных вероятностей на $\Co$.
\end{enumerate}

%%% Local Variables:
%%% mode: latex
%%% TeX-master: "../dissertation"
%%% End:
