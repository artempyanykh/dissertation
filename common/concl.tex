%% Согласно ГОСТ Р 7.0.11-2011:
%% 5.3.3 В заключении диссертации излагают итоги выполненного исследования, рекомендации, перспективы дальнейшей разработки темы.
%% 9.2.3 В заключении автореферата диссертации излагают итоги данного исследования, рекомендации и перспективы дальнейшей разработки темы.
\begin{enumerate}
  \item
    Получено решение повторяющейся игры с неполной информацией неограниченной продолжительности, моделирующей биржевые торги с дискретными ставками и обобщенным механизмом формирования сделки.
    На основе анализа функции значения игры в зависимости от параметра $\Co$ механизма формирования сделки показано, что механизмы, предписывающие продажу рискового актива по наибольшей или наименьшей цене гарантируют инсайдеру наименьший выигрыш.
    Полученные результаты обобщены на случай модели рынка со счетным множеством состояний.
  \item
    Найдено решение $n$-шаговой игровой модели торгов с двумя состояниями рынка, непрерывными ставками и обобщенным механизмом формирования сделки.
    Показано, что, хотя оптимальные стратегии игроков зависят от значения параметра $\Co$, значение $n$-шаговой игры от него не зависит, что существенно отличает непрерывный случай от дискретного.
  \item
    Проведен анализ динамики игрового взаимодействия, возникающей при применении игроками оптимальных стратегий, в дискретном и непрерывном случаях.
    Для дискретного случая показано, что последовательность апостериорных вероятностей представляет собой однородную марковскую цепь, и дан явный вид ожидаемой продолжительности торгов.
    Для непрерывной модели показано, что динамика не зависит от параметра механизма.
\end{enumerate}

Появление случайных блужданий цен сделок в более общей модели позволяет сделать вывод о том, что данный феномен не является специфическим свойством конкретной модели.
Это, в свою очередь, подтверждает гипотезу о стратегическом происхождении случайных флуктуаций цен на фондовых рынках.

%%% Local Variables:
%%% mode: latex
%%% TeX-master: "../dissertation"
%%% End:
