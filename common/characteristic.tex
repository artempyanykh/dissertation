{\actuality}
Повторяющиеся игры с неполной информацией представляют собой естественную модель для анализа информационного аспекта в продолжительном стратегическом взаимодействии агентов и позволяют ответить на вопросы о том, как быстро происходит раскрытие приватной информации, каковы эффективные механизмы ее сокрытия и какую выгоду из нее могут извлечь агенты.

Теория получила свое рождение в классических работах Харшаньи~\cite{harsanyi67}, Аумана, Машлера и Стернса~\cite{aumann95}. 
Наиболее полно исследованы повторяющиеся антагонистические игры двух лиц с неполной информацией у одной из сторон. 
В таких играх информационная неопределенность моделируется введением множества $S$ возможных состояний природы. 
Перед началом игры ходом случая выбирается конкретное состояние $s \in S$ в соответствии с некоторым вероятностным распределением, известным обоим игрокам.
После чего игроки на протяжении $n$ шагов играют в игру, соответствующую состоянию $s$.
При этом первый игрок осведомлен о выбранном значении $s$, в то время как второй знает только то, что первый обладает приватной информацией.

Одной из областей приложения данной теории является анализ поведения агентов на финансовых рынках.
Начиная с работы Башелье~\cite{bachelier1900}, для описания эволюции цен на активы используются винеровские процессы или "--- в дискретном случае --- случайные блуждания.
Возникновение случайных колебаний цен на рынке принято объяснять влиянием на процесс ценообразования множества слабых независимых внешних факторов. 
Однако гипотеза о полностью экзогенном происхождении случайных колебаний цен не является удовлетворительной. 
Гипотеза об их стратегическом происхождении была продемонстрирована в работе Де~Мейера и Салей~\cite{demeyer02}, где винеровская компонента в эволюции цен возникает в
следствие асимметричной информированности агентов. 
В рамках данной модели два игрока на протяжении $n$ шагов ведут торговлю однотипными рисковыми активами, причем один из них знает настоящую цену актива. 
На каждом шаге они делают вещественные ставки, и игрок, предложивший б\'{о}льшую ставку, покупает у
другого актив по предложенной цене; при равенстве ставок сделка не состоится.
В работе Марино и Де~Мейера~\cite{demeyer05}, а также в работе В.~К.~Доманского~\cite{domansky07} была исследована модель биржевых торгов с дискретными ставками, где было показано, что последовательность цен актива образует простое случайное блуждание.

В указанных работах использовался один и тот же механизм торгов, а именно "--- продажа по наибольшей цене.
При этом известно, что выбор конкретного механизма может существенным образом влиять на стратегическое поведение агентов и на их выигрыши в результате взаимодействия.
Как было отмечено авторами в работе~\cite{demeyer02}, предположительно, причиной возникновения броуновского движения является полная симметрия между игроками (кроме информационной асимметрии).
В частности, анализ модели с более общим симметричным механизмом торгов отмечен как одно из дальнейших направлений исследования.

В работе~\cite{demeyer10} Де~Мейером рассмотрена непрерывная модель с достаточно общим механизмом торгов.
Основной результат данной статьи заключается в том, что в асимптотике процесс эволюции цен на рисковый актив не зависит от конкретного механизма, а зависит только от априорного распределения цены актива.
Однако, одним из условий, накладываемых на механизм торгов, в рамках которых автором получены результаты, является существование значения соответствующих игр конечной продолжительности.

В работе М.~C.~Сандомирской~\cite{sandomirskaya14} рассмотрена дискретная модель торгов с механизмом, в рамках которого игрок назначает цену покупки акции $p_b$, при этом цена продажи определяется как $p_a = p_b + x$, так называемый механизм с фиксированным спрэдом $x$.
Другой механизм предложен в работе Чаттерджи и Самуэльсона~\cite{chatterjee83}.
Ими была рассмотрена модель двухстороннего аукциона с неполной информацией, в котором цена сделки равна выпуклой комбинации предложенных ставок с коэффициентом $\beta \in [0, 1]$.
При этом в работе Майерсона и Саттертвейтa~\cite{myerson83} показано, что при определенных условиях механизм со значением $\beta = 1/2$ является оптимальным с точки зрения максимизации дохода от торгов.
Подробное рассмотрение вопросов, связанных с переговорами при заключении сделок дано в книге В.~В.~Мазалова, А.~Э.~Менчера, Ю.~С.~Токаревой~\cite{mazalov12}.

{\progress} 
В диссертации рассмотрены дискретные и непрерывные модели биржевых торгов с симметричным механизмом торгов, отвечающим механизму Чаттерджи и Самуэльсона, т.е. продажа актива по цене, равной выпуклой комбинации предложенных ставок.

{\aim} Исследование повторяющихся игр с неполной информацией, моделирующих биржевые торги с модифицированным механизмом торгов.

{\researchsubject} Объектом исследования являются математические модели механизмов взаимодействия агентов на финансовых рынках.
Предметом исследования являются повторяющиеся игры с неполной информацией, моделирующие биржевые торги между двумя различно информированными агентами.

Для~достижения поставленной цели необходимо было решить следующие {\tasks}:
\begin{enumerate}
\item 
Исследовать модель биржевых торгов с дискретными ставками и неограниченным количеством шагов.
Определить влияние модифицированного механизма торгов на поведение агентов и результат торгов.
\item 
Обобщить результаты анализа модели с дискретными ставками на случай рынка со счетным множеством возможных значений цены рискового актива.
\item 
Исследовать модель биржевых торгов с непрерывными ставками в случаях конечного и бесконечного количества шагов.
Сравнить оптимальное поведение агентов при использовании модифицированного механизма торгов с результатами оригинальной модели.
\end{enumerate}

{\methods} В диссертации применялись методы теории игр, выпуклого анализа, теории двойственности и вариационного исчисления.

{\novelty}
\begin{enumerate}
\item 
Модель биржевых торгов с дискретными ставками и указанным механизмом торгов исследуются впервые.
Найдена оптимальная стратегии инсайдера, принципиально отличающаяся от существующих.
\item 
Исследование конечношаговой модели биржевых торгов с вещественными ставками на случай использования указанного механизма торгов проведено впервые.
Получен результат о независимости значения соответствующей повторяющейся игры от параметра $\Co$.
\end{enumerate}

{\influence} Получено решение ряда повторяющихся игр с неполной информацией, моделирующих биржевые торги с более общим механизмом торгов, что позволяет оценить влияние конкретного вида механизма на оптимальное поведение агентов и результат торгов.

{\defpositions}
\begin{enumerate}
\item
  Получено решение повторяющейся антагонистической биржевой игры с двумя, а также со счетным числом состояний рынка и дискретными ставками.
\item
  Найдено решение конечношаговой игровой модели биржевых торгов с двумя состояниями рынка и непрерывными ставками.
  Разработан алгоритм численного построения оптимальных стратегий игроков.
\item
  Для дискретной модели дан явный вид ожидаемой продолжительности торгов при использовании игроками оптимальных стратегий.
  Для непрерывной модели показано, что динамика игрового взаимодействия не зависит от параметра механизма формирования сделки.
\end{enumerate}

{\reliability} полученных в работе результатов обусловлена строгостью формулировок задач и математических доказательств.
Результаты находятся в соответствии с результатами, полученными другими авторами.

{\probation} Основные результаты, полученные в диссертации, были представлены на ежегодных научных конференциях в МГУ им. М.В. Ломоносова <<Тихоновские чтения>> (2014), <<Ломоносовские чтения>> (2016) и на 8-ой Московской международной конференции по исследованию операций ORM 2016.

\ifnumequal{\value{bibliosel}}{0}{% Встроенная реализация с загрузкой файла через движок bibtex8
    \publications\ По теме диссертации имеется 7 публикаций \cite{pyanykh14, pyanykh16:discr:eng, pyanykh16:discr:ru, pyanykh16:cont, pyanykh:tikhon2014, pyanykh:lomonosov2016, pyanykh:orm2016}.
        Основные результаты диссертационной работы опубликованы в 3 статьях из перечня ВАК \cite{pyanykh14, pyanykh16:discr:ru, pyanykh16:cont}, 3 "--- в тезисах докладов \cite{pyanykh:tikhon2014, pyanykh:lomonosov2016, pyanykh:orm2016}.
}{% Реализация пакетом biblatex через движок biber
  % Сделана отдельная секция, чтобы не отображались в списке цитированных материалов
    \begin{refsection}%
        \printbibliography[heading=countauthornotvak, env=countauthornotvak, keyword=biblioauthornotvak, section=1]
        \printbibliography[heading=countauthorvak, env=countauthorvak, keyword=biblioauthorvak, section=1]
        \printbibliography[heading=countauthorconf, env=countauthorconf, keyword=biblioauthorconf, section=1]
        \printbibliography[heading=countauthor, env=countauthor, keyword=biblioauthor, section=1]
        \publications\ По теме диссертации имеется \arabic{citeauthor} публикаций. \nocite{pyanykh14, pyanykh16:discr:eng, pyanykh16:discr:ru, pyanykh16:cont, pyanykh:tikhon2014, pyanykh:lomonosov2016}
        Основные результаты диссертационной работы опубликованы в \arabic{citeauthorvak} статьях из перечня ВАК, \nocite{pyanykh14, pyanykh16:discr:ru, pyanykh16:cont}
        \arabic{citeauthorconf} "--- в тезисах докладов\nocite{pyanykh:tikhon2014, pyanykh:lomonosov2016, pyanykh:orm2016}.
    \end{refsection}
}

%%% Local Variables:
%%% mode: latex
%%% TeX-master: "../synopsis"
%%% End:
