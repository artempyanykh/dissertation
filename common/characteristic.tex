{\actuality} Повторяющиеся игры с неполной информацией представляют собой
естественную модель для анализа информационного аспекта в продолжительном
стратегическом взаимодействии агентов и позволяют ответить на вопросы о том, как
быстро происходит раскрытие приватной информации, каковы эффективные механизмы
ее сокрытия и какую выгоду из нее могут извлечь агенты. Начиная с работ Аумана,
Машлера и Стернса (см. \cite{aumann95}), теория повторяющихся игр с неполной
информацией получила дальнейшее развитие в работах Харта \todo{cite}, Сорина
\todo{cite}, Замира \todo{cite} и др.

Наиболее полное развитие получила теория повторяющихся антагонистических игр
двух лиц с неполной информацией у одной из сторон. В таких играх информационная
неопределенность моделируется введением множества возможных состояний природы
$S$. Перед началом игры ходом случая выбирается конкретное состояние $s \in S$ в
соответствии с некоторым вероятностным распределением, известным обоим игрокам.
При этом игрок 1 осведомлен о выбранном значении $s$, в то время как игрок 2
знает только то, что первый обладает приватной информацией. Анализ таких игр
является довольно сложной задачей, и аналитические решения получены лишь в
относительно небольшом числе случаев.

Одной из областей приложения данной теории является анализ поведения агентов на
финансовых рынках. Начиная с работы Башелье \cite{bachelier1900}, для описания
эволюции цен на активы используются винеровские процессы или "--- в дискретном
случае --- случайные блуждания. Возникновение случайных колебаний цен на рынке
принято объяснять влиянием не процесс ценообразования множества слабых
независимых внешних факторов. Однако гипотеза о полностью экзогенном
происхождении случайных колебаний цен не является удовлетворительной. Гипотеза
об эндогенном происхождении колебаний цен впервые была выдвинута в работе
\cite{kyle85}, однако броуновское движение по сути было введено в модель извне.
Более убедительно гипотеза о стратегическом происхождении случайных колебаний
цен была продемонстрирована в работе Де Мейер, Салей \cite{demeyer02}, где
винеровская компонента в эволюции цен возникает в следствие асимметричной
информированности агентов. В модели Де Мейер, Салей случайная ликвидная цена
акции может принимать два возможных значения $0$ и $1$ и определяется перед
началом торгов. Игрок 1 осведомлен о настоящем значении цены, игрок 2 знает
только ее вероятностное распределение. На каждом шаге торгов игроки одновременно
назначают цены за одну единицу рискового актива. Игрок, предложивший б\'{о}льшую
цену, покупает у второго актив по названной цене. В рамках данной модели игроки
могут делать произвольные вещественные ставки. Модели с дискретными ставками
впервые были рассмотрены в работах \cite{demeyer05, domansky07}. Ключевое
отличие дискретных моделей от непрерывных заключается в ограниченности значений
$n$-шаговых игр в силу меньшей стратегической свободы действий игрока 1.
Показано, что в модели с дискретными ставками ожидаемый момент раскрытия
приватной информации конечен, в то время как в модели с непрерывными ставками
раскрытие информации до последнего шага торгов стремиться к $0$ при стремлении
числа шагов к бесконечности.

В указанных работах использовался один и тот же механизм проведения сделки, а
именно "--- продажа по наибольшей цене. При этом известно, что выбор конкретного
механизма может существенным образом влиять на стратегическое поведение агентов
и на их выигрыши в результате взаимодействия. Как было отмечено авторами в
работе \cite{demeyer02}, предположительно, причиной возникновения броуновского
движения является полная симметрия между игроками (кроме информационной
асимметрии). В частности, анализ модели с более общим симметричным механизмом
проведения сделки отмечен как одно из дальнейших направлений исследования.

В работе \cite{sandomirskaya14} рассмотрена дискретная модель торгов с
механизмом в котором, игрок назначает цену продажи акции $p_b$, при этом цена
покупки определяется как $p_a = p_b - x$, т.н. механизм с фиксированным спрэдом $x$.

{\progress} В работе \cite{chatterjee83} рассмотрена модель двухстороннего
аукциона с неполной информацией, в котором цена сделки равна выпуклой комбинации
предложенных ставок с коэффициентом $\beta \in [0, 1]$, при этом в работе
\cite{myerson83} авторами показано, что при определенных условиях механизм со
значением $\beta = 1/2$ является оптимальным с точки зрения максимизации дохода.
В диссертации рассмотрены дискретные и непрерывные модели биржевых торгов с
симметричным механизмом проведения сделки, отвечающим механизму Чаттерджи и
Самуэльсона.

\aim\ Исследование повторяющихся игр с неполной информацией, моделирующих
биржевые торги с модифицированным механизмом проведения сделки.

\researchsubject\ Объектом исследования являются математические модели
механизмов взаимодействия агентов на финансовых рынках. Предметом исследования
являются повторяющиеся игры с неполной информацией, моделирующие биржевые торги
между двумя различно информированными агентами.

Для~достижения поставленной цели необходимо было решить следующие {\tasks}:
\begin{enumerate}
  \item Исследовать модель биржевых торгов с дискретными ставками и
    неограниченным количеством шагов. Определить влияние модифицированного
    механизма проведения сделки на поведение агентов и результат торгов.
  \item Исследовать модель биржевых торгов с непрерывными ставками в случаях
    конечного и бесконечного количества шагов. Сравнить оптимальное поведение
    агентов при использовании модифицированного механизма проведения сделки с
    результатами оригинальной модели.
  \item Обобщить результаты анализа модели с дискретными ставками на случай
    счетного множества возможных значений цены рискового актива \todo{и на
      случай торгов несколькими рисковыми активами}.
\end{enumerate}

\methods\ В диссертации применялись методы теории игр, выпуклого анализа, теории
двойственности и вариационного исчисления.

\novelty
\begin{enumerate}
\item Модель биржевых торгов с дискретными ставками и указанным механизмом
  проведения сделки исследуются впервые. Содержательно обобщение оптимальной
  стратегии инсайдера в главе 1.
\item Обобщение существующих результатов для конечношаговой модели биржевых
  торгов с вещественными ставками на случай использования указанного механизма
  проведения сделки получено впервые. Содержателен результат главы 2 о
  независимости значения соответствующей повторяющейся игры от конкретного вида
  механизма.
\end{enumerate}

\influence\ Получено решение ряда повторяющихся игр с неполной информацией,
моделирующих биржевые торги с более общим механизмом проведения сделки, что
позволяет оценить влияние конкретного вида механизма на оптимальное поведение
агентов и результат торгов.

\defpositions
\begin{enumerate}
  \item Решение бесконечной повторяющейся игры биржевых торгов с дискретными ставками и
    более общим механизмом проведения сделки.
  \item Решение конечношаговой повторяющейся игры биржевых торгов с непрерывными
    ставками и более общим механизмом проведения сделки.
  \item Обобщение результатов, полученных для модели с дискретными ставками, на
    случай счетного множества возможных цен рискового актива. 
\end{enumerate}

\reliability\ полученных в работе результатов обусловлена строгостью
формулировок задач и математических доказательств. Результаты находятся в
соответствии с результатами, полученными другими авторами.


\probation\ Основные результаты, полученные в диссертации, были представлены на
ежегодных научных конференциях в МГУ им. М.В. Ломоносова <<Тихоновские чтения>>
(2014) и <<Ломоносовские чтения>> (2016).

% \contribution\ Автор принимал активное участие \ldots

%\publications\ Основные результаты по теме диссертации изложены в ХХ печатных изданиях~\cite{Sokolov,Gaidaenko,Lermontov,Management},
%Х из которых изданы в журналах, рекомендованных ВАК~\cite{Sokolov,Gaidaenko}, 
%ХХ --- в тезисах докладов~\cite{Lermontov,Management}.

% \publications\ По теме диссертации имеется 7 публикаций~\cite{pyanykh14,
%   pyanykh16:discr:eng, pyanykh16:discr:ru, pyanykh16:cont, pyanykh16:countable,
%   pyanykh:tikhon2014, pyanykh:lomonosov2016}. %
% Основные результаты диссертационной работы опубликованы в 5 статьях из перечня
% ВАК~\cite{pyanykh14, pyanykh16:discr:eng, pyanykh16:discr:ru, pyanykh16:cont,
%   pyanykh16:countable}.

\ifthenelse{\equal{\thebibliosel}{0}}{% Встроенная реализация с загрузкой файла через движок bibtex8
    \publications\ Основные результаты по теме диссертации изложены в XX печатных изданиях, 
    X из которых изданы в журналах, рекомендованных ВАК, 
    X "--- в тезисах докладов.%
}{% Реализация пакетом biblatex через движок biber
Сделана отдельная секция, чтобы не отображались в списке цитированных материалов
    \begin{refsection}%
        \printbibliography[heading=countauthornotvak, env=countauthornotvak, keyword=biblioauthornotvak, section=1]%
        \printbibliography[heading=countauthorvak, env=countauthorvak, keyword=biblioauthorvak, section=1]%
        \printbibliography[heading=countauthorconf, env=countauthorconf, keyword=biblioauthorconf, section=1]%
        \printbibliography[heading=countauthor, env=countauthor, keyword=biblioauthor, section=1]%
        
        \publications\ По теме диссертации имеется \arabic{citeauthor}
        публикаций. %
        \nocite{pyanykh14, pyanykh16:discr:eng, pyanykh16:discr:ru,
          pyanykh16:cont, pyanykh16:countable, pyanykh:tikhon2014,
          pyanykh:lomonosov2016}%
        Основные результаты диссертационной работы опубликованы в
        \arabic{citeauthorvak} статьях из перечня ВАК, %
        \nocite{pyanykh14, pyanykh16:discr:eng, pyanykh16:discr:ru,
          pyanykh16:cont, pyanykh16:countable}%
        \arabic{citeauthorconf} "--- в тезисах
        докладов\nocite{pyanykh:tikhon2014, pyanykh:lomonosov2016}.
    \end{refsection}
}

%%% Local Variables:
%%% mode: latex
%%% TeX-master: "../dissertation"
%%% End:
