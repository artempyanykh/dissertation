%%% Основные сведения %%%
\newcommand{\thesisAuthor}             % Диссертация, ФИО автора
{%
    \texorpdfstring{% \texorpdfstring takes two arguments and uses the first for (La)TeX and the second for pdf
        Пьяных Артем Игоревич% так будет отображаться на титульном листе или в тексте, где будет использоваться переменная
    }{%
        Пьяных Артем Игоревич% эта запись для свойств pdf-файла. В таком виде, если pdf будет обработан программами для сбора библиографических сведений, будет правильно представлена фамилия.
    }%
}
\newcommand{\thesisAuthorShort}             % Диссертация, ФИО автора инициалами
{А.И.~Пьяных}

\newcommand{\thesisUdk}                % Диссертация, УДК
{519.83}
\newcommand{\thesisTitle}              % Диссертация, название
{\texorpdfstring{\MakeUppercase{Решение игровых задач для биржевых торгов с~обобщенным механизмом формирования сделки}}{Решение игровых задач для биржевых торгов с~обобщенным механизмом формирования сделки}}
\newcommand{\thesisSpecialtyNumber}    % Диссертация, специальность, номер
{\texorpdfstring{01.01.09}{01.01.09}}
\newcommand{\thesisSpecialtyTitle}     % Диссертация, специальность, название
{\texorpdfstring{дискретная математика и математическая кибернетика}{дискретная математика и математическая кибернетика}}
\newcommand{\thesisDegree}             % Диссертация, ученая степень
{кандидата физико-математических наук}
\newcommand{\thesisDegreeShort}        % Диссертация, ученая степень, краткая запись
{канд. физ.-мат. наук}
\newcommand{\thesisCity}               % Диссертация, город защиты
{Москва}
\newcommand{\thesisYear}               % Диссертация, год защиты
{2016}
\newcommand{\thesisOrganization}       % Диссертация, организация
{Московский государственный университет имени~М.В.Ломоносова}
\newcommand{\thesisOrganizationShort}  % Диссертация, краткое название организации для доклада
{МГУ им. М.В.Ломоносова}

\newcommand{\thesisInOrganization}       % Диссертация, организация в предложном падеже: Работа выполнена в ...
{кафедре исследования операций факультета вычислительной математики и кибернетики Федерального государственного бюджетного образовательного учреждения высшего образования <<Московский государственный университет имени М.В.Ломоносова>>}

\newcommand{\supervisorFio}            % Научный руководитель, ФИО
{Морозов Владимир Викторович}
\newcommand{\supervisorRegalia}        % Научный руководитель, регалии
{кандидат физико-математических наук, доцент}
\newcommand{\supervisorFioShort}            % Научный руководитель, ФИО
{В.В.~Морозов}
\newcommand{\supervisorRegaliaShort}        % Научный руководитель, регалии
{к.ф.-м.н.,~доц.}


\newcommand{\opponentOneFio}           % Оппонент 1, ФИО
{Кукушкин Николай Серафимович}
\newcommand{\opponentOneRegalia}       % Оппонент 1, регалии
{доктор физико-математических наук, ведущий научный сотрудник Вычислительного центра им. А.А. Дородницына Российской академии наук Федерального исследовательского центра <<Информатика и управление>> Российской академии наук}
\newcommand{\opponentOneJobPlace}      % Оппонент 1, место работы
{\todo{Не очень длинное название для места работы}}
\newcommand{\opponentOneJobPost}       % Оппонент 1, должность
{\todo{старший научный сотрудник}}

\newcommand{\opponentTwoFio}           % Оппонент 2, ФИО
{Сандомирская Марина Сергеевна}
\newcommand{\opponentTwoRegalia}       % Оппонент 2, регалии
{кандидат физико-математических наук, старший преподаватель департамента теоретической экономики факультета экономических наук, научный сотрудник лаборатории теории рынков и пространственной экономики Федерального государственного автономного образовательного учреждения высшего образования <<Национальный исследовательский университет <<Высшая школа экономики>>}
\newcommand{\opponentTwoJobPlace}      % Оппонент 2, место работы
{\todo{Основное место работы c длинным длинным длинным длинным названием}}
\newcommand{\opponentTwoJobPost}       % Оппонент 2, должность
{\todo{старший научный сотрудник}}

\newcommand{\leadingOrganizationTitle} % Ведущая организация, дополнительные строки
{Федеральное государственное бюджетное учреждение науки Санкт-Петербургский экономико-математический институт Российской академии наук}

\newcommand{\defenseDate}              % Защита, дата
{23 декабря 2016~г.~в~11.00}
\newcommand{\defenseCouncilNumber}     % Защита, номер диссертационного совета
{Д~501.001.44}
\newcommand{\defenseCouncilTitle}      % Защита, учреждение диссертационного совета
{Московском государственном университете имени М.В.Ломоносова}
\newcommand{\defenseCouncilAddress}    % Защита, адрес учреждение диссертационного совета
{119991, ГСП-1, Москва, Ленинские горы, 2-й учебный корпус, факультет ВМК, аудитория 685}
\newcommand{\defenseCouncilPhone}      % Телефон для справок
{\todo{+7~(0000)~00-00-00}}

\newcommand{\defenseSecretaryFio}      % Секретарь диссертационного совета, ФИО
{Шестаков О.~В.}
\newcommand{\defenseSecretaryRegalia}  % Секретарь диссертационного совета, регалии
{д.ф.-м.н., доцент}            % Для сокращений есть ГОСТы, например: ГОСТ Р 7.0.12-2011 + http://base.garant.ru/179724/#block_30000

\newcommand{\synopsisLibrary}          % Автореферат, название библиотеки
{\todo{Название библиотеки}}
% \newcommand{\synopsisDate}             % Автореферат, дата рассылки
% {\todo{<<__>> ________ YYYY года}}

% To avoid conflict with beamer class use \providecommand
\providecommand{\keywords}%                 % Ключевые слова для метаданных PDF диссертации и автореферата
{многошаговые игры, повторяющиеся игры, неполная информация, асимметричная информация, инсайдерская торговля}
