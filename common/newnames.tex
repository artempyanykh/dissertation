% Новые переменные, которые могут использоваться во всём проекте
% ГОСТ 7.0.11-2011
% 9.2 Оформление текста автореферата диссертации
% 9.2.1 Общая характеристика работы включает в себя следующие основные структурные
% элементы:
% актуальность темы исследования;
\newcommand{\actualityTXT}{Актуальность темы.}
% степень ее разработанности;
\newcommand{\progressTXT}{Степень разработанности темы.}
% цели и задачи;
\newcommand{\aimTXT}{Цель работы.}
\newcommand{\tasksTXT}{задачи}
\newcommand{\researchsubjectTXT}{Объект и предмет исследования.}
% научную новизну;
\newcommand{\noveltyTXT}{Научная новизна:}
% теоретическую и практическую значимость работы;
\newcommand{\influenceTXT}{Теоретическая и практическая значимость.}
% или чаще используют просто
% \newcommand{\influenceTXT}{Практическая значимость}
% методологию и методы исследования;
\newcommand{\methodsTXT}{Методы исследования.}
% положения, выносимые на защиту;
\newcommand{\defpositionsTXT}{Основные результаты, выносимые на~защиту:}
% степень достоверности и апробацию результатов.
\newcommand{\reliabilityTXT}{Достоверность}
\newcommand{\probationTXT}{Апробация работы.}

\newcommand{\contributionTXT}{Личный вклад.}
\newcommand{\publicationsTXT}{Публикации.}


\newcommand{\authorbibtitle}{Публикации автора по теме диссертации}
\newcommand{\fullbibtitle}{Список литературы} % (ГОСТ Р 7.0.11-2011, 4)

%%%%%%%%%%%%%%%%%%%%%%%%%%%%%%%%%%%%%%%%%%%%%%%%%%%%%%%%%%%%%%%%%%%%%%%

\newcommand{\overbar}[1]%
{\mkern 1.5mu\overline{\mkern-1.3mu#1\mkern-1.3mu}\mkern 1.3mu}

\DeclareMathOperator{\E}{\mathbb{E}}
\DeclareMathOperator{\D}{\mathbb{D}}
\newcommand*\di{\mathop{}\!\mathrm{d}}
\newcommand{\Ind}[1]{\ensuremath{\mathbf{1}_{#1}}}
\newcommand*\dom{\ensuremath{\mathrm{dom\;}} }
\newcommand*\range{\ensuremath{\mathrm{range\;}}}
\newcommand{\inter}{\mathrm{int}}

\newcommand{\eqdef}{\overset{\mathrm{def}}{=}}

\newcommand{\N}{\mathbb{N}}
\newcommand{\Z}{\mathbb{Z}}
\newcommand{\R}{\mathbb{R}}
\newcommand{\Borel}{\mathscr{B}}

\newcommand{\Co}{\ensuremath{\todo{\beta}}}
\newcommand{\DCo}{\ensuremath{\todo{{\overline{\beta}}}}}

\newcommand{\Tau}{\ensuremath{\mathrm{T}}}

\newcommand{\p}{\overbar{p}}

\newcolumntype{P}[1]{>{\centering\arraybackslash}p{#1}}

\newcommand{\ceil}[1]{\ensuremath{\lceil #1 \rceil}}
\newcommand{\floor}[1]{\ensuremath{\lfloor #1 \rfloor}}

%% Distributions
\newcommand{\pd}{\ensuremath{\todo{\overbar{p}}}} %%
\newcommand{\pc}[1]{\ensuremath{\todo{p(#1)}}} %% component of a distribution

% Posterior distributions
\newcommand{\pad}[1]{\ensuremath{\todo{\overbar{p}(#1)}}}
\newcommand{\padn}[2]{\ensuremath{\todo{\overbar{p}^{#1}(#2)}}}
\newcommand{\pac}[2]{\ensuremath{\todo{p(#1|#2)}}}
\newcommand{\pacn}[3]{\ensuremath{\todo{p^{#1}(#2|#3)}}}

% Full probabilities
\newcommand{\qd}{\ensuremath{\todo{q}}}
\newcommand{\qdn}[1]{\ensuremath{\todo{q^{#1}}}}
\newcommand{\qc}[1]{\ensuremath{\todo{q(#1)}}}
\newcommand{\qcn}[2]{\ensuremath{\todo{q^{#1}(#2)}}}

% First player strategies
\newcommand{\sigmar}{\ensuremath{\todo{\sigma}}} % in repeated game
\DeclareDocumentCommand{\sigmas}{O{} m}{\ensuremath{\todo{\sigma^{#1}_{#2}}}}
\DeclareDocumentCommand{\sigmasc}{O{} O{1} m}{\ensuremath{\todo{\sigma^{#1}_{#2,#3}}}}
\newcommand{\sigmara}[1]{\ensuremath{\todo{\sigma(#1)}}}
\newcommand{\sigmaran}[2]{\ensuremath{\todo{\sigma^{#1}(#2)}}}
\newcommand{\sigmak}{\ensuremath{\todo{\hat{\sigma_k}}}}

% Second player strategies
\newcommand{\taur}{\ensuremath{\todo{\tau}}}
\newcommand{\taus}[1]{\ensuremath{\todo{\tau_{#1}}}}
\newcommand{\tausc}[2][1]{\ensuremath{\todo{\tau_{#1,#2}}}}
\newcommand{\taura}[1]{\ensuremath{\todo{\tau(#1)}}}

%% Payoffs
\DeclareDocumentCommand{\as}{O{s} O{\Co}}{\ensuremath{\todo{a^{#1,#2}}}}

%% Game
\DeclareDocumentCommand{\theG}{s O{\Co} m}{%
\todo{%
\IfBooleanTF {#1}
{G^{m,#2}_{#3}}
{G^{#2}_{#3}}%
}%
}

%% Values
\DeclareDocumentCommand{\K}{s O{\Co} m}{%
\todo{%
\IfBooleanTF {#1}
{K^{m,#2}_{#3}}
{K^{#2}_{#3}}%
}%
}

\DeclareDocumentCommand{\V}{s O{\Co} m}{%
\todo{%
\IfBooleanTF {#1}
{V^{m,#2}_{#3}}
{V^{#2}_{#3}}%
}%
}

\DeclareDocumentCommand{\LowV}{s O{\Co} m}{%
\todo{%
\IfBooleanTF {#1}
{\underline{V}^{m,#2}_{#3}}
{\underline{V}^{#2}_{#3}}%
}%
}

\DeclareDocumentCommand{\HighV}{s O{\Co} m}{%
\todo{%
\IfBooleanTF {#1}
{\overbar{V}^{m,#2}_{#3}}
{\overbar{V}^{#2}_{#3}}%
}%
}

%% Guaranteed values

\DeclareDocumentCommand{\H}{s O{\Co} m}{%
\todo{%
\IfBooleanTF {#1}
{H^{m,#2}_{#3}}
{H^{#2}_{#3}}%
}%
}

\DeclareDocumentCommand{\L}{s O{\Co} m}{%
\todo{%
\IfBooleanTF {#1}
{L^{m,#2}_{#3}}
{L^{#2}_{#3}}%
}%
}

% $\sigmas{t}$

% $\sigmasc[s][t]{k}, \sigmasc{k}$

% $\sigmara{i}$

% $\tausc[t]{i}, \tausc{i}, \taura{i}$

% $\as, \as[H], \as[H][\DCo]$

% $\theG{n}, \theG*{n}, \theG[\DCo]{n}, \theG*[\DCo]{n}$

% $\K{n}, \K*{n}, \K[\DCo]{n}, \K*[\DCo]{n}$

% $\V{n}, \V*{n}, \V[\DCo]{n}, \V*[\DCo]{n}$

% $\LowV{n}, \LowV*{n}, \LowV[\DCo]{n}, \LowV*[\DCo]{n}$

% $\HighV{n}, \HighV*{n}, \HighV[\DCo]{n}, \HighV*[\DCo]{n}$

% $\H{n}, \H*{n}, \H[\DCo]{n}, \H*[\DCo]{n}$

% $\L{n}, \L*{n}, \L[\DCo]{n}, \L*[\DCo]{n}$