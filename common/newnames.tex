% Новые переменные, которые могут использоваться во всём проекте
% ГОСТ 7.0.11-2011
% 9.2 Оформление текста автореферата диссертации
% 9.2.1 Общая характеристика работы включает в себя следующие основные структурные
% элементы:
% актуальность темы исследования;
\newcommand{\actualityTXT}{Актуальность темы.}
% степень ее разработанности;
\newcommand{\progressTXT}{Степень разработанности темы.}
% цели и задачи;
\newcommand{\aimTXT}{Цель работы.}
\newcommand{\tasksTXT}{задачи}
\newcommand{\researchsubjectTXT}{Объект и предмет исследования.}
% научную новизну;
\newcommand{\noveltyTXT}{Научная новизна:}
% теоретическую и практическую значимость работы;
\newcommand{\influenceTXT}{Теоретическая и практическая значимость.}
% или чаще используют просто
% \newcommand{\influenceTXT}{Практическая значимость}
% методологию и методы исследования;
\newcommand{\methodsTXT}{Методы исследования.}
% положения, выносимые на защиту;
\newcommand{\defpositionsTXT}{Основные результаты, выносимые на~защиту:}
% степень достоверности и апробацию результатов.
\newcommand{\reliabilityTXT}{Достоверность}
\newcommand{\probationTXT}{Апробация работы.}

\newcommand{\contributionTXT}{Личный вклад.}
\newcommand{\publicationsTXT}{Публикации.}


\newcommand{\authorbibtitle}{Публикации автора по теме диссертации}
\newcommand{\fullbibtitle}{Список литературы} % (ГОСТ Р 7.0.11-2011, 4)

%%%%%%%%%%%%%%%%%%%%%%%%%%%%%%%%%%%%%%%%%%%%%%%%%%%%%%%%%%%%%%%%%%%%%%%

\newcommand{\overbar}[1]%
{\mkern 1.5mu\overline{\mkern-1.3mu#1\mkern-1.3mu}\mkern 1.3mu}

\DeclareMathOperator{\E}{\mathbb{E}}
\DeclareMathOperator{\D}{\mathbb{D}}
\newcommand*\di{\mathop{}\!\mathrm{d}}
\DeclareMathOperator*{\Ind}{\mathbbm{1}}

\newcommand{\Z}{\mathbb{Z}}
\newcommand{\R}{\mathbb{R}}

\newcommand{\Co}{\ensuremath{\beta}}
\newcommand{\DCo}{\ensuremath{\overline{\beta}}}

\newcommand{\Tau}{\ensuremath{\mathrm{T}}}

\newtheorem{lemma}{Лемма}
\newtheorem{theorem}{Теорема}
\newtheorem{proposition}{Утверждение}
\newtheorem{remark}{Замечание}

%%% Chapter 1
%%%
\newcommand{\generalGame}[2]{%
G_{#1}^{m, \beta}\left({#2}\right)%
}

\newcommand{\infiniteGame}[1]{%
G_{\infty}^{m, \beta}\left({#1}\right)%
}

\newcommand{\firstPlayerPayoff}[4]{%
K_{#1}^{m,\beta}\left({#2}, {#3}, {#4}\right)%
}

\newcommand{\infiniteFirstPlayerPayoff}[3]{%
K_{\infty}^{m,\beta}\left({#1}, {#2}, {#3}\right)%
}

\newcommand{\gameValue}[2]{%
V_{#1}^{m,\beta}\left({#2}\right)%
}

\newcommand{\infiniteGameValue}[1]{%
V_{\infty}^{m,\beta}\left({#1}\right)%
}

\newcommand{\symm}[1]{%
\overline{#1}%
}

\newcommand{\symmGameExpression}[2][\infty]{%
G_{#1}^{m, \alpha}\left({#2}\right)%
}

\newcommand{\upperBound}[1]{%
H_{\infty}^{m,\beta}\left({#1}\right)%
}

\newcommand{\fGeneral}[2][1]{%
\sigma_{#1}^{#2}%
}

\newcommand{\pEven}[1][k]{%
p^0_{#1}%
}

\newcommand{\pOdd}[1][k]{%
p^{\beta}_{#1}%
}

\newcommand{\fEven}[1][k]{%
\phi^0_{#1}%
}

\newcommand{\fOdd}[1][k]{%
\phi^{\beta}_{#1}%
}

\newcommand{\fOpt}{\sigma^*}

\newcommand{\lowerBound}[2][\infty]{%
L_{#1}^{m,\beta}\left({#2}\right)%
}