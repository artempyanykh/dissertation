\section*{Общая характеристика работы}

\newcommand{\actuality}{\underline{\textbf{\actualityTXT}}}
\newcommand{\progress}{\underline{\textbf{\progressTXT}}}
\newcommand{\aim}{\underline{{\textbf\aimTXT}}}
\newcommand{\tasks}{\underline{\textbf{\tasksTXT}}}
\newcommand{\researchsubject}{\underline{\textbf{\researchsubjectTXT}}}
\newcommand{\novelty}{\underline{\textbf{\noveltyTXT}}}
\newcommand{\influence}{\underline{\textbf{\influenceTXT}}}
\newcommand{\methods}{\underline{\textbf{\methodsTXT}}}
\newcommand{\defpositions}{\underline{\textbf{\defpositionsTXT}}}
\newcommand{\reliability}{\underline{\textbf{\reliabilityTXT}}}
\newcommand{\probation}{\underline{\textbf{\probationTXT}}}
\newcommand{\contribution}{\underline{\textbf{\contributionTXT}}}
\newcommand{\publications}{\underline{\textbf{\publicationsTXT}}}


{\actuality}
������������� ���� � �������� ����������� ������������ ����� ������������ ������ ��� ������� ��������������� ������� � ��������������� �������������� �������������� ������� � ��������� �������� �� ������� � ���, ��� ������ ���������� ��������� ��������� ����������, ������ ����������� ��������� �� �������� � ����� ������ �� ��� ����� ������� ������.

������ �������� ���� �������� � ������������ ������� ��������~\cite{harsanyi67}, ������, ������� � �������~\cite{aumann95}. 
�������� ����� ����������� ������������� ����������������� ���� ���� ��� � �������� ����������� � ����� �� ������. 
� ����� ����� �������������� ���������������� ������������ ��������� ��������� $S$ ��������� ��������� �������. 
����� ������� ���� ����� ������ ���������� ���������� ��������� $s \in S$ � ������������ � ��������� ������������� ��������������, ��������� ����� �������.
����� ���� ������ �� ���������� $n$ ����� ������ � ����, ��������������� ��������� $s$.
��� ���� ������ ����� ���������� � ��������� �������� $s$, � �� ����� ��� ������ ����� ������ ��, ��� ������ �������� ��������� �����������.

����� �� �������� ���������� ������ ������ �������� ������ ��������� ������� �� ���������� ������.
������� � ������ �������~\cite{bachelier1900}, ��� �������� �������� ��� �� ������ ������������ ����������� �������� ��� "--- � ���������� ������ --- ��������� ���������.
������������� ��������� ��������� ��� �� ����� ������� ��������� �������� �� ������� ��������������� ��������� ������ ����������� ������� ��������. 
������ �������� � ��������� ���������� ������������� ��������� ��������� ��� �� �������� ������������������. 
�������� �� �� �������������� ������������� ���� ������������������ � ������ ��~������ � �����~\cite{demeyer02}, ��� ����������� ���������� � �������� ��� ��������� �
��������� ������������� ����������������� �������. 
� ������ ������ ������ ��� ������ �� ���������� $n$ ����� ����� �������� ����������� ��������� ��������, ������ ���� �� ��� ����� ��������� ���� ������. 
�� ������ ���� ��� ������ ������������ ������, � �����, ������������ �\'{�}����� ������, �������� �
������� ����� �� ������������ ����; ��� ��������� ������ ������ �� ���������.
� ������ ������ � ��~������~\cite{demeyer05}, � ����� � ������ �.~�.~����������~\cite{domansky07} ���� ����������� ������ �������� ������ � ����������� ��������, ��� ���� ��������, ��� ������������������ ��� ������ �������� ������� ��������� ���������.

� ��������� ������� ������������� ���� � ��� �� �������� ������, � ������ "--- ������� �� ���������� ����.
��� ���� ��������, ��� ����� ����������� ��������� ����� ������������ ������� ������ �� �������������� ��������� ������� � �� �� �������� � ���������� ��������������.
��� ���� �������� �������� � ������~\cite{demeyer02}, ����������������, �������� ������������� ������������ �������� �������� ������ ��������� ����� �������� (����� �������������� ����������).
� ���������, ������ ������ � ����� ����� ������������ ���������� ������ ������� ��� ���� �� ���������� ����������� ������������.

� ������~\cite{demeyer10} ��~������� ����������� ����������� ������ � ���������� ����� ���������� ������.
�������� ��������� ������ ������ ����������� � ���, ��� � ����������� ������� �������� ��� �� �������� ����� �� ������� �� ����������� ���������, � ������� ������ �� ���������� ������������� ���� ������.
������, ����� �� �������, ������������� �� �������� ������, � ������ ������� ������� �������� ����������, �������� ������������� �������� ��������������� ��� �������� �����������������.

� ������ �.~C.~������������~\cite{sandomirskaya14} ����������� ���������� ������ ������ � ����������, � ������ �������� ����� ��������� ���� ������� ����� $p_b$, ��� ���� ���� ������� ������������ ��� $p_a = p_b + x$, ��� ���������� �������� � ������������� ������� $x$.
������ �������� ��������� � ������ ��������� � �����������~\cite{chatterjee83}.
��� ���� ����������� ������ �������������� �������� � �������� �����������, � ������� ���� ������ ����� �������� ���������� ������������ ������ � ������������� $\beta \in [0, 1]$.
��� ���� � ������ ��������� � �����������a~\cite{myerson83} ��������, ��� ��� ������������ �������� �������� �� ��������� $\beta = 1/2$ �������� ����������� � ����� ������ ������������ ������ �� ������.

{\progress} 
� ����������� ����������� ���������� � ����������� ������ �������� ������ � ������������ ���������� ������, ���������� ��������� ��������� � �����������, �.�. ������� ������ �� ����, ������ �������� ���������� ������������ ������.

{\aim} ������������ ������������� ��� � �������� �����������, ������������ �������� ����� � ���������������� ���������� ������.

{\researchsubject} �������� ������������ �������� �������������� ������ ���������� �������������� ������� �� ���������� ������.
��������� ������������ �������� ������������� ���� � �������� �����������, ������������ �������� ����� ����� ����� �������� ���������������� ��������.

���~���������� ������������ ���� ���������� ���� ������ ��������� {\tasks}:
\begin{enumerate}
\item 
����������� ������ �������� ������ � ����������� �������� � �������������� ����������� �����.
���������� ������� ����������������� ��������� ������ �� ��������� ������� � ��������� ������.
\item 
����������� ������ �������� ������ � ������������ �������� � ������� ��������� � ������������ ���������� �����.
�������� ����������� ��������� ������� ��� ������������� ����������������� ��������� ������ � ������������ ������������ ������.
\item 
�������� ���������� ������� ������ � ����������� �������� �� ������ ����� �� ������� ���������� ��������� �������� ���� ��������� ������ \todo{� �� ������ ������ ����������� ��������� ��������}.
\end{enumerate}

{\methods} � ����������� ����������� ������ ������ ���, ��������� �������, ������ �������������� � ������������� ����������.

{\novelty}
\begin{enumerate}
\item 
������ �������� ������ � ����������� �������� � ��������� ���������� ������ ����������� �������.
������� ����������� ��������� ���������, ������������� ������������ �� ������������.
\item 
������������ �������������� ������ �������� ������ � ������������� �������� �� ������ ������������� ���������� ��������� ������ ��������� �������.
������� ��������� � ������������� �������� ��������������� ������������� ���� �� ��������� $\Co$.
\end{enumerate}

{\influence} �������� ������� ���� ������������� ��� � �������� �����������, ������������ �������� ����� � ����� ����� ���������� ������, ��� ��������� ������� ������� ����������� ���� ��������� �� ����������� ��������� ������� � ��������� ������.

{\defpositions}
\begin{enumerate}
\item
������� ����������� ������������� ���� �������� ������ � ����������� �������� � ����� ����� ���������� ������.
\item
������� �������������� ������������� ���� �������� ������ � ������������ �������� � ����� ����� ���������� ������.
\item
��������� �����������, ���������� ��� ������ � ����������� ��������, �� ������ ����� �� ������� ���������� ��������� ��� ��������� ������.
\end{enumerate}

{\reliability} ���������� � ������ ����������� ����������� ���������� ������������ ����� � �������������� �������������.
���������� ��������� � ������������ � ������������, ����������� ������� ��������.

{\probation} �������� ����������, ���������� � �����������, ���� ������������ �� ��������� ������� ������������ � ��� ��. �.�. ���������� <<����������� ������>> (2014) � <<������������� ������>> (2016).

\ifthenelse{\equal{\thebibliosel}{0}}{%% ���������� ���������� � ��������� ����� ����� ������ bibtex8
    \publications\ �� ���� ����������� ������� 7 ���������� \cite{pyanykh14, pyanykh16:discr:eng, pyanykh16:discr:ru, pyanykh16:cont, pyanykh16:countable, pyanykh:tikhon2014, pyanykh:lomonosov2016}.
        �������� ���������� ��������������� ������ ������������ � 4 ������� �� ������� ��� \cite{pyanykh14, pyanykh16:discr:ru, pyanykh16:cont, pyanykh16:countable}, 2 "--- � ������� �������� \cite{pyanykh:tikhon2014, pyanykh:lomonosov2016}.
  }{% ���������� ������� biblatex ����� ������ biber
    \begin{refsection}%
        \printbibliography[heading=countauthornotvak, env=countauthornotvak, keyword=biblioauthornotvak, section=1]
        \printbibliography[heading=countauthorvak, env=countauthorvak, keyword=biblioauthorvak, section=1]
        \printbibliography[heading=countauthorconf, env=countauthorconf, keyword=biblioauthorconf, section=1]
        \printbibliography[heading=countauthor, env=countauthor, keyword=biblioauthor, section=1]
        \publications\ �� ���� ����������� ������� \arabic{citeauthor} ����������. \nocite{pyanykh14, pyanykh16:discr:eng, pyanykh16:discr:ru, pyanykh16:cont, pyanykh16:countable, pyanykh:tikhon2014, pyanykh:lomonosov2016}
        �������� ���������� ��������������� ������ ������������ � \arabic{citeauthorvak} ������� �� ������� ���, \nocite{pyanykh14, pyanykh16:discr:ru, pyanykh16:cont, pyanykh16:countable}
        \arabic{citeauthorconf} "--- � ������� ��������\nocite{pyanykh:tikhon2014, pyanykh:lomonosov2016}.
    \end{refsection}
}

%%% Local Variables:
%%% coding: cp1251
%%% mode: latex
%%% TeX-master: "../dissertation"
%%% End:
 % Характеристика работы по структуре во введении и в автореферате не отличается (ГОСТ Р 7.0.11, пункты 5.3.1 и 9.2.1), потому её загружаем из одного и того же внешнего файла, предварительно задав форму выделения некоторым параметрам

%Диссертационная работа была выполнена при поддержке грантов ...

%\underline{\textbf{Объем и структура работы.}} Диссертация состоит из~введения, четырех глав, заключения и~приложения. Полный объем диссертации \textbf{ХХХ}~страниц текста с~\textbf{ХХ}~рисунками и~5~таблицами. Список литературы содержит \textbf{ХХX}~наименование.

%\newpage
\section*{Содержание работы}
Во \underline{\textbf{введении}} обосновывается актуальность исследований, проводимых в рамках данной диссертационной работы, приводится обзор научной литературы по изучаемой проблеме, формулируется цель, ставятся задачи работы, сформулированы научная новизна и практическая значимость представляемой работы.

\underline{\textbf{Первая глава}} посвящена ....

 картинку можно добавить так:
\begin{figure}[ht] 
  \center
  \includegraphics [scale=0.27] {latex}
  \caption{Подпись к картинке.} 
  \label{img:latex}
\end{figure}

Формулы в строку без номера добавляются так:
\[ 
  \lambda_{T_s} = K_x\frac{d{x}}{d{T_s}}, \qquad
  \lambda_{q_s} = K_x\frac{d{x}}{d{q_s}},
\]

\underline{\textbf{Вторая глава}} посвящена исследованию 

\underline{\textbf{Третья глава}} посвящена исследованию 

В \underline{\textbf{четвертой главе}} приведено описание 

В \underline{\textbf{заключении}} приведены основные результаты работы, которые заключаются в следующем:
%% �������� ���� � 7.0.11-2011:
%% 5.3.3 � ���������� ����������� �������� ����� ������������ ������������, ������������, ����������� ���������� ���������� ����.
%% 9.2.3 � ���������� ������������ ����������� �������� ����� ������� ������������, ������������ � ����������� ���������� ���������� ����.
\begin{enumerate}
  \item �� ������ ������� \ldots
  \item ��������� ������������ ��������, ��� \ldots
  \item �������������� ������������� �������� \ldots
  \item ��� ���������� ������������ ����� ��� ������ \ldots
\end{enumerate}

%%% Local variables:
%%% coding: cp1251
%%% End:



%\newpage
При использовании пакета \verb!biblatex! список публикаций автора по теме
диссертации формируется в разделе <<\publications>>\ файла
\verb!../common/characteristic.tex!  при помощи команды \verb!\nocite! 

\ifdefmacro{\microtypesetup}{\microtypesetup{protrusion=false}}{} % не рекомендуется применять пакет микротипографики к автоматически генерируемому списку литературы
\ifnumequal{\value{bibliosel}}{0}{% Встроенная реализация с загрузкой файла через движок bibtex8
  \renewcommand{\refname}{\large \authorbibtitle}
  \nocite{*}
  \insertbiblioauthor                          % Подключаем Bib-базы
  %\insertbiblioother   % !!! bibtex не умеет работать с несколькими библиографиями !!!
}{% Реализация пакетом biblatex через движок biber
  \insertbiblioauthor                          % Подключаем Bib-базы
  \insertbiblioother
}
\ifdefmacro{\microtypesetup}{\microtypesetup{protrusion=true}}{}


%%% Local variables:
%%% mode: latex
%%% TeX-master: "../synopsis"
%%% End:
