
\section*{Общая характеристика работы}

\newcommand{\emphasis}[1]{\underline{\textbf{{#1}}}}
\newcommand{\actuality}{\emphasis{\actualityTXT}}
\newcommand{\progress}{\emphasis{\progressTXT}}
\newcommand{\aim}{\emphasis{\aimTXT}}
\newcommand{\tasks}{\emphasis{\tasksTXT}}
\newcommand{\researchsubject}{\emphasis{\researchsubjectTXT}}
\newcommand{\novelty}{\emphasis{\noveltyTXT}}
\newcommand{\influence}{\emphasis{\influenceTXT}}
\newcommand{\methods}{\emphasis{\methodsTXT}}
\newcommand{\defpositions}{\emphasis{\defpositionsTXT}}
\newcommand{\reliability}{\emphasis{\reliabilityTXT}}
\newcommand{\probation}{\emphasis{\probationTXT}}
\newcommand{\contribution}{\emphasis{\contributionTXT}}
\newcommand{\publications}{\emphasis{\publicationsTXT}}


{\actuality}
������������� ���� � �������� ����������� ������������ ����� ������������ ������ ��� ������� ��������������� ������� � ��������������� �������������� �������������� ������� � ��������� �������� �� ������� � ���, ��� ������ ���������� ��������� ��������� ����������, ������ ����������� ��������� �� �������� � ����� ������ �� ��� ����� ������� ������.

������ �������� ���� �������� � ������������ ������� ��������~\cite{harsanyi67}, ������, ������� � �������~\cite{aumann95}. 
�������� ����� ����������� ������������� ����������������� ���� ���� ��� � �������� ����������� � ����� �� ������. 
� ����� ����� �������������� ���������������� ������������ ��������� ��������� $S$ ��������� ��������� �������. 
����� ������� ���� ����� ������ ���������� ���������� ��������� $s \in S$ � ������������ � ��������� ������������� ��������������, ��������� ����� �������.
����� ���� ������ �� ���������� $n$ ����� ������ � ����, ��������������� ��������� $s$.
��� ���� ������ ����� ���������� � ��������� �������� $s$, � �� ����� ��� ������ ����� ������ ��, ��� ������ �������� ��������� �����������.

����� �� �������� ���������� ������ ������ �������� ������ ��������� ������� �� ���������� ������.
������� � ������ �������~\cite{bachelier1900}, ��� �������� �������� ��� �� ������ ������������ ����������� �������� ��� "--- � ���������� ������ --- ��������� ���������.
������������� ��������� ��������� ��� �� ����� ������� ��������� �������� �� ������� ��������������� ��������� ������ ����������� ������� ��������. 
������ �������� � ��������� ���������� ������������� ��������� ��������� ��� �� �������� ������������������. 
�������� �� �� �������������� ������������� ���� ������������������ � ������ ��~������ � �����~\cite{demeyer02}, ��� ����������� ���������� � �������� ��� ��������� �
��������� ������������� ����������������� �������. 
� ������ ������ ������ ��� ������ �� ���������� $n$ ����� ����� �������� ����������� ��������� ��������, ������ ���� �� ��� ����� ��������� ���� ������. 
�� ������ ���� ��� ������ ������������ ������, � �����, ������������ �\'{�}����� ������, �������� �
������� ����� �� ������������ ����; ��� ��������� ������ ������ �� ���������.
� ������ ������ � ��~������~\cite{demeyer05}, � ����� � ������ �.~�.~����������~\cite{domansky07} ���� ����������� ������ �������� ������ � ����������� ��������, ��� ���� ��������, ��� ������������������ ��� ������ �������� ������� ��������� ���������.

� ��������� ������� ������������� ���� � ��� �� �������� ������, � ������ "--- ������� �� ���������� ����.
��� ���� ��������, ��� ����� ����������� ��������� ����� ������������ ������� ������ �� �������������� ��������� ������� � �� �� �������� � ���������� ��������������.
��� ���� �������� �������� � ������~\cite{demeyer02}, ����������������, �������� ������������� ������������ �������� �������� ������ ��������� ����� �������� (����� �������������� ����������).
� ���������, ������ ������ � ����� ����� ������������ ���������� ������ ������� ��� ���� �� ���������� ����������� ������������.

� ������~\cite{demeyer10} ��~������� ����������� ����������� ������ � ���������� ����� ���������� ������.
�������� ��������� ������ ������ ����������� � ���, ��� � ����������� ������� �������� ��� �� �������� ����� �� ������� �� ����������� ���������, � ������� ������ �� ���������� ������������� ���� ������.
������, ����� �� �������, ������������� �� �������� ������, � ������ ������� ������� �������� ����������, �������� ������������� �������� ��������������� ��� �������� �����������������.

� ������ �.~C.~������������~\cite{sandomirskaya14} ����������� ���������� ������ ������ � ����������, � ������ �������� ����� ��������� ���� ������� ����� $p_b$, ��� ���� ���� ������� ������������ ��� $p_a = p_b + x$, ��� ���������� �������� � ������������� ������� $x$.
������ �������� ��������� � ������ ��������� � �����������~\cite{chatterjee83}.
��� ���� ����������� ������ �������������� �������� � �������� �����������, � ������� ���� ������ ����� �������� ���������� ������������ ������ � ������������� $\beta \in [0, 1]$.
��� ���� � ������ ��������� � �����������a~\cite{myerson83} ��������, ��� ��� ������������ �������� �������� �� ��������� $\beta = 1/2$ �������� ����������� � ����� ������ ������������ ������ �� ������.

{\progress} 
� ����������� ����������� ���������� � ����������� ������ �������� ������ � ������������ ���������� ������, ���������� ��������� ��������� � �����������, �.�. ������� ������ �� ����, ������ �������� ���������� ������������ ������.

{\aim} ������������ ������������� ��� � �������� �����������, ������������ �������� ����� � ���������������� ���������� ������.

{\researchsubject} �������� ������������ �������� �������������� ������ ���������� �������������� ������� �� ���������� ������.
��������� ������������ �������� ������������� ���� � �������� �����������, ������������ �������� ����� ����� ����� �������� ���������������� ��������.

���~���������� ������������ ���� ���������� ���� ������ ��������� {\tasks}:
\begin{enumerate}
\item 
����������� ������ �������� ������ � ����������� �������� � �������������� ����������� �����.
���������� ������� ����������������� ��������� ������ �� ��������� ������� � ��������� ������.
\item 
����������� ������ �������� ������ � ������������ �������� � ������� ��������� � ������������ ���������� �����.
�������� ����������� ��������� ������� ��� ������������� ����������������� ��������� ������ � ������������ ������������ ������.
\item 
�������� ���������� ������� ������ � ����������� �������� �� ������ ����� �� ������� ���������� ��������� �������� ���� ��������� ������ \todo{� �� ������ ������ ����������� ��������� ��������}.
\end{enumerate}

{\methods} � ����������� ����������� ������ ������ ���, ��������� �������, ������ �������������� � ������������� ����������.

{\novelty}
\begin{enumerate}
\item 
������ �������� ������ � ����������� �������� � ��������� ���������� ������ ����������� �������.
������� ����������� ��������� ���������, ������������� ������������ �� ������������.
\item 
������������ �������������� ������ �������� ������ � ������������� �������� �� ������ ������������� ���������� ��������� ������ ��������� �������.
������� ��������� � ������������� �������� ��������������� ������������� ���� �� ��������� $\Co$.
\end{enumerate}

{\influence} �������� ������� ���� ������������� ��� � �������� �����������, ������������ �������� ����� � ����� ����� ���������� ������, ��� ��������� ������� ������� ����������� ���� ��������� �� ����������� ��������� ������� � ��������� ������.

{\defpositions}
\begin{enumerate}
\item
������� ����������� ������������� ���� �������� ������ � ����������� �������� � ����� ����� ���������� ������.
\item
������� �������������� ������������� ���� �������� ������ � ������������ �������� � ����� ����� ���������� ������.
\item
��������� �����������, ���������� ��� ������ � ����������� ��������, �� ������ ����� �� ������� ���������� ��������� ��� ��������� ������.
\end{enumerate}

{\reliability} ���������� � ������ ����������� ����������� ���������� ������������ ����� � �������������� �������������.
���������� ��������� � ������������ � ������������, ����������� ������� ��������.

{\probation} �������� ����������, ���������� � �����������, ���� ������������ �� ��������� ������� ������������ � ��� ��. �.�. ���������� <<����������� ������>> (2014) � <<������������� ������>> (2016).

\ifthenelse{\equal{\thebibliosel}{0}}{%% ���������� ���������� � ��������� ����� ����� ������ bibtex8
    \publications\ �� ���� ����������� ������� 7 ���������� \cite{pyanykh14, pyanykh16:discr:eng, pyanykh16:discr:ru, pyanykh16:cont, pyanykh16:countable, pyanykh:tikhon2014, pyanykh:lomonosov2016}.
        �������� ���������� ��������������� ������ ������������ � 4 ������� �� ������� ��� \cite{pyanykh14, pyanykh16:discr:ru, pyanykh16:cont, pyanykh16:countable}, 2 "--- � ������� �������� \cite{pyanykh:tikhon2014, pyanykh:lomonosov2016}.
  }{% ���������� ������� biblatex ����� ������ biber
    \begin{refsection}%
        \printbibliography[heading=countauthornotvak, env=countauthornotvak, keyword=biblioauthornotvak, section=1]
        \printbibliography[heading=countauthorvak, env=countauthorvak, keyword=biblioauthorvak, section=1]
        \printbibliography[heading=countauthorconf, env=countauthorconf, keyword=biblioauthorconf, section=1]
        \printbibliography[heading=countauthor, env=countauthor, keyword=biblioauthor, section=1]
        \publications\ �� ���� ����������� ������� \arabic{citeauthor} ����������. \nocite{pyanykh14, pyanykh16:discr:eng, pyanykh16:discr:ru, pyanykh16:cont, pyanykh16:countable, pyanykh:tikhon2014, pyanykh:lomonosov2016}
        �������� ���������� ��������������� ������ ������������ � \arabic{citeauthorvak} ������� �� ������� ���, \nocite{pyanykh14, pyanykh16:discr:ru, pyanykh16:cont, pyanykh16:countable}
        \arabic{citeauthorconf} "--- � ������� ��������\nocite{pyanykh:tikhon2014, pyanykh:lomonosov2016}.
    \end{refsection}
}

%%% Local Variables:
%%% coding: cp1251
%%% mode: latex
%%% TeX-master: "../dissertation"
%%% End:
 % Характеристика работы по структуре во введении и в автореферате не отличается (ГОСТ Р 7.0.11, пункты 5.3.1 и 9.2.1), потому её загружаем из одного и того же внешнего файла, предварительно задав форму выделения некоторым параметрам

%Диссертационная работа была выполнена при поддержке грантов ...

\emphasis{Объем и структура работы.}
Диссертация состоит из~введения, трех глав и заключения.
Полный объем диссертации составляет \todo{\textbf{ХХХ}}~страниц текста с~\todo{\textbf{ХХ}}~рисунками и~\todo{XXX}~таблицами.
Список литературы содержит \todo{\textbf{ХХX}}~наименование.

%\newpage
\section*{Содержание работы}
Во \emphasis{введении} обосновывается актуальность исследований, проводимых в рамках данной диссертационной работы, приводится обзор научной литературы по изучаемой проблеме, формулируется цель, ставятся задачи работы, сформулированы научная новизна, теоретическая и практическая значимость представляемой работы, а также результаты, выносимые на защиту.

% * Chapter 1
\emphasis{Первая глава} посвящена исследованию теоретико-игровой модели биржевых торгов с дискретными ставками и двумя состояниями.

% ** Основные понятия и введение в тему
В \emphasis{разделе 1.1} диссертации дано введение в теорию повторяющихся игр с неполной информацией, определены основные термины и понятия.
Теоретико-игровые постановки задач из глав~2 и~3 будут отличаться от данного классического описания, о чем будет сказано отдельно.

Рассмотрим антагонистическую игру двух лиц, которая повторяется $n$ раз, где $n\leq \infty$.
Будем считать, что первый игрок знает функцию выигрыша в данной игре, в то время как второй игрок такой информацией не обладает.
Однако второй игрок знает, что настоящая функция выигрыша является одной из $\kappa$ возможных альтернатив.
Каждой такой альтернативе второй игрок приписывает некоторую вероятность того, что данная функция выигрыша является истинной функцией выигрыша в рассматриваемой
игре.
Таким образом, априорные убеждения второго игрока задаются вероятностным вектором
$\p = (p_1, p_2, \ldots, p_\kappa),\ \sum_{i=1}^\kappa p_i = 1$.
С данными функциями выигрыша можно связать игры $G_1, G_2, \ldots, G_\kappa$.
В дальнейшем мы будем считать, что информационная неопределенность второго игрока заключается именно в том, что он не знает какая из игр $G_1, G_2, \ldots, G_\kappa$ разыгрывается.

На каждом шаге игры первый игрок может совершать действия из множества $I$, второй игрок --- действия из множества $J$, при этом мы считаем, что множества действий одного игрока известно другому.
Кроме того, положим, что второй игрок знает, что первый обладает точной информацией о том, какая именно игра разыгрывается, а первый игрок знает априорные убеждения второго.

Игры $G_1, G_2, \ldots, G_\kappa$ будем называть \emph{одношаговыми играми}.
Все одношаговые игры описываются матрицами размера $|I| \times |J|$, где элементы матрицы задают выплаты первому игроку.
Мы предполагаем, что оба игрока точно знают платежные матрицы игр $G_1, G_2, \ldots, G_\kappa$.
На каждом шаге игры первый игрок выбирает номер строки, и одновременно с ним второй игрок выбирает номер столбца.
В конце каждого хода действия игроков оглашаются, и элемент из матрицы, отвечающей настоящей игре, прибавляется к выигрышу первого игрока и вычитается из выигрыша второго.
Таким образом, первый игрок знает свой выигрыш на каждом этапе игры, в то время как второй может лишь рассчитать свой ожидаемый выигрыш.
В завершение, мы считаем, что данное описание известно обоим игрокам.

Данная игра с неполной информацией описывается в виде игры в нормальной форме следующим образом.
Обозначим через $S = \{1, 2, \ldots, \kappa\}$ множество возможных альтернатив или \emph{состояний природы}.
Перед началом игры ходом случая в соответствии с вероятностным распределением $\p$ выбирается состояние $s \in S$.
Далее на протяжении $n$ шагов разыгрывается игра $G_s$.
Первый игрок информирован о результате хода случая, второй игрок "--- нет.
В остальном правила данной игры совпадают с описанными выше.

Пусть $h_t = \left((i_1, j_1), (i_2, j_2), \ldots, (i_t, j_t)\right)$ "--- история ходов после завершения шага $t$.
Множество все таких $h_t$ обозначим через $H_t$. 

Стратегией первого игрока в такой игре является последовательность ходов (отображений) $\sigma = (\sigma_1, \sigma_2, \ldots, \sigma_n)$, где $\sigma_t = (\sigma^1_t, \sigma^2_t, \ldots, \sigma^\kappa_t)$, и $\sigma^s_t: H_{t-1} \rightarrow \Delta(I)$ "--- смешанная стратегия, зависящая от предыдущих ходов, которую первый игрок использует если ходом случая реализовалось состояние $s$.

Аналогичным образом, определим стратегию второго игрока как последовательность ходов (отображений) $\tau = (\tau_1, \tau_2, \ldots, \tau_n)$, где $\tau_t: H_{t-1} \rightarrow \Delta(I)$ "--- смешанная стратегия, зависящая от предыдущих ходов.
Как видно, ход второго игрока на каждом шаге игры зависит только от предыдущих ходов, и не зависит от состояния, в силу того, что второй игрок не информирован о результате хода случая.

Отметим, что, как показано в монографии Аумана, Машлера~\cite{aumann95}, достаточно рассматривать только стратегии, которые зависят лишь от предыдущих ходов первого игрока и не зависят от ходов второго.
%
Также нужно отметить, что в диссертации рассмотрены игры, в которых выигрыш равен суммарным выплатам, в отличие от постановки из \cite{aumann95}, в которых рассматривались игры с усредненными выплатами.

% ** Описание дискретной модели
В \emphasis{разделе 1.2} приводится описание дискретной модели биржевых торгов.
Следуя работе~\cite{domansky07}, Рассматривается упрощенная модель финансового рынка, на котором два игрока ведут торговлю однотипными акциями на протяжении $n \leqslant \infty$ шагов.

Перед началом торгов случайный ход определяет цену акции на весь период торгов, которая может быть либо $m \in \N$ с вероятностью $p$, либо $0$ с вероятностью $1-p$.
Таким образом определенный ход случая является упрощенным аналогом некоторого шокового события на финансовом рынке.
Такого, например, как публикация отчетов о доходах некоторой компании.
Выбранная цена сообщается первому игроку и не сообщается второму, при этом второй игрок знает, что первый "--- инсайдер.

Рассмотрим $t$-й шаг торгов, где $t = \overline{1,n}$.
На данном шаге первый игрок выбирает ставку $i_t \in I = \{0, 1, \ldots, m\}$, а второй --- ставку $j_t \in J = \{0, 1, \ldots, m\}$.
Игрок предложивший б\'{о}льшую ставку покупает у другого акцию по цене равной
\[
  \Co \max(i_t, j_t) + \DCo \min(i_t, j_t),
\]
где $\Co \in (0, 1),\ \DCo = 1 - \Co$.
Если ставки равны, то сделка на $t$-м шаге не состоится.
Коэффициент $\Co$ можно интерпретировать как \emph{переговорную силу продавца} "--- чем ближе значение к $1$, тем большую сумму получит продавец акции в результате сделки.

Считаем, что игроки обладают неограниченными запасами рисковых и безрисковых активов, т.е. торги не могут прекратиться по причине того, что у одного из игроков закончатся деньги или акции.
Цель игроков состоит в максимизации стоимости итогового портфеля, состоящего из некоторого числа купленных акций и суммы денег, полученных в результате торгов.
Таким образом, не ограничивая общности, можно положить, что в начальный момент времени оба игрока имеют нулевые портфели.

% * Вторая глава
\emphasis{Вторая глава} посвящена исследованию 

% * Третья глава
\emphasis{Третья глава} посвящена исследованию 

% * Conclusion
В \emphasis{заключении} приведены основные результаты работы, которые заключаются в следующем:
%% �������� ���� � 7.0.11-2011:
%% 5.3.3 � ���������� ����������� �������� ����� ������������ ������������, ������������, ����������� ���������� ���������� ����.
%% 9.2.3 � ���������� ������������ ����������� �������� ����� ������� ������������, ������������ � ����������� ���������� ���������� ����.
\begin{enumerate}
  \item �� ������ ������� \ldots
  \item ��������� ������������ ��������, ��� \ldots
  \item �������������� ������������� �������� \ldots
  \item ��� ���������� ������������ ����� ��� ������ \ldots
\end{enumerate}

%%% Local variables:
%%% coding: cp1251
%%% End:


%\newpage
% При использовании пакета \verb!biblatex! список публикаций автора по теме
% диссертации формируется в разделе <<\publications>>\ файла
% \verb!../common/characteristic.tex!  при помощи команды \verb!\nocite! 

\ifdefmacro{\microtypesetup}{\microtypesetup{protrusion=false}}{} % не рекомендуется применять пакет микротипографики к автоматически генерируемому списку литературы
\ifnumequal{\value{bibliosel}}{0}{% Встроенная реализация с загрузкой файла через движок bibtex8
  \renewcommand{\refname}{\large \authorbibtitle}
  \nocite{*}
  \insertbiblioauthor                          % Подключаем Bib-базы
  %\insertbiblioother   % !!! bibtex не умеет работать с несколькими библиографиями !!!
}{% Реализация пакетом biblatex через движок biber
  \insertbiblioauthor                          % Подключаем Bib-базы
  \insertbiblioother
}
\ifdefmacro{\microtypesetup}{\microtypesetup{protrusion=true}}{}


%%% Local variables:
%%% mode: latex
%%% TeX-master: "../synopsis"
%%% End:
