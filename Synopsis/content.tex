
\section*{Общая характеристика работы}

\newcommand{\emphasis}[1]{\underline{\textbf{{#1}}}}
\newcommand{\actuality}{\emphasis{\actualityTXT}}
\newcommand{\progress}{\emphasis{\progressTXT}}
\newcommand{\aim}{\emphasis{\aimTXT}}
\newcommand{\tasks}{\emphasis{\tasksTXT}}
\newcommand{\researchsubject}{\emphasis{\researchsubjectTXT}}
\newcommand{\novelty}{\emphasis{\noveltyTXT}}
\newcommand{\influence}{\emphasis{\influenceTXT}}
\newcommand{\methods}{\emphasis{\methodsTXT}}
\newcommand{\defpositions}{\emphasis{\defpositionsTXT}}
\newcommand{\reliability}{\emphasis{\reliabilityTXT}}
\newcommand{\probation}{\emphasis{\probationTXT}}
\newcommand{\contribution}{\emphasis{\contributionTXT}}
\newcommand{\publications}{\emphasis{\publicationsTXT}}

{\actuality} Повторяющиеся игры с неполной информацией представляют собой
естественную модель для анализа информационного аспекта в продолжительном
стратегическом взаимодействии агентов и позволяют ответить на вопросы о том, как
быстро происходит раскрытие приватной информации, каковы эффективные механизмы
ее сокрытия и какую выгоду из нее могут извлечь агенты.

Теория получила свое рождение в классических работах Харсаньи~\cite{harsanyi67},
Аумана, Машлера и Стернса~\cite{aumann95}. Наиболее полно исследованы
повторяющиеся антагонистические игры двух лиц с неполной информацией у одной из
сторон. В таких играх информационная неопределенность моделируется введением
множества $S$ возможных состояний природы. Перед началом игры ходом случая
выбирается конкретное состояние $s \in S$ в соответствии с некоторым
вероятностным распределением, известным обоим игрокам, после чего игроки на
протяжении $n$ шагов играют в игру, соответствующую состоянию $s$. При этом
первый игрок осведомлен о выбранном значении $s$, в то время как второй знает
только то, что первый обладает приватной информацией. В силу продолжительности
взаимодействия использование первым игроком стратегии, максимизирующей его
одношаговый выигрыш, повлечет немедленное раскрытие приватной информации. Это
вынуждает первого игрока к стратегическому маневрированию, проявляющемуся в
рандомизации своих действий.

Одной из областей приложения данной теории является анализ поведения агентов на
финансовых рынках. Начиная с работы Башелье \cite{bachelier1900}, для описания
эволюции цен на активы используются винеровские процессы или "--- в дискретном
случае --- случайные блуждания. Возникновение случайных колебаний цен на рынке
принято объяснять влиянием на процесс ценообразования множества слабых
независимых внешних факторов. Однако гипотеза о полностью экзогенном
происхождении случайных колебаний цен не является удовлетворительной. Гипотеза
об их стратегическом происхождении была продемонстрирована в работе Де Мейера и
Салей \cite{demeyer02}, где винеровская компонента в эволюции цен возникает в
следствие асимметричной информированности агентов. В рамках данной модели два
игрока на протяжении $n$ шагов ведут торговлю однотипными рисковыми активами,
причем один из них знает настоящую цену актива. На каждом шаге они делают
вещественные ставки, и игрок, предложивший б\'{о}льшую ставку, покупает у
другого актив по предложенной цене; при равенстве ставок сделка не состоится. В
работе Марино и Де Мейера \cite{demeyer05}, а также в работе Доманского
\cite{domansky07} была исследована модель биржевых торгов с дискретными
ставками, где было показано, что последовательность цен актива образует простое
случайное блуждание. 

% Ключевое отличие дискретных моделей от непрерывных заключается в
% ограниченности значений $n$-шаговых игр в силу меньшей стратегической свободы
% первого игрока. Показано, что в модели с дискретными ставками ожидаемый момент
% раскрытия приватной информации конечен, в то время как в модели с непрерывными
% ставками раскрытие информации до последнего шага торгов стремиться к $0$ при
% стремлении числа шагов к бесконечности.

В указанных работах использовался один и тот же механизм торгов, а именно "---
продажа по наибольшей цене. При этом известно, что выбор конкретного механизма
может существенным образом влиять на стратегическое поведение агентов и на их
выигрыши в результате взаимодействия. Как было отмечено авторами в
работе~\cite{demeyer02}, предположительно, причиной возникновения броуновского
движения является полная симметрия между игроками (кроме информационной
асимметрии). В частности, анализ модели с более общим симметричным механизмом
торгов отмечен как одно из дальнейших направлений исследования.

В работе Сандомирской~\cite{sandomirskaya14} рассмотрена дискретная модель
торгов с механизмом, в рамках которого игрок назначает цену продажи акции $p_b$,
при этом цена покупки определяется как $p_a = p_b - x$, т.н. механизм с
фиксированным спрэдом $x$. Другой механизм предложен в работе Чаттерджи и
Самуэльсона~\cite{chatterjee83}. Ими была рассмотрена модель двухстороннего
аукциона с неполной информацией, в котором цена сделки равна выпуклой комбинации
предложенных ставок с коэффициентом $\beta \in [0, 1]$. При этом в работе
Майерсона и Саттертвейтa~\cite{myerson83} показано, что при определенных
условиях механизм со значением $\beta = 1/2$ является оптимальным с точки зрения
максимизации дохода от торгов.

{\progress} В диссертации рассмотрены дискретные и непрерывные модели биржевых
торгов с симметричным механизмом торгов, отвечающим механизму Чаттерджи и
Самуэльсона, т.е. продажа актива по цене, равной выпуклой комбинации
предложенных ставок.

\aim\ Исследование повторяющихся игр с неполной информацией, моделирующих
биржевые торги с модифицированным механизмом торгов.

\researchsubject\ Объектом исследования являются математические модели
механизмов взаимодействия агентов на финансовых рынках. Предметом исследования
являются повторяющиеся игры с неполной информацией, моделирующие биржевые торги
между двумя различно информированными агентами.

Для~достижения поставленной цели необходимо было решить следующие {\tasks}:
\begin{enumerate}
\item Исследовать модель биржевых торгов с дискретными ставками и неограниченным
  количеством шагов. Определить влияние модифицированного механизма торгов на
  поведение агентов и результат торгов.
\item Исследовать модель биржевых торгов с непрерывными ставками в случаях
  конечного и бесконечного количества шагов. Сравнить оптимальное поведение
  агентов при использовании модифицированного механизма торгов с результатами
  оригинальной модели.
\item Обобщить результаты анализа модели с дискретными ставками на случай рынка
  со счетным множества возможных значений цены рискового актива \todo{и на
    случай торгов несколькими рисковыми активами}.
\end{enumerate}

\methods\ В диссертации применялись методы теории игр, выпуклого анализа, теории
двойственности и вариационного исчисления.

\novelty
\begin{enumerate}
\item Модель биржевых торгов с дискретными ставками и указанным механизмом
  торгов исследуются впервые. Содержательно обобщение оптимальной стратегии
  инсайдера в главе 1.
\item Исследование конечношаговой модели биржевых торгов с вещественными
  ставками на случай использования указанного механизма торгов проведено
  впервые. Содержателен результат о независимости значения соответствующей
  повторяющейся игры от конкретного вида механизма в главе 2.
\end{enumerate}

\influence\ Получено решение ряда повторяющихся игр с неполной информацией,
моделирующих биржевые торги с более общим механизмом торгов, что позволяет
оценить влияние конкретного вида механизма на оптимальное поведение агентов и
результат торгов.

\defpositions
\begin{enumerate}
  \item Решение бесконечной повторяющейся игры биржевых торгов с дискретными ставками и
    более общим механизмом торгов.
  \item Решение конечношаговой повторяющейся игры биржевых торгов с непрерывными
    ставками и более общим механизмом торгов.
  \item Обобщение результатов, полученных для модели с дискретными ставками, на
    случай рынка со счетным множеством возможных цен рискового актива.
\end{enumerate}

\reliability\ полученных в работе результатов обусловлена строгостью
формулировок задач и математических доказательств. Результаты находятся в
соответствии с результатами, полученными другими авторами.

\probation\ Основные результаты, полученные в диссертации, были представлены на
ежегодных научных конференциях в МГУ им. М.В. Ломоносова <<Тихоновские чтения>>
(2014) и <<Ломоносовские чтения>> (2016).

\ifthenelse{\equal{\thebibliosel}{0}}{%
  % Встроенная реализация с загрузкой файла через движок bibtex8
    \publications\ Основные результаты по теме диссертации изложены в XX печатных изданиях, 
    X из которых изданы в журналах, рекомендованных ВАК, 
    X "--- в тезисах докладов.%
  }{% Реализация пакетом biblatex через движок biber
    \begin{refsection}%
        \printbibliography[heading=countauthornotvak, env=countauthornotvak, keyword=biblioauthornotvak, section=1]%
        \printbibliography[heading=countauthorvak, env=countauthorvak, keyword=biblioauthorvak, section=1]%
        \printbibliography[heading=countauthorconf, env=countauthorconf, keyword=biblioauthorconf, section=1]%
        \printbibliography[heading=countauthor, env=countauthor, keyword=biblioauthor, section=1]%
        \publications\ По теме диссертации имеется \arabic{citeauthor}
        публикаций. %
        \nocite{pyanykh14, pyanykh16:discr:eng, pyanykh16:discr:ru,
          pyanykh16:cont, pyanykh16:countable, pyanykh:tikhon2014,
          pyanykh:lomonosov2016}%
        Основные результаты диссертационной работы опубликованы в
        \arabic{citeauthorvak} статьях из перечня ВАК, %
        \nocite{pyanykh14, pyanykh16:discr:eng, pyanykh16:discr:ru,
          pyanykh16:cont, pyanykh16:countable}%
        \arabic{citeauthorconf} "--- в тезисах
        докладов\nocite{pyanykh:tikhon2014, pyanykh:lomonosov2016}.
    \end{refsection}
}

%\publications\ Основные результаты по теме диссертации изложены в ХХ печатных изданиях~\cite{Sokolov,Gaidaenko,Lermontov,Management},
%Х из которых изданы в журналах, рекомендованных ВАК~\cite{Sokolov,Gaidaenko}, 
%ХХ --- в тезисах докладов~\cite{Lermontov,Management}.

% \publications\ По теме диссертации имеется 7 публикаций~\cite{pyanykh14,
%   pyanykh16:discr:eng, pyanykh16:discr:ru, pyanykh16:cont, pyanykh16:countable,
%   pyanykh:tikhon2014, pyanykh:lomonosov2016}. %
% Основные результаты диссертационной работы опубликованы в 5 статьях из перечня
% ВАК~\cite{pyanykh14, pyanykh16:discr:eng, pyanykh16:discr:ru, pyanykh16:cont,
%   pyanykh16:countable}.

%%% Local Variables:
%%% mode: latex
%%% TeX-master: "../dissertation"
%%% End:
 % Характеристика работы по структуре во введении и в автореферате не отличается (ГОСТ Р 7.0.11, пункты 5.3.1 и 9.2.1), потому её загружаем из одного и того же внешнего файла, предварительно задав форму выделения некоторым параметрам

%Диссертационная работа была выполнена при поддержке грантов ...

\emphasis{Объем и структура работы.}
Диссертация состоит из~введения, трех глав и заключения.
Полный объем диссертации составляет \todo{\textbf{ХХХ}}~страниц текста с~\todo{\textbf{ХХ}}~рисунками и~\todo{XXX}~таблицами.
Список литературы содержит \todo{\textbf{ХХX}}~наименование.

%\newpage
\section*{Содержание работы}
Во \emphasis{введении} обосновывается актуальность исследований, проводимых в рамках данной диссертационной работы, приводится обзор научной литературы по изучаемой проблеме, формулируется цель, ставятся задачи работы, сформулированы научная новизна, теоретическая и практическая значимость представляемой работы, а также результаты, выносимые на защиту.

% * Chapter 1
\emphasis{Первая глава} посвящена исследованию теоретико-игровой модели биржевых торгов с дискретными ставками и двумя состояниями.

% ** Основные понятия и введение в тему
В \emphasis{разделе 1.1} диссертации дано введение в теорию повторяющихся игр с неполной информацией, определены основные термины и понятия.
Теоретико-игровые постановки задач из глав~2 и~3 будут отличаться от данного классического описания, о чем будет сказано отдельно.

Рассмотрим антагонистическую игру двух лиц, которая повторяется $n$ раз, где $n\leq \infty$.
Будем считать, что первый игрок знает функцию выигрыша в данной игре, в то время как второй игрок такой информацией не обладает.
Однако второй игрок знает, что настоящая функция выигрыша является одной из $\kappa$ возможных альтернатив.
Каждой такой альтернативе второй игрок приписывает некоторую вероятность того, что данная функция выигрыша является истинной функцией выигрыша в рассматриваемой
игре.
Таким образом, априорные убеждения второго игрока задаются вероятностным вектором
$\p = (p_1, p_2, \ldots, p_\kappa),\ \sum_{i=1}^\kappa p_i = 1$.
С данными функциями выигрыша можно связать игры $G_1, G_2, \ldots, G_\kappa$.
В дальнейшем мы будем считать, что информационная неопределенность второго игрока заключается именно в том, что он не знает какая из игр $G_1, G_2, \ldots, G_\kappa$ разыгрывается.

На каждом шаге игры первый игрок может совершать действия из множества $I$, второй игрок --- действия из множества $J$, при этом мы считаем, что множества действий одного игрока известно другому.
Кроме того, положим, что второй игрок знает, что первый обладает точной информацией о том, какая именно игра разыгрывается, а первый игрок знает априорные убеждения второго.

Игры $G_1, G_2, \ldots, G_\kappa$ будем называть \emph{одношаговыми играми}.
Все одношаговые игры описываются матрицами размера $|I| \times |J|$, где элементы матрицы задают выплаты первому игроку.
Мы предполагаем, что оба игрока точно знают платежные матрицы игр $G_1, G_2, \ldots, G_\kappa$.
На каждом шаге игры первый игрок выбирает номер строки, и одновременно с ним второй игрок выбирает номер столбца.
В конце каждого хода действия игроков оглашаются, и элемент из матрицы, отвечающей настоящей игре, прибавляется к выигрышу первого игрока и вычитается из выигрыша второго.
Таким образом, первый игрок знает свой выигрыш на каждом этапе игры, в то время как второй может лишь рассчитать свой ожидаемый выигрыш.
В завершение, мы считаем, что данное описание известно обоим игрокам.

Данная игра с неполной информацией описывается в виде игры в нормальной форме следующим образом.
Обозначим через $S = \{1, 2, \ldots, \kappa\}$ множество возможных альтернатив или \emph{состояний природы}.
Перед началом игры ходом случая в соответствии с вероятностным распределением $\p$ выбирается состояние $s \in S$.
Далее на протяжении $n$ шагов разыгрывается игра $G_s$.
Первый игрок информирован о результате хода случая, второй игрок "--- нет.
В остальном правила данной игры совпадают с описанными выше.

Пусть $h_t = \left((i_1, j_1), (i_2, j_2), \ldots, (i_t, j_t)\right)$ "--- история ходов после завершения шага $t$.
Множество все таких $h_t$ обозначим через $H_t$. 

Стратегией первого игрока в такой игре является последовательность ходов (отображений) $\sigma = (\sigma_1, \sigma_2, \ldots, \sigma_n)$, где $\sigma_t = (\sigma^1_t, \sigma^2_t, \ldots, \sigma^\kappa_t)$, и $\sigma^s_t: H_{t-1} \rightarrow \Delta(I)$ "--- смешанная стратегия, зависящая от предыдущих ходов, которую первый игрок использует если ходом случая реализовалось состояние $s$.

Аналогичным образом, определим стратегию второго игрока как последовательность ходов (отображений) $\tau = (\tau_1, \tau_2, \ldots, \tau_n)$, где $\tau_t: H_{t-1} \rightarrow \Delta(I)$ "--- смешанная стратегия, зависящая от предыдущих ходов.
Как видно, ход второго игрока на каждом шаге игры зависит только от предыдущих ходов, и не зависит от состояния, в силу того, что второй игрок не информирован о результате хода случая.

Отметим, что, как показано в монографии Аумана, Машлера~\cite{aumann95}, достаточно рассматривать только стратегии, которые зависят лишь от предыдущих ходов первого игрока и не зависят от ходов второго.
%
Также нужно отметить, что в диссертации рассмотрены игры, в которых выигрыш равен суммарным выплатам, в отличие от постановки из \cite{aumann95}, в которых рассматривались игры с усредненными выплатами.

% ** Описание дискретной модели
В \emphasis{разделе 1.2} приводится описание дискретной модели биржевых торгов.
Следуя работе~\cite{domansky07}, Рассматривается упрощенная модель финансового рынка, на котором два игрока ведут торговлю однотипными акциями на протяжении $n \leqslant \infty$ шагов.

Перед началом торгов случайный ход определяет цену акции на весь период торгов, которая может быть либо $m \in \N$ с вероятностью $p$, либо $0$ с вероятностью $1-p$.
Таким образом определенный ход случая является упрощенным аналогом некоторого шокового события на финансовом рынке.
Такого, например, как публикация отчетов о доходах некоторой компании.
Выбранная цена сообщается первому игроку и не сообщается второму, при этом второй игрок знает, что первый "--- инсайдер.

Рассмотрим $t$-й шаг торгов, где $t = \overline{1,n}$.
На данном шаге первый игрок выбирает ставку $i_t \in I = \{0, 1, \ldots, m\}$, а второй --- ставку $j_t \in J = \{0, 1, \ldots, m\}$.
Игрок предложивший б\'{о}льшую ставку покупает у другого акцию по цене равной
\[
  \Co \max(i_t, j_t) + \DCo \min(i_t, j_t),
\]
где $\Co \in (0, 1),\ \DCo = 1 - \Co$.
Если ставки равны, то сделка на $t$-м шаге не состоится.
Коэффициент $\Co$ можно интерпретировать как \emph{переговорную силу продавца} "--- чем ближе значение к $1$, тем большую сумму получит продавец акции в результате сделки.

Считаем, что игроки обладают неограниченными запасами рисковых и безрисковых активов, т.е. торги не могут прекратиться по причине того, что у одного из игроков закончатся деньги или акции.
Цель игроков состоит в максимизации стоимости итогового портфеля, состоящего из некоторого числа купленных акций и суммы денег, полученных в результате торгов.
Таким образом, не ограничивая общности, можно положить, что в начальный момент времени оба игрока имеют нулевые портфели.

% * Вторая глава
\emphasis{Вторая глава} посвящена исследованию 

% * Третья глава
\emphasis{Третья глава} посвящена исследованию 

% * Conclusion
В \emphasis{заключении} приведены основные результаты работы, которые заключаются в следующем:
%% Согласно ГОСТ Р 7.0.11-2011:
%% 5.3.3 В заключении диссертации излагают итоги выполненного исследования, рекомендации, перспективы дальнейшей разработки темы.
%% 9.2.3 В заключении автореферата диссертации излагают итоги данного исследования, рекомендации и перспективы дальнейшей разработки темы.
\begin{enumerate}
  \item
    Получено решение повторяющейся игры с неполной информацией неограниченной продолжительности, моделирующей биржевые торги с дискретными ставками и более общим механизмом торгов.
    На основе анализа функции значения игры в зависимости от параметра $\Co$ механизма торгов показано, что механизмы, предписывающие продажу рискового актива по наибольшей или наименьшей цене гарантируют инсайдеру наименьший выигрыш.
  \item
    Найдено решение $n$-шаговой повторяющейся игры с неполной информацией, моделирующей биржевые торги с непрерывными ставками и более общим механизмом торгов.
    Показано, что, хотя стратегии инсайдера и неосведомленного игрока зависят от значения параметра $\Co$, значение $n$-шаговой игры от него не зависит, что существенно отличает непрерывный случай от дискретного.
  \item
    Получено обобщение результатов для дискретной модели на случай рынка со счетным множеством состояний.
    Показано, что введение более общего механизма торгов с параметром $\Co$ приводит к сдвигу решетки апостериорных вероятностей на $\Co$.
\end{enumerate}

%%% Local Variables:
%%% mode: latex
%%% TeX-master: "../dissertation"
%%% End:


%\newpage
% При использовании пакета \verb!biblatex! список публикаций автора по теме
% диссертации формируется в разделе <<\publications>>\ файла
% \verb!../common/characteristic.tex!  при помощи команды \verb!\nocite! 

\ifdefmacro{\microtypesetup}{\microtypesetup{protrusion=false}}{} % не рекомендуется применять пакет микротипографики к автоматически генерируемому списку литературы
\ifnumequal{\value{bibliosel}}{0}{% Встроенная реализация с загрузкой файла через движок bibtex8
  \renewcommand{\refname}{\large \authorbibtitle}
  \nocite{*}
  \insertbiblioauthor                          % Подключаем Bib-базы
  %\insertbiblioother   % !!! bibtex не умеет работать с несколькими библиографиями !!!
}{% Реализация пакетом biblatex через движок biber
  \insertbiblioauthor                          % Подключаем Bib-базы
  \insertbiblioother
}
\ifdefmacro{\microtypesetup}{\microtypesetup{protrusion=true}}{}


%%% Local variables:
%%% mode: latex
%%% TeX-master: "../synopsis"
%%% End:
