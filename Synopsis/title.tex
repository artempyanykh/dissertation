{\footnotesize
\thispagestyle{empty}

\begin{center}%
\MakeUppercase{\thesisOrganization}
\end{center}%

\vspace{0pt plus1fill} %число перед fill = кратность относительно некоторого расстояния fill, кусками которого заполнены пустые места
\begin{flushright}
  \large{На правах рукописи}
  % \includegraphics[height=1.5cm]{personal-signature} 
\end{flushright}

\vspace{0pt plus3fill} %число перед fill = кратность относительно некоторого расстояния fill, кусками которого заполнены пустые места
\begin{center}
\textbf {\large \thesisAuthor}
\end{center}

\vspace{0pt plus3fill} %число перед fill = кратность относительно некоторого расстояния fill, кусками которого заполнены пустые места
\begin{center}
\textbf {\Large \thesisTitle}

\vspace{0pt plus3fill} %число перед fill = кратность относительно некоторого расстояния fill, кусками которого заполнены пустые места
{\large Специальность \thesisSpecialtyNumber\ "---\par <<\thesisSpecialtyTitle>>}

\vspace{0pt plus1.5fill} %число перед fill = кратность относительно некоторого расстояния fill, кусками которого заполнены пустые места
\Large{Автореферат}\par
\large{диссертации на соискание учёной степени\par \thesisDegree}
\end{center}

\vspace{0pt plus4fill} %число перед fill = кратность относительно некоторого расстояния fill, кусками которого заполнены пустые места
\begin{center}
{\large{\thesisCity\ "--- \thesisYear}}
\end{center}

\newpage
% оборотная сторона обложки
\thispagestyle{empty}
\noindent Работа выполнена на \thesisInOrganization

\par\bigskip
\noindent%
\begin{tabularx}{\textwidth}{@{}lX@{}}
  Научный руководитель:   & \textbf{\supervisorFio}\par
                            кандидат физико-математических наук, доцент кафедры исследования операций факультета вычислительной математики и кибернетики Федерального государственного бюджетного образовательного учреждения высшего образования <<Московский государственный университет имени М.В.Ломоносова>>
                            \vspace{0.013\paperheight}\\
  Официальные оппоненты:  &
                            \textbf{\opponentOneFio,}\par
                            \opponentOneRegalia\par
                            \vspace{0.01\paperheight}
                            \textbf{\opponentTwoFio,}\par
                            \opponentTwoRegalia
                            \vspace{0.013\paperheight} \\
  Ведущая организация:    & \leadingOrganizationTitle
\end{tabularx}  
\par\bigskip

\noindent Защита диссертации состоится \defenseDate~на~заседании диссертационного совета \defenseCouncilNumber~в \defenseCouncilTitle~по адресу: \defenseCouncilAddress.
Желающие присутствовать на заседании диссертационного совета должны сообщить об этом за два дня по тел.~(495) 939-30-10 (для оформления заявки на пропуск).

\vspace{0.017\paperheight}
\noindent С диссертацией можно ознакомиться в Научной Библиотеке МГУ.
С текстом автореферата можно ознакомиться на официальном сайте факультета ВМК МГУ http://cs.msu.ru в разделе <<Диссертации>>.

% \vspace{0.017\paperheight}
% \noindent Отзывы на автореферат в двух экземплярах, заверенные печатью учреждения, просьба направлять по адресу: \defenseCouncilAddress, ученому секретарю диссертационного совета~\defenseCouncilNumber.

\vspace{0.017\paperheight}
\noindent{Автореферат разослан <<\_\_\_\_>> \_\_\_\_\_\_\_\_\_\_\_\_ 2016 года.}

\vspace{0.017\paperheight}
\par\bigskip
\noindent%
\begin{tabularx}{\textwidth}{@{}%
>{\raggedright\arraybackslash}b{18em}
>{\centering\arraybackslash}X
r
@{}}
    Ученый секретарь\par
    диссертационного совета\par
    \defenseCouncilNumber,\par
    \defenseSecretaryRegalia
    &
    % \includegraphics[width=2cm]{secretary-signature}
    &
    \defenseSecretaryFio
\end{tabularx} 

%%% Local Variables:
%%% mode: latex
%%% TeX-master: "../synopsis"
%%% End:
}
