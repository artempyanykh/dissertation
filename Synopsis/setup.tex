%%%%%%%%%%%%%%%%%%%%%%%%%%%%%%%%%%%%%%%%%%%%%%%%%%%%%%%
%%%% Файл упрощённых настроек шаблона автореферата %%%%
%%%%%%%%%%%%%%%%%%%%%%%%%%%%%%%%%%%%%%%%%%%%%%%%%%%%%%%

%%% Инициализирование переменных, не трогать!  %%%
\newcounter{tabcap}
\newcounter{tablaba}
\newcounter{tabtita}
%%%%%%%%%%%%%%%%%%%%%%%%%%%%%%%%%%%%%%%%%%%%%%%%%%

%%% Область упрощённого управления оформлением %%%


%% Подпись таблиц
\setcounter{tabcap}{0}              % 0 --- по ГОСТ, номер таблицы и название разделены тире, выровнены по левому краю, при необходимости на нескольких строках; 1 --- подпись таблицы не по ГОСТ, на двух и более строках, дальнейшие настройки: 
%Выравнивание первой строки, с подписью и номером
\setcounter{tablaba}{2}             % 0 --- по левому краю; 1 --- по центру; 2 --- по правому краю
%Выравнивание строк с самим названием таблицы
\setcounter{tabtita}{1}             % 0 --- по левому краю; 1 --- по центру; 2 --- по правому краю

%%% Цвета гиперссылок %%%
% Latex color definitions: http://latexcolor.com/
\definecolor{linkcolor}{rgb}{0.9,0,0}
\definecolor{citecolor}{rgb}{0,0.6,0}
\definecolor{urlcolor}{rgb}{0,0,1}
%\definecolor{linkcolor}{rgb}{0,0,0} %black
%\definecolor{citecolor}{rgb}{0,0,0} %black
%\definecolor{urlcolor}{rgb}{0,0,0} %black

\newtheorem{theorem}{Теорема}
\newtheorem{remark}{Замечание}
\newtheorem{lemma}{Лемма}
\newtheorem{corollary}{Следствие}

% Parts 1&2
\newcommand{\s}{\ensuremath{s}}
\newcommand{\q}{\ensuremath{\overbar{q}}}
\newcommand{\High}[1][\ensuremath{\infty}]{\ensuremath{H^\beta_{#1}}}
\newcommand{\sigmav}{\ensuremath{\overbar{\sigma}}}
\newcommand{\tauv}{\ensuremath{\overbar{\tau}}}
\newcommand{\xiv}{\ensuremath{\overbar{\xi}}}
\newcommand{\Low}[1][\ensuremath{\infty}]{\ensuremath{L^\beta_{#1}}}
\newcommand{\LL}{L^1(\{s^2\})}
\newcommand{\MM}{\ensuremath{\overline{P}}}
\newcommand{\Port}[1]{\ensuremath{\pi_{#1}}}

% Part 3
\newcommandx*\Ff[3][1={(f, Q)}, 2=n+1]{\ensuremath{F_{#2} \left(p, {#1}, {#3} \right)}}
\newcommand*\dualFPS{\ensuremath{(p,\ \sigma)}}
\newcommandx*\dualg[4][1=n, 2={\dualFPS}, 3=z, 4=\tau]{\ensuremath{g^*_{#1}({#3}, {#2}, {#4})}}
\newcommandx*\Gf[2][1=x, 2=\tau_1]{\ensuremath{G_{n+1} \left(z, {#1}, {#2} \right)}}
\DeclareDocumentCommand{\LV}{O{} m}{\ensuremath{\underline{V}^{#1}_{#2}}}
\DeclareDocumentCommand{\HV}{O{} m}{\ensuremath{\overbar{V}^{#1}_{#2}}}
\DeclareDocumentCommand{\HW}{O{} m}{\ensuremath{\overbar{W}^{#1}_{#2}}}
\DeclareDocumentCommand{\LW}{O{} m}{\ensuremath{\underline{W}^{#1}_{#2}}}